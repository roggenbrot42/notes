\documentclass[a4paper,11pt]{scrartcl}
\usepackage{amsmath}

\let\bb\mathbf
\newcommand{\SNR}{\mathop{SNR}}
\newcommand{\nrz}{\mathop{nrz}}
\newcommand{\rz}{\mathop{rz}}
\newcommand{\manc}{\mathop{manc}}
\newcommand{\rect}{\mathop{rect}}

\usepackage[T1]{fontenc}
\usepackage[utf8]{inputenc}

\begin{document}


\section{Einführung}

\subsection{ISO/OSI Modell}

Das ISO/OSI Modell bezeichnet das *Open Systems Interconnect* (OSI) Modell der
*International Standardization Organization* (ISO). Es unterteilt den
Kommunikationsvorgang im Internet in sieben Schichten, wobei jede Schicht
bestimmte Dienste erbringt. Die Schichten lauten wie folgt (von unten nach oben):

1. Physikalische Schicht (elektromagnetische Signale)
2. Sicherungsschicht (Ethernet, Direktverbindungen)
3. Vermittlungsschicht (IP)
4. Transportschicht (TCP/UDP)
5. Sitzungssschicht (Verbindungssitzungen, z.B. für TCP)
6. Darstellungsschicht (wie Daten Anwendungen präsentiert werden, z.B. JSON)
7. Anwendungsschicht (HTTP, DNS)

Innerhalb dieses Modells sprechen wir von *vertikaler* und *horizontaler*
Kommunikation:

* Vertikale Kommunikation beschreibt den Vorgang, wie Daten zwischen den
  Schichten übermittelt wird. Also beim Senden, die Übertragung der Daten "nach
  unten", und beim Empfangen, das Übertragen der Daten "nach oben".
* Horizontale Kommunikation meint, wie Schichten jeweils miteinander, bei
  Empfänger und Sender, kommunizieren. Beispielsweise spricht man von
  horizontaler Kommunikation, wenn die Vermittlungsschicht Daten sendet, die
  zur Interpretation der Vermittlungsschicht beim Empfänger gedacht sind.

Man beachte, dass das ISO/OSI Modell auch Schwächen hat. So kann man
beispielsweise nicht alle Dienste nur einer Schicht zuordnen, sondern oft
mehreren. Auch widerspricht die Anordnung der Schichten oft anderen Interessen,
beispielsweise der Effizienz der Übertragung.

\subsubsection{ Vertikale Kommunikation}

Es gibt nun ein Modell für die verschiedenen Bausteine, die Schichten bei der
vertikalen Kommunikation untereinander austauschen. Wir betrachten hierzu die
$N$-te Schicht, welche Informationen aus der vorherigen, höheren $N+1$-ten
Schicht erhält und seine eigenen Informationen an die nächste, niedrigere
$N-1$-te Schicht sendet.

1. Die $N$-Schicht erhält aus der oberen Schicht eine *Interface Data Unit*
   (IDU). Diese besteht aus *Interface Control Information* (ICI) sowie den
   eigentlichen Nutzdaten, der *Service Data Unit* (SDU).
2. Zunächst wird die ICI gelesen, welche Informationen der $N+1$-Schicht für die
   $N$-Schicht enthält. Diese ICI wird danach nicht weiter gesendet, sie ist
   also nur für diese Kommunikation zwischen den zwei Schichten notwendig.
3. Die SDU, also die eigentlichen, interessanten Nutzdaten, werden dann von der
   $N$-Schicht in eine Protocol Data Unit (PDU) verpackt.
4. Mit dazu kommt noch ein *Protocol Control Information* Block,
welcher Daten zur *horizontalen Kommunikation* enthält. Diese sollen also beim
Empfänger auch auf der $N$-Schicht interpretiert werden. Beispielsweise könnte
hier enthalten sein, dass das TCP Protokoll verwendet wird. Dann weiß die
Transportschicht beim Empfänger, dass sie ein TCP und kein UDP Paket entpacken
soll. Diese Information nennt man üblicherweise *Header*.
5. Sobald die PDU aus der SDU und den PCI fertig gebaut wurde, fügt die
   $N$-Schicht noch ihre eigenen ICI für die $N$-Schicht hinzu.
6. Das Resultat ist eine neue IDU für die $N-1$-Schicht.

Es gibt für die PDU auf manchen Schichten spezielle Namen:

* Transportschicht (4): Segment (TCP), Datagramm (UDP)
* Vermittlungsschicht (3): Paket
* Sicherungsschicht (2): Rahmen

\section{ OSI Modell: Layer 1}

%$\newcommand{\SNR}{\mathop{\rm SNR}\nolimits}$
%$\newcommand{\nrz}{\mathop{\rm nrz}\nolimits}$
%$\newcommand{\rz}{\mathop{\rm rz}\nolimits}$
%$\newcommand{\manc}{\mathop{\rm manc}\nolimits}$
%$\newcommand{\rect}{\mathop{\rm rect}\nolimits}$

\subsection{ Signale und Information}

Ein *Signal* ist eine *zeitabhängige*, messbare physikalische Größe. Dabei kann
man definierten messbaren *Signaländerungen* ein *Symbol* zuordnen. Diese
Symbole repräsentieren dann Information. Beispiele für Signale wären:

* Licht (Morsezeichen)
* Spannung (Telegraphie)
* Schall (Menschliche Sprache)

\subsubsection{ Informationstheorie}

\subsection{ Informationsgehalt}

Betrachten wir eine Informationsquelle (ein Signal) sowie ein konkretes Symbol,
dass von dieser Quelle emittiert wird. Nehmen wir nun an, die Quelle emittiert
nur dieses eine Symbol. Wenn wir nun dieses eine Symbol von der Quelle erhalten,
haben wir dann daraus irgendwelche interessanten Informationen gewonnen? Nein,
denn wir haben es ja erwartet, da die Quelle nur dieses Symbol emittiert. Es
sagt uns also nichts Neues, somit ist der *Informationsgehalt* dieses Symbols
gering (bzw. Null). Nehmen wir nun an, die Quelle emittiert noch ein weiteres
Symbol (Zeichen), welches aber nur sehr, sehr selten vorkommt. Wenn wir nun
dieses seltene Symbol aus der Quelle erhalten, dann muss das ein besonderes
Ereignis sein. Es enthält also mehr Information, über irgendein besonderes
Geschehnis. Insofern wäre der Informationsgehalt dieses Symbols also größer. Die
weniger seltenen Symbole würden auf "nichts Neues" schließen lassen, also
wiederum wenig Information enthalten.

Dieses Konzept kann man nun auch formalisieren. Wir betrachten hierzu die
Wahrscheinlichkeitsverteilung $p(x)$, welche angibt, wie wahrscheinlich es ist,
dass das nächste emittierte Symbol zu einem beliebigen Zeitpunkt gerade $x \in
\mathbf{X}$ ist, wo wir mit $\mathbf{X}$ die Menge aller möglichen Symbole
bezeichnen. Wir wollen nun ausdrücken, dass wenn die Wahrscheinlichkeit für ein
Symbol hoch, sein Informationgehalt niedrig ist und umgekehrt. Dafür verwenden
wir nun die Logarithmusfunktion, welche die Wahrscheinlichkeit sozusagen
invertiert. Für niedrige Werte, also kleine Wahrscheinlichkeiten $p(x)$ bei
seltenen Symbolen, wird $\log(p(x))$ ein größerer Wert sein, als bei einem
großen $p(x)$ bei häufiger auftretenden Symbolen. Der mathematische Hintergrund
hierfuer ist, dass kleine Wahrscheinlichkeiten hoehere bzw. negativere
Zehner-Potenzen sind. So ist $\log_{10}(10^{-10}) = -10$ und $\log_{10}(10^{-2})
= -2$. Da für Werte kleiner Eins der Logarithmus bei jeder Basis immer negativ
ist, negieren wir für den eigentlichen Informationsgehalt $I(x)$ eines Symbols
noch den Wert des Logarithmus: $I(x) = -\log(p(x))$. Hierbei ist also
$-\log(p(x)) = |\log(p(x))|$. Es gilt im Uebrigen, dass $\log(1/x) = -\log(x)$
fuer $x \in \mathbf{R}$.

Wir können dabei die Basis des Logarithmus prinizipell selbst wählen, wobei sich
die exakten Werte sowie Einheiten jedoch verändern würden. Wählen wir den
Logarithmus Dualis mit Basis 2, so erhalten wir den Informationsgehalt in
*Bits*. Wählen wir den natürlichen Logarithmus $\ln$, geben wir die Werte in
*Nats* an. Wir verwenden in der Vorlesung die Basis 2.

Was uns der Wert des Informationsgehalts noch sagt, ist, wieviele Bits man bei
einer idealen Quelle benötigen würde, um ein Symbol zu kodieren. Emittiert ein
Symbol immer das selbe Zeichen, so ist der Informationsgehalt des Zeichens
gleich null. Das bedeutet, dass wir prinzipiell das Symbol gar nicht erst zu
senden brauchen, weil es sowieso eindeutig ist, dass es als nächstes emittiert
würde. Emittiert eine Quelle aber viele Symbole, so würde man für häufige
Symbole schon auch Bits brauchen, um sie zu kodieren. Die seltenen Symbole
müsste man hierbei aber mit *mehr Bits* kodieren. Man stellt sich hierfür am
bestene eine unäre Kodierung der Symbole vor. Die häufigen Symbole kodiert man
mit wenigen Bits (/Strichen -> Unär), da man sie über die Quelle häufiger senden
muss. Die seltenen Symbole verteilen wir auf mehr Bits, welche teurer zu
übertragen sind.

\paragraph{ Entropie}

Die *Entropie* $H(\mathbf{X})$ einer Informationsquelle $Q$ beschreibt nun den
*mittleren* Informationsgehalt aller von dieser Quelle emittierten Symbole
$\mathbf{X}$. Dieser Wert wird also berechnet, indem man die Wahrscheinlichkeit
$p(x)$ jeden Symbols $x$ mit seinem Informationsgehalt $I(x)$ multipliziert, und
so über alle Symbole $x$ summiert:

$H(\mathbf{X}) = \sum_{x \in \mathbf{X}} p(x)I(x) = -\sum_{x \in \mathbf{X}}
p(x)\log(p(x))$

Wir berechnen also gerade den Erwartungswert $E_{x \sim p}[I]$ des
Informationsgehalts der Symbole. Der resultierende Wert, welcher ebenso in Bits
gemessen wird, beschreibt zwei Dinge:

* Die Unsicherheit bezüglich Veränderungen des Signals. Je höher die Entropie,
  desto höher ist also der gesammelte Informationsgehalt der Symbole bzw. desto
  uniformer ist auch die Wahrscheinlichkeitsverteilung $p$. Erhalten wir also
  eine hohe Entropie, bedeutet das, dass wir uns weniger sicher sein können,
  welches Symbol denn als nächstes von der Quelle emittiert wird. Ist die
  Entropie niedrig, können wir schon eher erahnen, wie die Quelle sich verhalten
  wird. Die Quelle hätte also insgesamt geringeren Informationsgehalt.
* Die durchschnittiche Anzahl an Bits, mit welchen man Symbole in einem idealen
  Code für diese Quelle kodieren müsste.

Man kann die Entropie in der Informationstheorie auch gut mit der Entropie in
der Thermodynamik beschreiben. Die Entropie eines thermodynamischen Systems
sagt aus, wieviele verschiedene Konfigurationen der Moleküle in dem System es
gibt. In einem unruhigen System, beispielsweise Gas, ist die Entropie hoch, da
die Gasmoleküle sich auf sehr viele Weisen konfigurieren können. Die
*Unsicherheit* über die nächste Konfiguration ist also hoch, ebenso wie die
Entropie einer Informationsquelle die Unsicherheit bezüglich dem als nächstes
emittierten Symbol beschreibt. In einem Kristall sind Moleküle hingegen starr.
Ihre Entropie wäre also niedrig, weil es nur eine Konfiguration des Systems
gibt.

\paragraph{ Maximale Entropie}

Wir fragen uns: Wie hoch kann die Entropie einer Quelle $Q$ mit Signalen
$\mathbf{X}$ maximal sein?

Zunaechst: wie hoch muessen die Wahrscheinlichkeiten sein? Intuitiv ist klar:
sie muessen alle gleich sein, die Wahrscheinlichkeitsverteilung muss also
uniform sein. Dann ist die Unsicherheit darueber, welches Symbol die Quelle als
naechstes emittiert, gerade am groessten, weil jedes der vielen Symbole gleich
wahrscheinlich ist. Mathematisch kann man es natuerlich durch Ableiten finden.

Wenn wir nun also wissen, dass die Wahrscheinlichkeit fuer eine maximale
Entropie uniform, also $1/n$ bei $n$ Signalen, sein muss, koennen wir die
Entropieformel leicht umformen:

$H_{max}(\mathbf{X}) = -\sum_{x \in \mathbf{X}} p(x)\log(p(x)) = -\sum_{x \in
\mathbf{X}} 1/n \cdot \log(1/n) = 1/n \cdot (-(n \cdot \log(1/n))) = -\log(1/n) =
\log(n)$

Die maximale Entropie $H_{max}$ ist also gerade immer gleich $\log(n)$, wo $n$
die Anzahl moeglicher Symbole ist. Da die Entropie ja den durchschnittichen
Informationsgehalt angibt, und dieser bei einer uniformen Wahrscheinlichkeit
jedes Symbols immer gleich ist, ist in diesem Fall die maximale Entropie auch
gleich dem Informationsgehalt jedes Symbols. Das bedeutet also, dass wenn wir
von maximale Entropie ausgehen, uns $\log(n)$ Bits genuegen, um alle unsere
Symbole zu kodieren.

Umgangssprachlich meint man ja beispielsweise oft, dass man fuer 256
verschiedene Werte eben 8 Bit braucht. Das ist aber theoretisch nur der Fall,
wenn jedes Symbol die selbe Auftrittswahrscheinlichkeit hat, sodass die Entropie
maximal ist. Denn nur dann ist $H(Q) = H_{max}(Q) = \log(|Q|)$

\subsection{ Klassifizierung von Signalen}

Wenn wir nun ein konkretes Signal $s(t)$ in Abhängigkeit des Zeitfortschritts
$t$ betrachten, müssen wir uns noch entscheiden, welchen Werten oder
Veränderungen des Signals wir welche Symbole zuordnen. Wir brauchen also eine
Abbildung vom momentanen Signalwert auf ein Symbol. Wir könnten für ein binäres
Alphabet $\mathbf{X} \in \{0, 1\}$ beispielsweise einfach die folgende Abbildung
wählen:

$x_t = \begin{cases}0, \text{ wenn } s(t) \leq 0\\1, \text{
sonst}\end{cases}.$

Jetzt stellen sich aber viele Fragen:

* In welchen Zeitabständen betrachten wir jeweils das Signal, um ein neues
  Symbol aufzunehmen? Theoretisch gäbe es für ein beliebig großes Intervall
  unendlich viele Symbole, da es unendlich viele Zeitpunkte $t$ in einem solchen
  kontinuierlichen Zeitintervall gibt. Da wir sicher nur endlich viele Symbole
  aufnehmen wollen (und auch speichern können), müssen wir die Zeit also
  *diskretisieren* und festlegen, wann wir *Samples* nehmen.
* Erhalten wir mehr Information, wenn wir öfter *abtasten*, also öfter Samples
  aufnehmen?
* Da bei einem Signal nicht nur die Zeit, sondern auch der Wertebereich
  (beispielsweise bei elektrischer Spannung) kontinuierlich sein kann, müssen
  wir wohl auch hier den Wertebereich diskretisieren oder abtasten. Den Begriff
  abtasten benutzt man aber nur für die Zeitachse. Für Signalwerte, also die
  Wertachse, nennen wir solches Abrunden von kontiniuerlichen Werten auf endlich
  viele, diskrete Intervalle (Bins/Buckets) *Quantisierung*.
* Auf welche Weise sollen längere Folgen von Symbolen interpretiert werden?
  Möglicherweise wollen wir nicht nur einzelne Zeichen senden, sondern
  irgendwelche Arten von *Paketen*, welche sowohl Daten als auch andere
  Informationen enthalten könnten (z.B. Steuerinformationen).
* Wie behandeln wir die Tatsache, dass Signale in der echten Welt nicht immer
  ideal, sondern oft verrauscht, gedämpft oder verzerrt sind?
* Wie wird Fehlern bei der Übertragung vorgebeugt?
* Wie wird ein Signal überhaupt erzeugt und von $A$ nach $B$ gesandt?

\subsubsection{ Zeit- und Frequenzbereich}

Nach dem *Fourier-Theorem* lassen sich alle periodischen Zeitsignale als
Überlagerungen von gewichteten Sinus- und Kosinusschwingungen unterschiedlicher
Frequenzen auffassen. Haben wir dabei ein Signal $s(t)$ in Abhaengigkeit der
Zeit $t$, so koennen wir nach Fourier dieses Signal durch eine *Fourier-Reihe*
ausdrucken:

$s(t) = \frac{a_0}{2} + \sum_{k=1}^\infty (a_k \cos(k\omega t) + b_k \sin(k
\omega t))$

Hierbei bezeichnet man das $k$-te Summenglied auch als die $k$-te
Harmonische. $a_k$ und $b_k$ sind dann die Gewichtungen fuer die Sinus- und
Kosinuskurven, welches nach Fourier diese $k$-te Harmonische ausmachen.  Das
konstante Glied $\frac{a_0}{2}$ wird hierbei auch *Gleichanteil* genannt, und
drueckt eine Verschiebung der Signalamplitude bezueglich der Ordinate (y-Achse)
aus. Es ist gewissermassen der Mittelwert der Funktion, zu welchem die
Schwingung immer wieder zurueckkehrt. $\omega$ drueckt in der Formel die
*Kreisfrequenz* $2\pi f$ bzw. $2\pi/T$ aus und normalisiert die Periode der
Schwingung auf das Intervall $T$. Ohne diese Kreisfrequenz $\omega$ wurde die
Schwingung eine Periode von $2\pi$ haben.

Kennen wir das Signal $s(t)$ schon, so koennen wir die Koffizienten $a_k$ und
$b_k$ des $k$-ten Gliedes berechnen. Wenn man das fuer alle $k \in \{1, ...,
\infty\}$ macht, so nennt man das eine *Fourier-Transformation*. Konkret
existieren folgende Formeln fuer das Kosinusgewicht $a_k$ und Sinusgewicht
$b_k$:

$a_k = \frac{2}{T}\int_0^T s(t)\cos(k\omega t) \mathrm{dt} \text{ und } b_k =
\frac{2}{T}\int_0^T s(t)\sin(k\omega t) \mathrm{dt}$

Der Gleichanteil ist hierbei $a_0/2$ und nicht $a_0$, um negative Frequenzen
unter Betracht zu ziehen. Negative Frequenzen gibt es nicht wirklich, sie sind
eher nur ein mathematisches Konstrukt das uns mathematische Berechnungen, vor
allem mit komplexen Zahlen, erleichtert. Wir addieren sie in der Praxis einfach
zu den positiven Frequenzen (klappen das Spektrum am 0-Punkt um). Deswegen ist
der Gleichanteil, symbolisch bzw. zur Veranschaulichung, $a_0/2$ und nicht
$a_0$. *Beachte also*: $a_0/2$ ist der Gleichanteil (wirklich die Verschiebung
entlang der Ordinate), $a_0$ ist lediglich der erste Koeffizient.

\subsubsection{ Diskretisierung}

Natuerliche Signale sind sowohl *zeitkontiniuerlich* (es gibt fuer jedes
Intervall ueberabzaehlbar unendlich viele Zeitpunkte $t$) sowie auch
*wertkontinuierlich* (das Signal kann in jedem Intervall des Wertebereichs
ueberabzaehlbar unendlich viele Werte aus diesem Intervall annehmen). Wir
koennen auf einem Computer aber natuerlich nur endlich viele Symbole speichern
und bearbeiten, da ein Computer nur endlich viel Speicher und auch nur endliche
Rechengenauigkeit hat.

Wir muessen Signale also diskretisieren. Im Zeitbereich nennt man das
*Abtastung* und im Wertbereich *Quantisierung*. Ist ein Signal erst einmal in
beiden Bereichen diskretisiert, so koennen wir es *digital* speichern. Wir
koennen dann also fuer jedes diskrete Zeitintervall $[t_i, t_i + 1]$ einen
diskreten Wert festhalten und in einem Datenwort fester Laenge (z.B. 16 Bits)
speichern.

\paragraph{ Abtastung}

Beim Abtasten legen wir ein Abtastintervall $T_a$ fest, sodass wir immer, wenn
der kontinuierliche Zeitwert $t$ ein ganzzahliges Vielfaches von $T_a$ ist, den
entsprechenden Wert des Signals $s(t)$ zu diesem Zeitpunkt erhalten. Wenn $t$
aber kein ganzzahliges Vielfaches von $T_a$ ist, so soll unsere *Abtastfunktion*
eine Null zurueckgeben. Wir bezeichnen dabei die Abtastfunktion fuer das Signal
$s(t)$ durch $\hat{s}(t)$. Sie ist wie folgt definiert:

$\hat{s}(t) = s(t)\sum_{n=-\infty}^\infty \delta[t - nT_a]$

Hierbei ist $\delta[t]$ der *Dirac*- oder *Delta-Impuls*, welcher genau fuer $t
= 0$ einen Wert von $1$ annimmt, und sonst immer $0$. Er ist nuetzlich, um Werte
einer kontinuierlichen Funktion nach irgendeiner Bedingung "auszuwaehlen". Das
ist auch gerade was wir hier machen wollen. Betrachten wir zum Verstaendnis
dieser Funktion einen konkreten Zeitpunkt $t$. Wir berechnen zuerst den
Signalwert $s(t)$ des kontinuierlichen Signals $s$, welches wir diskretisieren
wollen, zum diesem Zeitpunkt $t$. Dann gibt die Summe konzeptuell alle
Zeitpunkte $nT_a$ an, zu welchen wir abtasten wollen. Indem wir in die Dirac
Funktion $t - nT_a$ einsetzen, erhalten wir folgende Situation: wenn $t$ gerade
einer dieser Abtastpunkte $nT_a$ ist, dann wird $\delta[t - nT_a] = \delta[0] =
1$ sein und der Signalwert wird aufgenommen. Hierbei werden dann aber auch alle
anderen Summenglieder $0$ sein, da fuer alle $m \neq n$ gelten wird: $t - mT_a
\neq 0$. Wenn $t$ aber mit keinem dieser Abtastpunkte uebereinstimmt, dann wird
die Summe fuer alle Glieder null sein, und $s(t) \cdot 0$ wird uns also keinen
Signalwert geben. Der Koeffizient fuer $s(t)$, also die ganze Summe, ist also
eins, immer dann wenn $t = nT_a$ fuer irgendein $n \in \mathbf{Z}$, und sonst
Null. Wir probieren also fuer jeden Zeitpunkt $t$ alle moeglichen $nT_a$ von
minus unendlich bis plus unendlich aus (mathematisch ist das ja leicht
realisierbar). Wenn eines passt, wird der Wert aufgenommen. Sonst nicht.

So erhalten wir also eine mathematische Darstellung davon, was es bedeutet, ein
Signal bezuglich der Zeit zu diskretisieren. Wir benutzen im Weiteren dann, wenn
fuer $\hat{s}(t)$ gerade gilt $t = nT_a: n \in \mathbf{Z}$ die Notation
$\hat{s}[n]$. Dann ist also $\hat{s}[n] = \hat{s}(nT_a)$.

\paragraph{ Rekonstruktion}

Haben wir erst einmal die Abtastwerte, so ist es unter Umstaenden moeglich, das
urspruengliche Signal wiederherzustellen -- also zu rekonstruieren. Hierbei sagt
uns das Nyquist-Shannon Abtasttheorem (Nyquist-Shannon Theorem oder einfach
Nyquist Theorem), dass wenn unser Signal auf eine Bandbreite $B$ begrenzt, ist
sodass alle Frequenzen $f$ der Sinus- und Kosninuskomponenenten kleiner gleich
$B$ sind, erst eine Abtastrate

$f_a \geq 2B$

eine korrekte Rekonstruktion ermoeglicht. Die Abtastrate muss also groesser als
die groesste vorkommende Frequenz im Signal sein. Analog kann man sagen, dass
die Abtastperiode oder das Abtastintervall $T_a$ zwei mal so klein sein muss,
wie die kleinste Periode im Signal. Die Intuition dahinter ist, dass wir fuer
ein Signal mindestens einen Punkt ueber dem Equilibrium und einen Punkt unter
dem Equilibrium (aber noch in der selben Periode) benoetigen, um das Signal
richtig zu rekonstruieren.

Waehlen wir eine Abtastfrequenz, die kleiner als die hoechste Frequenz im Signal
ist, so kann es vorkommen, dass wir anstatt dem urspruenglichen Signal ein
anders rekonstruieren. Dieses faelschlicherweise rekonstruierte Signal haette
also gerade die selben Abtastpunkte, wie das urspruengliche Signal (welches wir
nicht darstellen konnten). Daher spricht man bei dem falschen Signal von einem
*Alias* des urspruenglichen Signal. Deswegen sagt man, dass wenn die
Abtastfrequenz zu niedrig gewaehlt wurde, das Phaenomen des *Aliasing* auftritt,
wobei Signale eben durch ihre Aliase "ersetzt" werden.

Sehen wir uns das Frequenzspektrum an, so koennen wir die Folgen von Aliasing
auch erkennen. Hierzu sei zuerst erklaert, wie sich Abtastung auf das
Frequenzspektrum eines Signals auswirkt. Beschraenken wr unser Signal auf eine
Bandbreite von maximal $B$ Hertz, so haetten wir fuer das ursprunegliche,
kontinuierliche Signal also genau ein einziges Segemnt im Frequenzspektrum, wo
die Amplitude nicht null ist. Da es positive wie auch negative Frequenzen gibt
(zumindest im Theoretischen), ist dieses Segment eigentlich $2B$ Hertz breit
($B$ positive und $B$ negative Hertz). So hat eine Abtastung mit einer Rate von
$f_a$ Hertz zur Folge, dass dieses $2B$-Hertz breite nach links und nach rechts
ins Unendliche um jeweils $f_a$ Hertz verschoben wird. Waehlen wir nun $f_a$
kleiner als $2B$, so passiert es, dass diese Verschiebungen miteinander
*ueberlappen*. Durch Ueberlagerung der einzelnen Frequenzamplituden ergibt sich
ein neues Frequenspektrum, welches aber nun ident zu dem ist, welches wir
erhalten haetten, wenn wir den Alias des Signals korrekt abgetastet haetten.

Man erkennt aus der Analyse des Frequenzbereichs auch, wieso $f_a$ groesser
*zwei* mal der hoechsten Frequenz sein muss, und nicht groesser der hochsten
Frequenz. Wenn wir $f_a = f_{max}$ waehlen, so verschieben wir unser auf
$f_{max}$ beschraenktes Spektrum also um jeweils $f_a$ nach links und rechts. Da
wir aber auch negative Frequenzen, zumindest im Theoretischen, betrachten
muessen, wurden sich genau bei $f_{max}$ der positive Teil des Spektrums mit dem
negativen Teil des ersten nach rechts verschobenen Segments ueberlappen, und die
Amplitude waere verdoppelt.

\paragraph{ Quantisierung}

Unter *Quantisierung* versteht man die Diskretisierung des urspruenglich
kontiniuerlichen Wertebreichs eines Signals. Betrachtet man beispielsweise das
Sinussignal $\sin(x)$, so kann dies uaberabzaehlbar unendlich viele Werte aus
dem Intervall $[-1, 1]$ annehmen. Wir koennen aber mit unseren guten 64-Bit IEEE
754 double-precision floating-point werten aber niemals mit perfekter Praezision
jeden infinitesimal kleinen Wert aus diesem Intervall darstellen. Folglich
muessen wir uns auch hier ausdenken, wie wir diesen Wertebereich in diskrete
Stufen aufteilen.

Allgemein speichern wir Datenwoerter natuerlich mit einer bestimmten Bit-Breite
$N$. Dann koennen wir also maximal $M = 2^N$ Symbole aus einem Wertebereich
darstellen. Nun fragt es sich noch, wie wir diese Symbole auf den Wertebereich
*sinnvoll* verteilen.

Die einfachste Moeglichkeit nennt man *lineare Quantisierung*. Wenn wir einen
Wertebereich $[a, b]$ haben, dann teilen wir diesen Wertebereich einfach in $M -
1$ aequidistante Intervalle der Groesse $\Delta = \frac{b - a}{M}$, wobei die
diskreten Werte dann die $M$ Intervallgrenzen sind. Wenn wir immer zum naechsten
Quantum (Intervallgrenze) runden, ergibt sich dabei ein maximaler
*Quantisierungsfehler* $q_{max}$ von $\Delta/2$. Man beachte natuerlich, dass
dies nur fuer Werte $\in [a, b]$ gilt. Ausserhalb ist der Quantisierungsfehler
unbeschraenkt.

Ausserdem beachte man, dass man bei der linearen Quantisierung die diskreten
Werte im optimalen Fall gar nicht an die Intervallsgrenzen gibt, sondern um
$\Delta/2$ nach innen "staucht". Sonst haette man genau im Nullpunkt naemlich
einen Quantisierungsfehler von $\Delta$, da die naechsten Intervallgrenzen bei
$+\Delta$ und $-\Delta$ waeren (im Nullpunkt selbst ist kein diskreter
Wert). Dadurch, dass wir die Werte um $\Delta/2$ nach innen verschieben, ist der
Quantisierungsfehler in jedem Intervall maximal $\Delta/2$.

Als Beispiel in der echten Welt: Telefonverbindungen quantisieren nicht
gleichmaessig (linear). Eher erhalten niedrigere Amplituden mehr Intervalle als
hoehere, weil es sich auf Grund der Eigenheiten des menschlichen Hoerspektrums
lohnt, fuer niedrigere Frequenzen eine hoehre Aufloesung zu haben. Bestimmte
Wertebereiche haben also mehr Quantisierungstufen als andere.

\subsection{ Uebertragungskanal}

In der realen Welt wird ein Signal nicht unbeeinflusst von $A$ nach $B$
gesendet. Es kommen noch Stoerfaktoren hinzu, zum Beispiel:

* Daempfung: Die Signalamplitude kann waehrend der Uebertragung reduziert
  werden.
* Tiefpassfilterung: Hohe Frequenzen gehen verloren.
* Verzoegerung: Die Uebertragung benoetigt eine gewisse Zeit.
* Rauschen: In Form von Additive White Gaussian Noise.

Wir betrachten hierfuer ein vereinfachtes Uebetragungsmodell:

``` x --- [Kanal] --- + --- y | n ```

Hierbei ist $n$ eine Stoerquelle (eigentlich $\eta$).

\subsubsection{ Kanalkapazitaet}

Wir nehmen an, dass ein Kanal wie ein Tiefpassfilter wirkt. Das heisst,
Frequenzen ab einer bestimmten Threshold/Cutoff Frequenz $f_0$ werden
gedaempft. Man kann also gewissermassen von einer *Kanalbandbreite* $B$
sprechen. Dann gilt:

* Frequenzen unter der Bandbreite (bzw. darin) gehen ungehindert durch den
  Kanal.
* Hoehere Frequenzen werden gedaempft.
* Ab einer bestimmten Frequenz sind Signale dann vernachlaessigbar klein.

Vereinfacht nehmen wir dabei an, dass die Grenze bei $B$ scharf ist, sprich der
Filter ist perfekt. Dann werden also alle Frequenzen $f$ mit $|f| < B$
durchgelassen, und alle Frequenzen $|f| \geq B$ werden vollstaendig gesperrt,
erhalten also eine Amplitude von Null.

https://en.wikipedia.org/wiki/Bandwidth\_(computing)
https://en.wikipedia.org/wiki/Goodput

\paragraph{ Rauschfreier, Binaerer Kanal}

Das Nyquist Theorem sagt uns, dass wir fuer ein auf $B$ bandbegrenztes Signal
eine minimale Abtastfrequenz von $2B$ brauchen. Das bedeutet aber auch, dass
$f_a = 2B$ (unter der Annahme perfekter Interpolationsalgorithmen) auch
ausreicht, um ein solches bandbegrenztes Signal vollstaendig zu
rekonstruieren. Insofern erhalten wir also durch eine hoehere Abtastrate auch
keine neuen Informationen, da wir mit $f_a = 2B$ schon ausreichend schnell
abtasten. Anders kann man sagen, dass wir in einer bestimmten Zeiteinheit also
auch nicht mehr als $2B$ unterscheidbare und *unabhaengige* Werte aufnehmen
koenenn. Das wichtige ist hierbei *unabhaengig*. Wir koennen natuerlich mit
$100B$ fuenfzig mal mehr Samples aufnehmen. Da wir aber mit $f_A$ das Signal
schon vollkommen rekonstruieren koennen, sind diese zusaetzlichen Samples
sozusagen alle von den $2B$ abhaengig und insofern redundant. Wir brauchen
hierbei dann immer zwei Samples pro Signalaenderung, um den Zustand sowohl des
Sinus- als auch Kosinusanteils des Signals zu erfassen, da das Signal ja eine
Mischung aus beiden sein kann.

Deswegen gibt uns die Nyquist Rate $f_N = 2B$ zwei Informationen ueber einen
\_\_binaeren\_\_ Kanal:

* Die minimale Abtastfrequenz, um das Signal korrekt zu rekonstruieren.
* Die maximale Anzahl an unterscheidbaren und unabhaengigen Signalwerten.

\paragraph{ Rauschfreier, $M$-aerer Kanal}

Wir nehmen nun an, dass unsere Quelle mehr als nur zwei Symbole emittieren kann,
naemlich $M = 2^N$ viele. Dann gilt, dass die Kanalkapazitaet des Kanals gleich
$f_N \cdot H(Q)$ ist, wo $H(Q)$ die Entropie der Quelle ist. Wir wissen nun
aber, dass die Entropie einer Quelle maximal $\log(M) = N$ ist. Somit sagt uns
*Hartley's Gesetz*, dass die maximale Kanalkapazitaet, also die maximale Anzahl
an Symbolen die wir in einem Zeitintervall senden koennen, gleich

$C_H = f_N\log(M)$

ist. Wir wollen dies intuitiv verstehen. Wir wissen, dass wir bei einem binaeren
Kanal maximal $f_N = 2B$ Symbole ueber einen auf $B$ bandbegrenzten Kanal senden
koennen. Jetzt haben wir aber mehr als zwei moegliche Symbole
bzw. Signalzustaende. Wenn die Entropie der Quelle maximal ist, wissen wir, dass
wir genau $\log(M)$ Bits brauchen, um unsere Symbole optimal zu kodieren. Wenn
wir also $2B$ viele Symbole uebertragen koennen, koennen wir insgesamt $2B \cdot
\log(M)$ viele Bits durch diesen Kanal schleussen. Denn jedes Symbol ist ein
Zustand, und ein Zustand ist eine Konfiguration der Bits. Wenn wir also 16
Symbole haben, hat ein Signalzustand 4 Bit an Information. Deswegen haben wir
$2B$ moegliche Symbol, die je $\log(M)$ Bit an Information tragen koennen. Das
ist gerade, was uns Hartley's Gesetz ueber die maximale Kapazitaet eines Kanals
sagt.

Nun zwei weitere Beobachtungen:

1. Wenn die Entropie $H(Q)$ nicht maximal ist, bedeutet das, dass die
   Wahrscheinlichkeit der Symbole nicht uniform ist. Im Weiteren heisst das,
   dass manche Symbole eine hoehere, insbesondere manche aber eine kleinere
   Wahrscheinlichkeit, als $1/n$ haben. Diese haetten dann auch einen hoeheren
   Informationsgehaelt bzw. braeuchten mehr Bit, um kodiert zu werden. Wenn sie
   mehr Bit brauchen, koennen wir im schlimmsten Fall wohl auch weniger Symbole
   aus dieser Quelle senden. Das spiegelt sich gerade dadurch wieder, dass $f_N
   \cdot H(Q)$ dann geringer waere (weil die Entropie geringer waere
   bzw. bestimmte Symbole, mit hoeherer Gewichtgung $p(x)$ dann eben mit weniger
   Bit kodiert wuerden).
2. Wenn $M = 2$ funktioniert Hartey's Gesetz natuerlich auch: $C_H f_N\log(2) =
   f_N \cdot 1 = f_N$, wie oben beschrieben.

\subsubsection{ Rauschen}

Rauschen tritt auf, wenn das Signal durch externe Stoereinfluesse
beeintraechtigt wird. Rauschen macht insbesondere unserer Quantisierung das
Leben schwer, da Signalstufen schwieriger auseinanderzuhalten sind. So koennte
Rauschen ein Symbol ein oder zwei Quantisierungsstufen nach oben oder unten
schieben und das Signal vollkommen verfaelschen. Hierbei ist natuerlich die
Verwendung von Gray-Codes ratsam, wo nebeneinander liegende Signalstufen eine
maximale Hamming Distanz von 1 haben (unterscheiden sich in maximal einem Bit).

Wir betrachten hierbei das sogenannte *Signal to Noise Ratio* (SNR):

$SNR = \frac{\text{Signalleistung}}{\text{Rauschleistung}} =
\frac{P_S}{P_N} .$

Das SNR wird dabei in Dezibel (dB) angegeben. Ein Bel ist dabei der
Zehner-Logarithmus eines Verhaeltnisses von Leistungen. Hat man also zwei
Leistungswerte (Powerwerte) $a$ und $b$ bzw. deren Verhaeltnis $a/b$, so
entspricht ein Bel dem Faktor 10 zwischen $a$ und $b$, da dann $\log_{10}(a/b) =
\log_{10}(10b/b) = \log_{10}(10) = 1$. Bzw. waeren das dann 10 Dezibel. Eine
Veraenderung um Faktor 10 der beiden Werte entspricht also 10 Dezibel bzw. einem
Bel. Also gilt dann, dass das SNR in Bel definiert ist durch: $10 \cdot
\log_{10}(P_S/P_N)$. Ist das SNR nun klein, ist das schlecht. Ist es gross, ist
das gut.

\paragraph{ Beispiel}

Sei die Signalleistung $P_S = 1 \text{ mW}$ und die Rauschleistung $P_N = 0.5
\text{ mW}$. Dann ist das SNR gleich

$10 \cdot \log_{10}(1/0.5) \approx 0.3 .$

\paragraph{ Shannon-Hartley Theorem}

Wenn wir nun das SNR wissen, gibt uns das Shannon-Hartley Theorem die
maximale Kanalkapazitaet $C_S$ eines rauschbehafteten, $M$-aeren Kanals durch:

$C_S = B \log_2(1 + SNR) = B \log_2(1 + \frac{P_S}{P_N})$

Hierbei ist das $SNR$ nicht in Dezibel gemessen, sondern einfach das
Verhaeltnis. Das $+ 1$ im Logarithmus ist die Varianz bzw. Leistung des
Rauscheinflusses $\eta$.

\subsubsection{ Schranken der Kanalkapazitaet}

Die Kapazitaet $C$ eines Kanals ist also durch zwei Faktoren beschraenkt:

1. Die Anzahl $M$ der unterscheidbaren Symbole. Wenn wir zu wenige
   unterscheidbare Symbole haben, dann kann unsere Kanalkapazitaet auch nicht
   sehr gross sein (Hartley-Theorem).
2. Das SNR. Ist das SNR zu klein (zu viel Rauschen relativ zur Signalleistung),
   muss der Abstand $\Delta$ zwischen Quantisierungsstufen moeglicherweise
   erhoeht werden. Dann koennen wir aber auch weniger Symbole senden, die
   Kanalkapazitaet waere also auch reduziert. Das erkennt man auch am
   Shannon-Hartley Theorem.

Deswegen gilt, fuer die Kanalkapazitaet:

$C < \min(C_H, C_S) = \min(2B\log(M), B\log(1 + SNR)) \text{ bit}$

Das Hartley Theorem mit $C_H$ begrenzt hierbei die Kanalkapazitaet anhand der
Bandbreite des Kanals, sowie insbesondere anhand der Anzahl an Signalstufen. Das
Shannon-Hartley Theorem begrenzt den Kanal dann anhand des Rauscheinflusses,
sowie wiederum der Bandbreite des Kanals. Zusammen ergeben sie also die obere
Schranke.

\subsection{ Nachrichtenuebertragung}

\subsubsection{ Quellenkodierung}

*Quellenkodierung* (*Source Coding*) ist der erste Schritt in unserem Modell der
Datenuebertragung. Das Ziel der Quellenkodierung ist es, Redundanz aus den zu
uebertragenden Daten zu entfernen, durch Abbildung von Bitsequenzen auf
Codewoerter. Dies sollte verlustlos sein, entspricht also einer verlustlosen
Datenkompression. Das ist aber nicht immer moeglich oder erwuenscht (da
verlustbehaftete Kompression meist weniger Speicherplatz erfordert).

Quellenkodierung passiert meist auf hoeheren Schichten, beispielsweise schon in
der Darstellungsschicht (6). Hierbei koennte man Daten zum Beispiel durch das
ZIP Dateiformat komprimieren.

\subsubsection{ Kanalkodierung}

Die Kanalkodierung folgt nach der Quellenkodierung. Daten werden in der realen
Welt nicht ohne Fehler uebertragen. Das *Ziel der Kanalkodierung ist es nun, den
Daten absichtlich Redundanz hinzuzufuegen*, sodass eine moeglichst grosse Anzahl
an Bitfehlern *erkannt* und *korrigiert* werden kann.

Ein Masstab fuer die Fehleranfaelligkeit von bestimmten Uebertragungsmedien ist
die *Bitfehlerwahrscheinlichkeit* $p_e$. Sie gibt an, wie hoch die
Wahrscheinlichkeit ist, dass ein gegebener, uebertragener Bit falsch sein wird.

* Bei Ethernet ueber Kupferkabel meist: $p_e \approx 10^{-8}$
* Bei WLAN charakteristisch: $p_e \approx 10^{-6}$

\paragraph{ Blockcodes}

Blockcodes sind eine Methode zur Fehlerbehandlung bei der
Datenuebertragung. Hierbei wird der Datenstrom (Bitfolge) in Sequenzen der
Laenge $k$ aufgeteilt und in *Bloecke* der Laenge $n > k$ gepackt. Dann werden
die uebrigen $n - k$ Bits zur Fehlererkennung und Rekonstruktion verwendet. Die
Coderate $R = \frac{k}{n}$ gibt dabei an, wieviel unsere Daten vom gesamten
Blockcode ausmachen.

\paragraph{ Repetition Codes}

Als Beispiel koennten wir jeden Bit, also jedes Wort der Laenge $k = 1$, auf
einen Blockcode der Laenge $n = 3$ abbilden, indem wir den einen Bit noch zwei
mal duplizieren. Bei der Dekodierung wird dieses Tripel dann auf jenen binaeren
Wert abgebildet, der am haeufigsten vorkommt. So wuerde also $0 \mapsto 000$ und
$1 \mapsto 111$ abgebildet. Dann koennen wir einen einzigen Bitfehler pro
Blockcode tolerieren. Wenn beispielsweise $000$ durch Uebertragungsfehler zu
$001$ wuerde, waere die $0$ noch immer am haeufigsten und somit der dekodierte
Bit. Erst ab *zwei* Bitfehlern waere der urspruengliche Bit verfaelscht.

Diese Art von Kanalkodierung nennt man *Repetition Codes*. Man merkt aber, dass
Repetition Codes nicht sehr effizient ist. Die Datenrate wuerde naemlich
gedrittelt werden. Interessant ist aber, dass man durch Repetition Codes Fehler
nicht nur erkennen, sondern sogar korrigieren kann.

\paragraph{ 4B5B}

Ein beliebter Blockcode ist der 4B5B Code, wobei alle 4 Bits auf ein neues
Datenwort mit 5 Bits abgebildet werden. Somit hat man also wiederum 4 Bits frei,
um bis zu 16 Kontrollinformationen pro Block zu senden (wenn der obere Bit 1 ist
hat man die unteren 4 Bit frei fuer Informationen). Das hat so aber eher in der
Leitungskodierung Anwendung. In der Kanalkodierung, welche sich ja lediglich mit
dem Erkennen und Korrigieren von Fehlern beschaeftigt, koennte der zusaetzliche
Bit z.B. fuer einen Parity Bit genutzt werden. Waere dieser letzte Bit gesetzt,
wuerde das aussagen, dass die vier Bit einen ungerade Kardinalitaet (Anzahl an
gesetzten Bits) haetten. Waere der Bit nicht gesetzt, heisst das, dass die vier
Bit *urspruenglich* eine gerade Kardinalitaet hatten. Kippt nun ein Bit um, so
aendert sich die Paritaet. Dann wuerde man also erkennen, dass das Parity Bit
anders ist, als die Paritaet der uebertragenen Daten. Somit wuesste man, dass
ein Fehler aufgetreten ist.

4B5B Codes in der Kanalkodierung sind also eine Massnahme zum *Erkennen* von
Bitfehlern. Korrigieren koennte man Fehler nicht, weil man nicht weiss, wieviele
Bits genau umgekippt sind (da jede ungerade Anzahl an Fehlerbits die selbe
Paritaet ergibt).

\paragraph{ Praxis}

In der Praxis ist am anderen Ende einer Leitung oft eine Einheit, die die
Coderate der Kanalkodierung je nach geschaetzer Bitfehlrewahrscheinlichkeit
anpasst.

\subsection{ Impulsformung}

Bei der *(Basisband-)Impulsformung* geht es darum, aus einem Datenstrom ein
analoges *Basisbandsignal* zu generieren. Das Basisbandsignal ist dabei jenes
Signal, welches spaeter bei der Uebertragung, z.B. bei Funk durch FM oder AM,
moduliert wird, um letztendlich abgesandt zu werden. Hierbei sendet man in
regelmaessigen Abstaenden gewichtete Sendegrundimpulse. Ein solcher Impuls
stellt dabei jeweils einen Bit oder eine Gruppe von Bits dar.

\subsubsection{ Hintergrund}

Im Weiteren wird die *Leitungskodierung* dafuer zustaendig sein, zu beschreiben,
wie Bitstroeme durch solche gewichteten Grundimpulse kodiert werden. Da, wie wir
sehen werden, die Grundimpulse in der Regel rechteckig sind, besitzen sie aber
theoretisch ein unbegrenztes Frequenzspektrum. Da es in der Praxis bei
bandbegrenzten Kanalen natuerlich nicht moeglich ist, solche unbegrenzten
Signale zu senden, muessen *Leitungscodes* noch entsprechend auf eine
Traegerwelle *moduliert* werden, sodass die Codes mit begrenzter Bandbreite
uebertragen werden koennen. Damit beschaeftigt sich folglich die *Modulation*,
welche weiter unten behandelt wird.

\subsubsection{ Grundimpulse}

Es gibt hierbei verschieden Arten von Grundimpulsen, welche jeweils ueber einem
Zeitraum von $-T/2$ bis $+T/2$ definiert sind:

* Der einfachste nennt sich *Non-Return-To-Zero* (NRZ) Impuls und ist wirklich
  nur der Rechtecksimpuls. Er kehrt innerhalb des Sendeintervalls kein einziges
  Mal zu Null zurueck, daher der Name.
* Ein aehnlicher Grundimpuls kehrt nach der halben Zeit wieder zu Null
  zurueck. Er nennt sich *Return-To-Zero* (RZ) Impuls und wird mit $\rz(t)$ notiert.
* Fuer Manchester Encoding werden *Manchester*-Impulse gesandt. Diese haben die
  Form einer Square Wave, haben also zwischen $-T/2$ und $0$ den Wert $+1$ und
  im Intervall $]0, T[$ den Wert $-1$. Ein Manchester Impuls $\manc(t)$ entsteht
  aus einem RZ Impuls $\rz(t)$ wie folgt: $\manc(t) = 2\cdot \rz(t) - 1$.
* Letztlich ist der $\cos^2$ Impuls auch eine Variante, welcher also gleich der
  quadierten Kosinusfunktion in diesem Intervall ist.

\subsubsection{ Leitungscodes}

*Leitungscodes* bestimmen dabei eine Abfolge solcher Grundimpulse und gewichten sie entsprechend den Daten (z.B. $\pm 1$). Im Kontext von Leitungscodes sprechen wir bei einem Symbol dann nicht notwendigerweise von einem Datum, sondern einfach von einer *physikalisch messbaren Veranderung des Zeitsignals*. Somit bestehen bestimmte Grundimpulse (Bits) also manchmal aus mehr als nur einem Symbol. Beispielsweise der RZ Impuls, welcher zwei Veraenderungen im Signal hat, besteht also aus zwei Symbolen, obwohl er nur einen Bit kodiert. Die Bedeutung des Wortes *Symbol* ist also gewissermassen ueberladen.

Wir charakterisieren verschiedene Leitungscodes dann nach:

* Der Anzahl an Signalstufen (binaere, ternaere, $m$-aere Kodierung)
* Der Anzahl kodierter Bits pro Symbol (pro Veraenderung des Signals)
* Die Schrittgeschwindigkeit, auch *Symbolrate* oder *Baudrate* genannt. Sie wird in Bit pro Symbol, oder *Baud* angegeben.

Optional interessieren wir uns noch dafuer, ob ein Leitungscode *Taktrueckgewinnung* ermoeglicht. Taktrueckgewinnung meint, Synchronisation zwischen Sender und Empfaenger bezueglich dem Takt. Der Empfaenger muss naemlich wissen, wann und in welchen Zeitabstaenden er abtasten muss. Beispielsweise erlaubt NRZ keinerlei solche Synchronisation. RZ hingegen hat immer den Edge zurueck zu Zero, sodass man weiss, dass man in einer halben Periode als naechstes wieder abtasten muss.

Eine weitere, moegliche Eigenschaft ist *Gleichstromfreiheit*. Ist ein Leitungscode gleichstromfrei, so bedeutet das, dass er keinen Gleichanteil hat. D.h. ueber eine unendilche Signalfolge wuerde der Code keinen offensichtlichen "Offset" an Strom haben. Ist das der Leitungscode beispielsweise einfach immer $+5$ Volt, so wuerde eine unendliche Signalfolge einen Gleichstrom von $+5$ Volt erzeugen. Ist der Leitungscode allerdings um den Nullpunkt zentriert, ist das nicht so.

Letztlich ist die Bandbreite des Leitungscodes noch relevant, falls wir mehrere Signale gleichzeitig ueber denselben Kanal senden wollen (Frequency Division Multiplex).

\paragraph{ Non-Return-To-Zero (NRZ)}

Bei dieser Kodierung ist der Grundimpuls der einfache Rechtecksimpuls $\rect(t)$ mit Periodendauer $T$. Die Gewichte fuer den Grundimpuls sind dabei entweder $+1$ oder $-1$. $+1$ wird gewaehlt, wenn der zu sendende Bit gleich $1$ ist. Wenn der Bit $0$ ist, wird der Grundimpuls durch ein Gewicht von $-1$ eben negiert bzw. invertiert. Somit ist das Sendesignal dann definiert durch $s(t) = \sum_{n=0}^\infty d_n \rect(t - nT)$.

Dieses Signal kann man wie die Formel zur Diskretisierung eines Signals interpretieren. Zu jedem Zeitpunkt $t$ befindet sich $t$ in genau einem Impuls, sodass $\rect(t - nT)$ gleich eins ist. Das ist fuer den $n$-ten Impuls also gerade dann, wann $t \in [nT - T/2, nT + T/2]$
liegt. Dann ist $t - nT$ naemlich $\in [-0.5, +0.5]$. Der Rechtecksimpuls ist
gerade fuer dieses Intervall definiert. Fuer alle anderen $m \neq n$ liegt $t -
mT$ ausserhalb des Nicht-Null Intervalls dieses Impulses. So kann man einen
solchen Leitungscode bzw. hier einen Non-Return-To-Zero Code also als
mathematische Funktion in Abhaengigkeit der Zeit $t$ definieren, sodass die
Funktion insbesondere fuer jedes $t$ definiert ist (linkstotal).

Die Eigenschaften dieses NRZ Codes sind:

* Der Code ist binaer (nur zwei Signalstufen)
* Effizienz: 1 Bit pro Symbol (Baud-Rate = 1)
* Keine Takrueckgewinnung. D.h. wenn man eine lange Folge von Einsen oder Nullen
  hat, kann es sein, dass man einiger Zeit nicht mehr weiss, wann man abtasten
  soll. Auch: Sendet man eine lange Folge von Nullen ueber eine Funkverbindung,
  erhaelt der Empfaenger also eigentlich keine neuen Daten oder zumindest nur
  Rauschen. Dann kann die Verbindung gaenzlich verloren gehen.
* Keine Gleichstromfreiheit
* Relativ breites Frequenzpektrum (brauchen mehr Bandbreite).

\paragraph{ Return-To-Zero (RZ)}

Bei *Return-To-Zero* wird der entsprechende Return-To-Zero Impuls $\rz(t)$
mit Periodendauer $T$ verwendet. Dieser ist fuer das halbe Intervall $[-T/2, 0]$
gleich eins und fuer die andere Haelfte $[0, T/2[$ gleich Null. Die Gewichte sind dann wie
bei NRZ gerade $d = +1$ fuer einen gesetzten Datenbit und $d = -1$ fuer einen
geloeschten Datenbit. Das Signal ergibt sich somit durch:

$s(t) = \sum_{n=0}^\infty d_n \rz(t - nT)$.

```
         s(t)
          |
    |-----|
    |     |
    |     |
    |     |
    |     |------|
---------------------> t
```

Eigenschaften von RZ:

* Da das Signal drei Stufen $\in \{-1, 0, +1\}$ annehmen kann, ist der Code also ternaer* (bei NRZ hatten wir nur $\{-1, +1\}$).
* Effizienz von 1 Bit / 2 Symbolen, somit ist die Baud-Rate $1/2$. Wir haben zwei Symbole, weil wir einmal den Edge fuer das Datum haben, und einmal den Edge zurueck zu Zero.
* Taktrueckgewinnung durch Pegelwechsel!
* Keine Gleichstromfreiheit (wenn nur 1-en, dann Gleichanteil 0.5)
* Breiteres Spektrum als NRZ (abruptere Spruenge?)

\paragraph{ Manchester Code}

Der *Manchester*-Code besteht aus Manchester-Impulsen, welche um die halbe
Amplitude nach unten verschobene, skalierte RZ-Impulse sind. Die Gewichtung ist
wie bei RZ und NRZ, also ergibt sich das Signal durch:

$s(t) = \sum_{n = 0}^\infty d_n \cdot \{manc(t - nT).$

```
         s(t)
          |
    |-----|
    |     |
    |     |
    |     |
    |     |
---------------------> t
          |     |
          |     |
          |     |
          |     |
          |-----|
```

Seine Eigenschaften sind:

* Binaerer Code (zwei Signalstufen)
* Effizienz von 1 bit / 2 Symbolen (Baud Rate = $1/2$, wie RZ)
* Taktrueckgewinnung wie bei RZ moeglich
* Gleichstromfreiheit! Fuer jede Signalaenderung gibt es naemlich auch die inverse Signalaenderung. Somit ist jeder Impuls null-zentriert und es kann nie ein Offset entstehen.
* Sehr breites und langsam abklingendes Spektrum

\paragraph{ Multi-Level-Transmit 3 (MLT 3)}

Bei Multi-Level Transmit verwenden wir weider den Rechtecksimpuls und definieren
unser Signal durch $s(t) = \sum_{n=0}^\infty d_n \cdot \rect(t -
nT)$. Besonders ist jedoch, wie wir die Gewichte definieren:

$d_n = \sin\left(\frac{\pi}{2}\sum_{k=0}^n b_k\right)$

Fuer das $n$-te Gewichte betrachten bzw. summieren wir also alle vorherigen Bits, und gehen genau so viele Schritte jeweils um $\pi/2$ weiter. Fuer jeden neuen Bit wird die Summe um eines groesser und die Funktion geht $\pi/2$ (90 Grad) weiter. Was also passiert, ist dass wir durch die Stufen $0, \pi/2, \pi, 3\pi/2$ fuer jeden neuen gesetzten Bit quasi *durchrotieren*. Jedes Mal, wenn wir also einen neuen gesetzten Bit senden, geht unser Gewicht um $\pi/2$ Schritte weiter. D.h., dass die Gewichte also durch die Werte $\sin(0) = 0, \sin(\pi/2) = 1, \sin(\pi) = 0, \sin(3\pi/2) = -1$ gehen. Das 3 in MLT 3 kommt also davon, dass wir drei verschiedene Stufen $+1, 0, -1$ haben.

```
         s(t)
          |
    |-----|-----|
    |     |     |
    |     |     |
    |     |     |
    |     |     |
---------------------> t
          |
          |
          |
          |
          |
          |
```

Eigenschaften:

* MLT 3 benutzt einen ternaeren Code.
* Effizienz ist 1 Bit / Symbol (Baud Rate = $1$) (wobei eine Null eigentlich durch *kein Symbol*, also *keine Signalaenderung* definiert ist)
* Keine Taktrueckgewinnung, da eine lange Sequenz von gleichen Bits keine neue Signalaenderung bedeutet.
* *Schmales Spektrum*, da das Signal durch den periodischen Verlauf eine
  reduzierte Grundperiode hat.

\subsubsection{ Kodierung zusaetzlicher Informationen}

Woher wissen wir nun aber, ob auf unserem Medium wirklich Daten vorliegen, oder
ob gerade einfach nichts gesendet wird? Auch fragen wir uns, wie wir denn
ueberhaupt den Start und das Ende einer Nachricht festlegen koennen. Wir
brauchen also irgendeine Art von Kodierung, die uns gerade das erlaubt. Hierfuer
gibt es zwei Moeglichkeiten: *Coderegelverletzungen* oder *Steuerzeichen*.

\paragraph{ Coderegelverletzung}

Fuer diese Moeglichkeit nutzen wir die Tatsache aus, dass manche der
Leitungskodierungsverfahren strenge Regeln fuer die moeglichen Werte der
Signalpegel haben. Wir koennen also spezielle Informationen senden (quasi mit
hohem Informationsgehalt), wenn wir diese Regeln verletzen. So koennen wir also
beispielsweise signalisieren, dass eine Verbindung abgebrochen/vollendet
ist. Konkret:

* Vor Beginn einer Nachricht wird eine fest definierte Anzahl alternierender
  Bits gesendet. Diese Bits nennt man die *Praeambel* und dienen auch zur
  *Taktsynchronisation*.
* Um den Beginn der eigentlichen Nachricht zu signalisieren, wird eine bestimmte Bitsequenz festgelegt: Den *Start Frame Delimiter* (SFD). Diese Sequenz muss sich von der Praeambel auch irgendwie unterscheiden. Beispielsweise koennte man vier alternierende Bits mit *halber* Taktfrequenz senden. Diese Coderegelverletzung wuerde man dann erkennen.
* Um die Nachrichtuebertragung abzuschliessen, kehre entgegen der Regeln, zu
  Null zurueck.

Dieses Verfahren, die Regeln zu verletzen, indem man zu Null zurueckkehrt, funktioniert bei NRZ (indem man zur Null zurueckkehrt), RZ (kehre nicht zu Null zurueck) sowie Manchester Encoding (gehe auf Null). Es funktioniert jedoch nicht bei MLT3, da eine lange Folge von Nullen hierbei eine valide Nachricht signalisiert (nicht wie bei NRZ oder Manchester Encoding) und ein Signalwechsel (wie bei RZ) nicht immer notwendig ist.

Diese Methode wird bei 10 Mbit/s Ethernet verwendet.

\paragraph{ Steuerzeichen}

Die zweite Moeglichkeit ist es, einen speziellen Blockcode zu definieren, sodass fuer jeden uebertragenen Block ein Teil der Bits fuer Kontrollinformationen uebrig bleibt (siehe Sektion zu Blockcodes). Bei der Kanalkodierung dienten Blockcodes dazu, Fehler erkennen und sogar korrigieren zu koennen. In diesem Fall nutzen wir sie aber, um eben Steuerzeichen zu senden. Bleiben noch Codewoerter oder sogar Bits uebrig, so koennen diese dann natuerlich noch zur Fehlererkennung dienen.

Ein Beispiel ist wieder der 4B5B Code, welcher von 100 MBit/s FastEthernet verwendet wird. Die 16 moeglichen Codewoerter (wenn der erste Bit gesetzt ist) koennen fuer Start/Stop oder Idle (End of Transmission) Signale genutzt werden. Auch werden manchmal 8B10B Codes verwendet, wo wogar 768 Kontrollcodewoerter moeglich sind.

\subsection{ Modulation}

Die bisher besprochenen Basisbandsignale sind alle sehr abrupt und ecking, was
auf sehr viele hohe Harmonische hindeutet. Genauer hat ein Rechtecksimpuls
beispielsweise ein unendlich ausgedehntes Frequenzspektrum. Wenn wir nur dieses
eine Signal ueber unseren Kanal senden wollen, ist das auch kein Problem.

Wenn wir aber mehrere Signale gleichzeitig ueber den selben Kanal senden wollen, dann muessen wir dafuer sorgen, dass ihre Frequenzbereiche (Bandbreiten) nicht ueberlappen. Dann koennen wir naemlich zwei Signale nebenlaeufig auf demselben Medium uebertragen und am anderen Ende wieder "auseinanderziehen", indem wir eben die einzelnen Frequenzkomponenten trennen. Die Amplituden der beiden Signale wuerden dann zwar fuer das gesammelte Signal, dass aus dem Kanal ausstroemt, bzw. waehrend der Uebertragung, interferieren (additiv oder subtraktiv). Da wir die Frequenzkomponenten aber alleine betrachten koennen, erhalten wir spaeter wieder die urspruenglichen Amplituden. Dieses Konzept, mehrere Signale gleichzeitig mit verschiedenen Frequenzen ueber den selben Kanal zu senden, nennt man *Frequency Division Multiplex* (FDM).

Wir betrachten bei der Vorlesung nur Amplitudenmodulation. Der Ablauf der
Modulation ist dabei wie folgt:

1. Wir haben unser Signal, bestehend aus Grundimpulsen $g(t)$. Diese haben wie
  gesagt unendliche Bandbreite, was wir nicht wollen. Wir filtern diesen
  Grundimpuls also zunaechst mit einem Tiefpassfilter auf eine maximale Frequenz
  $f_{max}$. Die gefilterterten Impulse bezeichnen wir dabei mit $g_T(t)$, das
  resultierende Datensignal (das aus den Grundimpulsen besteht) mit $s_T(t)$. Anmerkung: Die gefilterten Impulse
  stellen eine Verzerrung des Basisbandsignals in der Zeit-Ebene dar, was aber
  nicht weiter schlimm ist (die Ecken der Rechteckskurven gehen weg), da es nur zur Modulation dient.
2. Wir nehmen dann eine *Traegerwelle* $c(t) = \sin(2\pi f_0 t)$ (carrier wave) mit
   einer bestimmten *Traegerfrequenz* $f_0$ (nur die Grundharmonische), und
   modulieren die Amplitude unseres Signals $s_T(t)$ auf die Tragerwelle $c(t)$
   auf. Wie diese Modulation aber genau passiert, wird weiter  unten besprochen.
3. Mathematisch ist unser moduliertes Signal $s(t)$ dann definiert durch: $s(t)
   = s_T(t) \cdot c(t)$. Man nennt dieses Signal auch *Bandpassignal* (anstatt
   *Basisbandsignal* fuer das unmodulierte).

Im Frequenzbereich passiert folgendes:

1. Das Signal wird durch die Filterung auf eine schmale Bandbreite $f_{max}$ begrenzt.
2. Die Modulierung verschiebt dieses schmale Basisband um genau $f_0$ Hertz (Multiplikation im Zeitbereich entspricht Faltung bzw. Verschiebung im Frequenzbereich).

Das Traegersignal besteht hierbei nur aus einer einzigen Grundharmonischen mit Frequenz $f_0$. Somit hat die Traegerwelle an sich eine winzig kleine Bandbreite (im Prinzip keine). Durch die Multiplikation dieses Traegersignals (diesem einen Sinus-Term) mit dem bandbegrenzten Datensignal $s_T(t)$ entsteht dann das modulierte Signal $s(t)$. Die Bandbreite dieses modulierten Signals ist dann gerade $f_{max}$, zentriert um $f_0$. D.h. die Tiefpassfilterung des Basisbandsignals bestimmt die letztendliche Bandbreite des Signals, das wir uebertragen. Das Traegersignal bestimmt nur, wo wir das modulierte Signal im Frequenzspektrum platzieren.

\paragraph{ 4-ASK}

Eine Variante der Amplitudenmodulation nennt sich *4-ASK* (*Amplitude Shift
Keying*). Hierbei kann das modulierte Signal (Bandpassignal) gerade *vier*
diskrete Stufen annehmen. Wir koennen also mit einem Symbol (einer
Signalaenderung) zwei Bits (aus einem Leitungscode) senden.

Ein Beispiel: Wir waehlen unsere vier Signalstufen mit $\mathbf{S} = \{-3/2, -1/2, +1/2, +3/2\}$. Dann bilden wir je zwei Datenbits aus unserem Datenstrom auf ein Symbol (eine Signalstufe) ab: $00 \mapsto -3/2, 01 \mapsto -1/2, 10 \mapsto +1/2, 11 \mapsto +3/2$. Wir wuerden also je zwei Bit aus einem einkommenden Leitungscode interpretieren und dann in einen Grundimpuls fuer unser neues Basisbandsignal (fuer die Modulation) mit entsprechender Gewichtung aus $\mathbf{S}$ transformieren. Das so entstehende Signal kann dann einfach mit dem Traegersignal multipliziert werden, um das modulierte Signal zu erzeugen, welches letztendlich uebertragen werden kann. Jedes Mal, wenn sich ein Gewicht eines Grundimpulses fuer das Basisbandisignal veraendert, wuerde sich auch die Amplitude des Traegersignals entsprechend der Abbildung veraendern.

Es gibt auch andere Varianten von ASK, wobei $n$-ASK also meint, dass es $n$ verschiedene Signalstufen gibt. $n$ waere hier wohl immer eine Zweierpotenz.

\paragraph{ QAM}

Quadratur-Amplituden-Modulation (QAM) ist eine Modulationstechnik, die
ausnutzt, dass Sinus- und Kosinusschwingungen von einander unabhhaengig sind, da
sie nicht in Phase sind. So koennen wir beispielsweise ein Signal $a \cdot
\sin(\omega t)$ nehmen und ein neues Signal $b \cdot \cos(\omega t)$ einfach
draufaddieren. Spaeter koennen wir den Sinus- und Kosinusanteil des Signals
wieder trennen und erhalten so zwei unabhaengige Informationsquellen. Somit ist
es uns moeglich, die Datenuebertragungsrate zu verdoppeln! Wir koennen naemlich
mit den Sinusanteilen $M$ Bits kodieren und mit den Kosinusanteilen noch einmal
$M$ Bits. Dieses Konzept ist aehnlich FDM, nur dass wir hier auf quasi auf
verschiedenen Phasen uebertragen und nicht auf verschiedenen Frequenzen. Der
\_\_Kosinusanteil des Signals wird dabei als *Inphase-Anteil* bezeichnet, der
Sinusanteil als *Quadratur-Anteil*:

* \_\_Kosinus: Inphase\_\_
* \_\_Sinus: Quadratur\_\_

Ein konkretes Beispiel von QAM waere 16-QAM, wo zwei Bit durch Sinus und zwei
durch Kosinus uebertragen werden. Insgesamt hat man also vier Bit und so 16
verschiedene Signalstufen pro uebertragenem Symbol.

Wenn wir die Sinus- und Kosinusanteile nun also separat betrachten, dann koennten wir meinen, wir koennen nun mehr als $2B$, naemlich $2 \cdot 2B$ viele unterscheidbare, unabhaengige Werte uebertragen. Wir haetten also Nyquist mit seiner Nyquist Rate $f_N$ widerlegt. Das ist aber nicht der Fall. Es stellt sich naemlich heraus, dass sich durch QAM die Bandbreite verdoppelt. Die Datenrate, also die Kanalkapazitaet, ist im Vergleich zu 4-ASK verdoppelt, da die Datenrate ja neben der Bandbreite noch von den moeglichen Symbolen (Signalstufen) $M$ abhaengt ($C_H = 2B\log(M)$), welche bei QAM eben mehr sind. Aber gleichzeitig verbrauchen wir aber auch doppelt so viel Platz im Frequenzspektrum, sodass wir eben nicht mehr Signale bzw. mehr Daten uebertragen koennen, da wir noch immer durch die Bandbreite $B$ des Kanals beschraenkt sind. D.h. ja, die Datenrate verdoppelt sich, aber nein, wir koennen nicht mehr Daten uebertragen.

\paragraph{ PSK}

Eine weitere Modulationsvariante, *Phase Shift Keying* (PSK), veraendert nicht
die Amplitude des Signal, sondern die *Phase*. Das Prinzip ist dem ASK sehr
aehnlich. Bei 4-PSK gaebe es drei verschiedene Signalstufen:

1. 0 Grad Verschiebung
2. 90 Grad Verschiebung
3. 180 Grad Verschiebung
4. 270 Grad Verschiebung

PSK ist wie FM Modulation robust gegenueber Stoerungen in der Amplitude des
Signals, da diese die Phase (wie die Frequenz) nicht veraendern.

\subsection{ Uebertragungsmedien}

Wir haben nun besprochen, wie Daten kodiert und moduliert werden. Jetzt fehlt
es nur noch zu erfahren, wie diese Daten letztendlich eigentlich physisch
uebertragen werden. Hierbei unterscheiden wir zwischen:

1. Leitungsgebundener und
2. Nicht-leitungsgebundener (wireless)

Uebertragung. Sowie dann noch zwischen:

1. akustischen (Schallwellen) und
2. elektromagnetischen

Wellen. Elektromagnetische Wellen werden wir als erstes unter die Lupe nehmen.

\subsubsection{ Elektromagnetische Wellen}

Elektromagnetische (EM) Wellen bestehen aus einer *elektrischen* Komponente
$\vec{E}$ und einer *magnetischen* Komponente $\vec{B}$. Diese sind jeweils
orthogonal zueinander sowie zur Ubertragungsrichtung. EM Wellen haben dabei
folgende Eigenschaften:

* \_\_Im Vakuum\_\_ breiten sie sich mit Lichtgeschwindigkeit $c \approx 3 \cdot 10^8 \text{ m/s}$ aus.
* Im Gegensatz zu Schallwellen brauchen sie kein Medium (propagieren auch im
  Vakuum).
* Innerhalb eines Mediums (z.B. Leiter, Luft, Waende) ist die
  Ausbreitungsgeschwindigkeit geringer als im Vakuum. Wir betrachten also meist
  die Geschwindigkeit $\nu c$, wo $0 < \nu < 1$ als *relative
  Ausbreitungsgeschwindigkeit* bezeichnet wird.
* Die Wellenlaenge $\lambda$ beschreibt die raeumliche Ausdehnung einer
  Wellenperiode.
* Die Frequenz $f$ ergibt sich aus der Wellenlaenge und der Lichtgeschwindigkeit
  (nach der Geschwindigkeit-Weg-Zeit Formel): $f = c/\lambda$.

Das elektromagnetische Spektrum beschreibt dabei den Wertebereich, den elektromagnetische Wellen ueblicherweise annehmen. Wir betrachten darin Frequenzen zwischen ungefaehr zwischen $10^2$ (Wechselstrom) und $10^{23}$ (Hoehenstrahlung). Fuer Datenuebertragung interessiert uns jedoch hauptsaechlich der Bereich zwischen MHz ($10^6$) und einigen Gigahertz ($10^9$).

\subsubsection{ Koaxialkabel}

Koaxialkabel sind eine bestimmte Art von Kabel, welche unter Anderem fuer
Ethernet 802.3a mit 10 MBit/s verwendet werden. Hierbei wird meist ein langer
Bus betrachtet, an welchem mehrere Teilnehmer angeschlossen sind. Ausser diesem
Gebiet sind die Anwendungsbereiche von Koaxialkabel jedoch eher gering. Sie
werden noch fuer Kabel-TV verwendet. Auch Cinch-Kabel im Audio Bereich benutzen
diese Art von Kabel.

Solche Kabel bestehen aus vier Schichten: einem Inneleiter (auch *Seele*
genannt), einem Isolator, einem Aussenleiter und letztlich einem auesseren
Mantel.

\subsubsection{ Twisted-Pair Kabel}

*Twisted-Pair* Kabel werden fuer viele heutige Netwerkanwendungen benoetigt,
 beispielsweise den meisten Ethernet Standards. Hierbei besteht ein Kabel aus
 mehreren (2 - 4) *Paaren*, bestehend aus jeweils zwei *Adern*, wobei durch ein
 Paar immer dasselbe Signal gesendet wird.

 In der einen Ader eines Paarse wird der Signalpegel jedoch immer
 invertiert. Beim Empfaenger wird das Signal dann wieder rueckinvertiert. Das
 eigentliche empfangene Signal wird dann dadurch gebildet, dass die Differenz
 (Mittelwert) der beiden Signale gebildet wird. Dies nennt man *differential
 mode*. Wenn die Signale unveraendert ankommen, dann wird der Mittelwert (nach dem Invertieren des invertierten Signals) der
 Signale wieder das urspruengliche Signal sein. Wenn nun aber Rauschen
 hinzukommt, dann wird das Invertieren sehr nuetzlich: Das Rauschen wuerde also
 sowohl beim Signal in der einen Ader als auch beim invertierten Signal additiv
 wirken. Das bedeutet, dass es beim einen Signal die Amplitude erhoeht und beim
 anderen die Amplitude um den selben Wert reduziert. Nach dem
 Mittelwert-Verfahren werden sich die Amplituden der beiden Signale dann wieder
 zurueck zum urspruenglichen Signal normalisieren. Somit hat man also das
 Rauschen entfernt.

 Neben der Invertierung ist es noch eine interessante Eigenheit, dass Paare
 immer verdrillt (*twisted*) sind. Das reduziert das gegenseitige Uebersprechen
 (*Crosstalk*) der Signale in den nebeneinanderliegenden Paaren. Genauer ist der
 Sinn davon, dass sich die Signale in den Aderpaaren weniger beeinflussen, wenn
 ihre Amplituden invers sind (Hochpunkt vom einen = Tiefpunkt vom anderen). Wenn
 sich die Adern also beeinflussen, dann soll sich das ueber Zeit ausgleichen,
 was durch Verdrillung ermoeglicht wird.

 Am Ende eines solchen Twisted-Pair Kabels werden meist RJ-45 oder RJ-11
 Steckverbinder angeschlossen, wie sie von den meisten Ethernet Kabeln bekannt
 sind.

Man unterscheidet Twisted-Pair Kabel noch nach ihrer Schirmung:

* Unshielded Twisted Pair (UTP)
* Shielded Twisted Pair (STP)
* Screened Unshielded Pair (S/UTP): Hierbei werden Kabel noch durch eine
  spezielle Aluminium-Folie umgeben.
* Screened Shielded Pair (S/STP)

Die Schirmung hat dabei Einfluss auf sowohl die Signalqualitaet (z.B. bezueglich
Crosstalk) wie auch die Flexibilitaet der Kabel (gut geschirmte Kabel sind
steifer).

Es gibt noch einen Unterschied darin, wie die Kabel innerhalb der Schirmung
konfiguriert werden. Es gibt dabei die *Straight-Through* (halbduplex) Kabel,
welche nur verschiedene Geraetetypen (z.B. PC und Hub oder Switch) verbinden
duerfen, sowie die *Cross-Over* (vollduplex) Kabel, welche auch gleiche Geraete (PC und
Router, Hub und Switch) verbinden koennen. Das hat damit zu tun, auf welchen
Pins (Adern) diese Geraete senden und auf welchen sie empfangen.

\subsubsection{ Optische Leiter}

In *optischen Leitern* werden nicht elektromagnetische Wellen, sondern Licht
uebertragen. Ein solcher Leiter besteht dabei aus einem *Kern*, in welchem das
Licht propagiert; einem Mantel sowie einer Huelle. Kern und Mantel haben dabei
einen verschiedenen Brechungsindex, sodass es innerhalb des Kerns annaehernd zu
Totalreflexion kommt. Wir unterscheiden dabei noch zwischen zwei Arten von
optischen Fasern (Leitern):

1. Single-Mode Fasern vermeiden Streuung des Lichts durch sehr geringen
   Kernduchmesser. Dadurch hat man weniger Signalverluste, jedoch ist das Kabel
   auch sehr empfindlich bezueglich Kabelbruch.
2. Multi-Mode Fasern haben einen groessern Kerndurchmesser und neigen daher eher
   zum Streuen. Sie haben also hoehere Verluste (nach aussen), sind aber weniger
   fragil.

Vorteile von optischen Leitern gegenueber elektromagnetischen sind:

1. Sehr hohe Datenraten moeglich.
2. Weite Strekcen ueberbrueckbar.
3. Kein Uebersprechen.
4. Galvanische Entkopplung von Sender und Empfaenger.
5. Energiesparender als andere Leitungen.

\section{OSI Modell: Layer 2} 

\subsection{Graphdarstellung} 

Wir stellen Netzwerke oft als Graphen dar, wobei wir eine bestimmte Notation
verwenden.

\subsubsection{Gerichtete Graphen} 

Ein gerichteter Graph stellt ein *asymmetrisches* Netzwerk dar, wobei Kanten
also nicht symmetrisch sind. Ein solcher Graph $G = (N, A)$ ist durch eine Menge
an Knoten $N$ (Nodes oder Vertices) und einer Menge an *gerichteten* Kanten $A$
(*Arcs*) definiert. Die Menge an Kanten $A$ enthaelt dabei *2-Tupel* bzw. Paare,
sodass fuer eine gerichtete Kante von $i$ nach $j$ ein Paar $(i, j) \in A$ ist.

Hierbei kann eine Kante $(i, j)$ noch *Kosten* $c_{i,j}$ haben. Das waeren
beispielsweise die Kosten, um Daten vom Knoten $i$ zum Knoten $j$ zu uebertragen.

\subsubsection{Ungerichtete Graphen} 

Ein *symmetrisches* Netzwerk laesst sich als *ungerichteter* Graph $G = (N, E)$
darstellen, wobei $N$ wieder eine Menge an Knoten ist und $E$ eine Menge an
Kanten (hier: *Edges*). Die Kanten koennen wieder Kosten haben.

\subsubsection{Pfade} 

Ein Pfad zwischen zwei Knoten $s, t \in N$ ist eine Menge $P_{s,t} = \{(s, i),
..., (j, t)\}$ gerichteter Kanten, die $s$ und $t$ miteinander
verbinden. Hierbei nennt man $s$ oft den *Source* und $t$ den *Terminal* Knoten.

Die *Kosten* $c(P)$ eines Pfades $P_{s,t}$ entsprechen dabei den Kosten der Kanten
in diesem Pfad: $c(P) = \sum_{(i,j) \in P_{s,t}} c_{i,j}$

Die *Laenge* eines Pfades entspricht der \_\_Anzahl an Kanten\_\_ auf dem Pfad:
$l(P_{s,t}) = |P_{s,t}|$. Diese Pfadlaenge nennt man auch *Hop Count*.

\subsubsection{Besondere Baumstrukturen} 

Ein Baum ist ein azyklischer, zusammenhaengender Graph. Wir unterscheiden dabei
zwischen zwei Arten von Baeumen, die fuer uns interessant sind:

1. *Shortest Path Tree* (SPT): Verbindet einen Wurzelknoten mit jedem anderen
   Knoten im Graphen mit minimalen Pfadkosten.
2. *Minimum Spanning Tree* (MST): Verbindet alle Knoten im Netzwerk mit
   insgesamt minimalen Kosten.

Diese Baeume sind im Allgemeinen nicht identisch.

\subsection{Verbindungen} 

Wir koennen eine Verbindung bezueglich mehreren Eigenschaften charakterisieren:

* *Uebertragungsrate*
* *Uebertragungsverzoegerung*
* *Uebertragungsrichtung*
* *Mehrfachzugriff* (Multiplexing)

Diese Eigenschaften wollen wir fuer einfache *Punkt-zu-Punkt* Verbindungen
charakterisieren.

\subsubsection{Uebertragungsrate} 

Die *Uebertragungsrate* $r$ einer Verbindung wird in Bit/s gemessen und sagt
aus, mit welcher Rate Daten auf eine Leitung gelegt werden koennen. Sie haengt
eng mit der *Uebertragungsverzoegerung* $t_s$, auch *Serialisierungszeit*
(*Serialization Delay* oder *Transmission Delay*) genannt, zusammen. Die Formel
ergibt sich dabei analog zum Geschwindigkeit-Weg-Zeit Verhaeltnis:

$t_s = \frac{L}{r}$

Hierbei ist $L$ die Anzahl an Bits, die mit der Uebertragungsrate $r$ in $t_s$
Sekunden uebertragen werden sollen. Wollen wir so also zum Beispiel 1500 Byte
mit einer Rate von 100 MBit/s (FastEthernet) senden, so erhalten wir eine
Serialisierungszeit von:

$t_s = \frac{L}{r} = \frac{1500 \cdot 8}{100 \cdot 10^6} = 120 \mu\text{s}$

Intuitiv kann man sich die Serialisierungszeit als die Zeit vorstellen, die man
benoetigt, um Murmeln (Daten) in ein Rohr zu stopfen. Man kann diese Murmeln
nicht alle gleichzeitg in das Rohr stopfe, insofern dauert es eine Weile, bis
alle Murmeln im Rohr ankommen. Das ist gerade die *Serialisierungszeit*.

\subsubsection{Ausbreitungsgeschwindigkeit} 

Wenn wir Daten ueber ein Kabel senden wollen, dann gibt uns die
Serialisierungszeit an, wie lange es dauert, bis die Daten allesamt auf das
Kabel "gelegt" wurden. Wie lange dauert es aber, bis die Daten durch das Kabel
gewandert sind? Dies wird von der *Ausbreitungsverzoegerung* $t_p$ (*Propagation
Delay*) bestimmt.

Im Allgemeinen propagieren elektromagnetische (EM) Wellen mit
Lichtgeschwindigkeit $c$. Das gilt aber nur, wenn das Medium Vakuum ist. In
anderen Medien, beispielsweise Kupfer oder Glas, ist die Lichtgeschwindigkeit
reduziert. Um welchen Faktor die Lichtgeschwindigkeit reduziert ist, haengt von
der *relativen Ausbreitungsgeschwindigkeit* $\nu$ ab. Die letztendliche
Geschwindigkeit, mit welcher eine EM Welle durch ein Medium propagiert, ist also
gerade $\nu c$. Haben wir also nach $t_s$ Sekunden unsere Daten serialisiert, so
gibt uns die Ausbreitungsverzoegerung $t_p$ an, wie lange es dauert, bis der
erste Bit beim Empfaenger ankommt. In Abhaengigkeit von $\nu$ und der *Distanz*
$d$ der Uebertragung berechnet sie sich also durch:

$t_p = \frac{d}{\nu c}$

Wenn wir jetzt also wieder an die Murmeln und das Rohr denken, so sagt und die
Ausbreitungsverzoegerung, wie lange es dauert, bis die erste Murmel am anderen
Ende des Rohres ankommt, was natuerlich von der Distanz (Laenge) des Rohres und
der Art des Rohres (Medium) ankommt. Die Serialisierungszeit sagte uns nur, wie
lange es dauert, die Murmeln in das Rohr zu stopfen.

\subsubsection{Uebertragungszeit} 

Jetzt wissen wir also, wie lange es dauert unsere Daten auf das Kabel zu legen
(Serialisierungszeit) und wie lange es dauert, bis der erste Bit dann am anderen
Ende ankommt (Ausbreitungsverzoegerung). Wie lange dauert die ganze Uebertragung
dann? Also, wann kommt der *letzte* Bit, und somit die ganze Nachricht, an?

Nun ja: Nach $t_s$ Sekunden liegt der letzte Bit auf der Leitung. Nach $t_p$
Sekunden kommt dieser dann am anderen Ende an. Folglich ist die gesamte
Uebertragungszeit $t_d$ (*delay*) gerade gleich:

$t_d = t_s + t_p = \frac{L}{r} + \frac{d}{\nu c}$

Man kann diese verschiedenen Zeiten in einem *Weg-Zeit-Diagramm* grafisch
veranschaulichen. Hierbei betrachtet man zwei Zeitachsen fuer zwei Knoten $i,
j$, zwischen welchen Daten gesendet werden. Die Distanz auf der "$x$"-Achse
repraesentiert dann die Distanz $d$ zwischen Knoten $i$ und $j$. Man sieht dann
naemlich auch schoen, dass es eine Weile ($t_p$) dauert, bis der erste und dann
der letzte serialisierte Bit von $i$ bei $j$ ankommt. Das wird dadurch
visualisiert, dass die "Uebertragungskante" von $i$ nach $j$ schraeg ist.

\begin{verbatim}
    i               j
    |_______________|
    |    \_____     | t_p
t_s |          \____|
    |               |
    |___            | t_s
t_p |   \_____      |
    |_________\_____|
    |               |
    v               v
\end{verbatim}

\subsubsection{Speicherkapazitaet eines Kanals} 

Da sich Signale nur mit endlicher Geschwindigkeit ausbreiten, dauert es wie
gezeigt eine Weile ($t_p$), bis der erste Bit beim Empfaenger ankommt. In der
Zwischenzeit werden aber durch die Serialisierung der Daten neue Bits auf das
Kabel gelegt. Das geschieht eben $t_s$ Sekunden lang. Wenn man also das Kabel zu
einem konkreten Zeitpunkt waehrend der Serialiserung "einfrieren" wuerde, so
koennte man auf dem Kabel gerade propagierende Bits betrachten. Z.B. waeren nach
der halben Ausbreitungsverzoegerung $t_p$ gerade $t_p/2 \cdot r$ neue Bits auf
das Kabel gelegt (unter der Annahme, dass $t_p/2 < t_s$ ist). Denn in dieser
Zeit serialisieren wir ja mit Datenrate $r$ neue Bits. Somit koennten wir also
fast sagen, dass das Kabel eine gewisse *Speicherkapazitaet* hat. Diese
Speicherkapazitaet $C$ wird auch *Bandbreitenverzoegerungsprodukt* genannt und
ist also gegeben durch:

$C = t_p \cdot r$

Diese Kapazitaet ist also die maximale Anzahl an Bits, die sich in einer
Senderichtung gleichzeitig auf der Leitung befinden koennen. Man kann dann auch
sagen, dass ein Bit eine gewisse "Breite" hat, indem man die Distanz durch die
Kapazitaet teilt. Kann man beispielsweise 50 bit auf einem 100 Meter breiten
Kabel speichern, dann ist ein Bit also $100 / 50 = 2$ Meter lang.

\subsubsection{Uebertragungsrichtung} 

Wir unterscheiden noch hinsichtlich der Richtung einer Uebertragung zwischen
zwei Knoten $i, j$. Es gibt drei Moeglichkeiten:

1. Simplex: Unidirektional, also immer nur von $A$ nach $B$ und nie von $B$ nach
   $A$.
2. Halbduplex: Bidirektional, aber es kann immer nur in eine Richtung zu einem
   Zeitpunkt gesendet werden.
3. Vollduplex: Bidirektional, wobei gleichzeitig in beide Richtungen gesendet
   werden kann.

\subsubsection{Multiplexing} 

Mehrfachzugriff oder *Multiplexing* ist die Bezeichnung fuer die Situation, wo
mehrere Nachrichten (verschiedener Teilnehmer) *gleichzeitig* ueber einen Kanal
gesendet werden sollen. Es gibt hierbei vier verschiedene Arten von Verfahren:

1. *Zeitmultiplex* (*Time Division Multiplex*; TDM)
   * Mehrere Teilnehmer koennen auf ein Medium zugreifen, indem die *Zeit*
     aufgeteilt wird (slicing; einer nach dem Anderen).
   * Das kann deterministisch passieren, zum Beispiel wenn Nachrichten in eine
     Queue gegeben werden.
   * Oder auch nicht, beispielsweise bei einem Hub, wo mehrere Teilnehmer
     nicht-deterministisch (willkuerlich) Nachrichten senden koennen, welche
     womoeglich auch kollidieren (und verloren gehen).
2. *Frequenzmultiplex* (*Frequency Division Multiplex*; FDM)
   * Aufteilung des Kanals in unterschiedliche Frequenzbaender und Zuweisung von
     bestimmten Baendern an bestimmte Kommunikationspartner.
   * Wird bei allen Funkuebertragungen, sowie auch optischen Leitern verwendet.
3. *Raummultiplex* (*Space Division Multiplex*; SDM)
   * Verwendung mehrerer paralleler, physischer Uebertragungskanaele
     (z.B. mehrere Kabel oder Adern pro Kabel).
   * *Kanalbuendelung* bei ISDN.
   * Mehrere Antennen bei kabellosen Uebertragungen.
4. *Codemultiplex* (*Code Divison Multiplex*; CDM)
   * Verwendung orthogonaler Alphabete und Zuweisung der Alphabete an
     Kommunikationspartner.
   * Z.B. bei 4B5B Codes: das eine Signal hat den ersten Bit immer gesetzt, das
     andere Signal immer nicht.
   * Bei UMTS verwendet.

Wir haben dabei mehrere Bewertungskriterien fuer Medienzugriffsverfahren mit
mehreren Teilnehmern:

1. *Durchsatz*: Wieviele Nachrichten wir pro Zeiteinheit uebertragen koennen.
2. *Verzoegerung* einzelner Nachrichten (wenn ich eine Nachricht senden will,
   wie lange dauert es, bis ich sie senden darf).
3. *Fairness* zwischen Teilnehmern, die sich ein Medium teilen.
4. *Implementierungsaufwand* fuer Sender und Empfaenger.

Weiters kann TDM, auch *TDMA* (*time-division-multiple-access*) genannt, auf zwei
Weisen implementiert werden:

1. *Synchron* (deterministisch): Jeder Teilnehmer bekommt abwechselnd einen
   Timeslot, in welchem er senden darf. Will er in einem Timeslot nicht senden,
   so passiert eben nichts.  Man nennt dies auch *statisches* Aufteilen der Zeit
   bzw. des Kanals. Das Problem ist aber, dass Datenverkehr meist ploetzlich
   bzw. stossartig auftritt. Das bedeutet, dass ein Teilnehmer ploetzlich viel
   senden will, und danach fuer laengere Zeit nicht mehr. Bei einem statischen
   Verfahren muesste ein Teilnehmer, der sofort senden koennte, moeglicherweise
   erst alle anderen Teilnehmer abwarten, obwohl gerade gar niemand anderes (in
   ihren Timeslots) senden moechte.
2. *Asynchron*: Der Zugriff auf das Medium ist ungeregelt. Die Teilnehmer
   *konkurrieren* also quasi um das Medium. Diese Variante loest das Problem der
   hohen Verzoegerung von synchronem TDM.

\subsubsection{ALOHA} 

*ALOHA* ist ein *Random Access* (asynchrones) TDMA Verfahren, bei welchem
 Empfaenger um ein Medium *konkurrieren*. Konkret geht man bei diesem Verfahren
 von mehreren *Stationen* aus, welche allesamt an eine *Basisstation* senden
 wollen. Eine Station *sendet dabei einfach, sobald sie Daten hat*. Da alle
 Stationen auf der selben Frequenz (in der selben Bandbreite) senden, kann es
 dabei also zu Kollisionen kommen.

 Wenn es zu *keiner* Kollision kam, sendet die Basisstation an den Sender eine
 "Quitting" um zu bestaetigen, dass die Uebertragung erfolgreich war. Diese
 Bestaetigung wird dabei *auf einer anderen Frequenz* als der
 Uebertragungsfrequenz gesendet, damit es zwischen Nachrichten und
 Bestaetigungen nicht zu Kollisionen kommen kann. Das nennt man auch
 *out-of-band* Bestaetigung.

Da alle Stationen willkuerlich und vollkommen unkontrolliert senden koennen,
kann es also wie gesagt zu Kollisionen kommen. Wieviele Kollisionen es dabei
gibt, kann man mathematisch modellieren und beschreiben. Wir machen dabei
folgende Annahmen:

* Wir haben eine mittlere bis beliebig grosse Anzahl an Stationen ($N > 15$)
* Alle Stationen senden mit gleicher, unabhaengiger und im Allgemeinen geringer
  Wahrscheinlichkeit
* Alle Nachrichten sind gleich gross und brauchen also auch gleich lange
  (Sendedauer $T$)

Dann modellieren wir die Situation wie folgt:

* Ob ein bestimmter Knoten (Station) $i$ innerhalb eines Zeitintervalls
  $[t, t + T[$ sendet, oder nicht, entspricht einem Bernoulli-Experiment (binaer) mit Sendewahrscheinlichkeit $p_i$.
* Da alle Knoten mit selber Wahrscheinlichkeit $p$ senden, ist $p_i = p \forall i$.
* Da die $N$ Knoten $N$ mal jeweils unabhaengig voneinander senden, koennen wir den ganzen Sendevorgang durch eine *Binomialverteilung* modellieren.
* Der Erwartungswert dieser Binomialverteilung ist dabei gerade $\lambda = Np$
* Fuer sinnvoll grosse $N$, wie wir sie annehmen, kann man eine solche Binomialverteilung dabei durch eine Poisson-Verteilung approximieren.

Dabei betrachten wir eine Zufallsvariable $X_t$, welche beschreibt, wieviele Knoten im Intervall $[t, t + T[$ gleichzeitg senden wollen. Die Wahrscheinlichkeit, dass $X_t = k$ fuer $k$ Knoten ist durch die Poission-Verteilung dann wie folgt beschrieben:

$\Pr[X_t = k] = \frac{\lambda^k e^{-\lambda}}{k!}$

Die Besonderheit an einer Poisson-Verteilung im Vergleich zu einer
Bernoulli-Verteilung hierbei ist, dass wir nicht mehr $N$ unabhaengige
Teilexperimente bzw. $N$ Elemente betrachten, fuer welche wir eine
Erfolgswahrscheinlichkeit $p$ wissen. Mit der Poisson-Verteilung modelliert man
Ereignisse, die mit einem gewissen Erwartungswert $\lambda$ in einem
Zeitintervall auftreten. Insofern ist nur dieser Erwartungswert $\lambda$
relevant und so auch der einzige Parameter der Verteilung. Fuer ein
Zufallsexperiment sagen wir nun nicht mehr, dass $k$ Ereignisse (mit
Wahrscheinlichkeit $p$) eintreten und $N - k$ nicht. Wir sagen einfach, dass nur
$k$ Ereignisse eintreten, wobei die Wahrscheinlichkeit dafuer durch den
Erwartungswert parametrisiert ist. Das ist dann nuetzlich, wenn wir die
Wahrscheinlichkeit eines Erfolges nicht explizit wissen, sowie wenn die Anzahl
an betrachteten Elementen unklar oder zu gross ist. Beispielsweise kann man mit
Poission modellieren, wieviele E-Mails man pro Tag bekommt. Man beobachtet, dass
man im Schnitt 4 E-Mails taeglich erhaelt. Man kann hier nicht wirklich
beschreiben, mit welcher Wahrscheinlichkeit man eine E-Mail erhaelt oder
nicht. Auch gibt es keine Anzahl $N$ moeglicher E-Mails, somit kann man nicht
sagen, dass man $X$ E-Mails "nicht erhalten hat". Waere die Wahrscheinlichkeit
bekannt und gaebe es nur $N$ moegliche E-Mails, so koennte man mit einer
Binomialverteilung beschreiben, dass man $k$ erhaelt und $N - k$ *nicht*. Wenn
es aber kein $N$ und kein $p$ gibt, sondern nur einen Erwartungswert $\lambda$,
dann ist eine Poisson-Verteilung geeigneter.

Fuer das Modell unserer Stationen ist Poisson insbesondere interessant, weil wir
leichter sagen koennen, wieviele Stationen im Schnitt pro Intervall senden, als
eine Wahrscheinlichkeit $p$ dafuer anzugeben. Hierfuer muessen wir aber eben
annehmen, dass es ausreichend viele Stationen gibt. Hierbei sei angemerkt, dass
es dennoch besser ist, die Wahrscheinlichkeit fuer eine erfolgreiche
Uebertragung mit der Binomialverteilung zu berechnen, wenn die
Wahrscheinlichkeit schon gegeben ist. Denn die Poisson-Verteilung ist nur eine
Approximation der Binomialverteilung.

Eine beliebige Station sendet also zum Zeitpunkt $t_0$ eine Nachricht. Eine
Kollision genau dann auf, wenn eine andere Station:

* Im Intervall $[t_0, t_0 + T[$ selbst sendet, oder
* Im vorherigen Intervall $]t_0 - T, t_0[$ angefangen hat zu senden.

Das Intervall $]t_0 - T, t_0 + T[$ nennt man folglich das *kritische Intervall*. Durch die Poisson-Verteilung erhalten wir dann die Wahrscheinlichkeit $p_0$ fuer eine erfolgreiche Ubertragung durch:

1. Die Wahrscheinlichkeit, dass im vorherigen Intervall (infinitisemal nach dem Start) keiner sendet: $\Pr[X_{t_0 - T} = 0]$
2. Die Wahrscheinlichkeit, dass im momentanen Intervall genau einer sendet:
   $\Pr[X_{t_0} = 1]$

Somit:


\begin{align}
	p_0 &= \Pr[X_{t_0 - T} = 0] \cdot \Pr[X_{t_0} = 1]\\
	    &= \frac{0^k e^{-\lambda}}{0!} \cdot \frac{\lambda^1 e^{-\lambda}}{1!}\\
		&= e^{-\lambda} \cdot \lambda e^{-\lambda}
		&= \lambda e^{-2\lambda}
\end{align}


Hierbei sei angemerkt, dass das konkrete Zeitintervall fuer die
Poisson-Verteilung egal ist, da wir annehmen, dass die Anzahl sendender
Stationen fuer jedes Zeitintervall gleichverteilt ist.

\paragraph{Slotted ALOHA} 

Eine Variante von ALOHA, welche ein wenig besser funktioniert, ist *Slotted
ALOHA*. Hierbei duerfen Stationen nicht mehr zu beliebigen Zeitpunkten zu senden
beginnen, sondern nur mehr am Anfang diskreter Zeitintervalle $nT: n \in
\mathbf{N}_0$. Somit verschwindet also das hintere Intervall $]t_0 - T,
t_0[$ aus dem kritischen Bereich, da zum Zeitpunkt $t_0$ jede Uebertragung aus dem vorherigen Slot (ab $t_0 - T$) schon vollendet sein muss. Der kritische Bereich ist somit auch nur mehr $T$ lang und nicht $2T$, wodurch sich die Wahrscheinlichkeit fuer eine erfolgreiche Uebertragung auf den ersten Term von oben reduziert:

$p_0 = \Pr[X_{t_0} = 1] = \lambda e^{-\lambda} = \frac{\lambda}{e^\lambda}$

\subsubsection{CSMA} 

Eine einfache Verbesserung von ALOHA bzw. Slotted ALOHA (welches immer ueber
normalen ALOHA zu bevorzugen ist) ist das Prinzip *Listen before Talk*, einer
Variante von *Carrier-Sense-Multiple-Access* (*CSMA*). Hierbei hoert jede Station
das Uebertragungsmedium einfach ab, bevor es erst zu senden beginnt (falls das
Medium auch frei war).

Weiters betrachten wir verschiedene Arten von CSMA:

* *Non-persistent* CSMA:
  1. Hoere das Medium ab. Wenn es frei ist, uebertrage im naechstmoeglichen
     Slot.
  2. Wenn es nicht frei ist, warte eine feste oder zufaellige Zeitspanne
     (z.B. einen Slot) und hoere dann wieder ab (1).

* *1-persistent* CSMA:
  1. Hoere das Medium ab. Wenn es frei ist, uebertrage im naechstmoeglichen
     Intervall.
  2. Wenn es nicht frei ist, warte busy bis es frei ist (spinning).

* *$p$-persistentes* CSMA:
  1. Hoere das Medium ab. Wenn es frei ist, dann:
	 1. Uebertrage mit Wahrscheinlichkeit $p$ im naechstmoeglichen Slot, oder
	 2. Verzoegere mit Wahrscheinlichkeit $1 - p$ und hoere danach wieder ab.
  2. Wenn es nicht frei ist, warte bis es frei ist. Dann (1).

So merken:

1. non-persistent: wartet feste Zeitspanne.
2. persistent: spinned.
   $1$: sendet immer im naechstmoeglichen Slot.
   $p$: sendet mit Wahrscheinlichkeit $p$.

Es sei angemerkt, dass wir hierbei natuerlich immer von *Slotted* ALOHA
ausgehen, sonst macht das mit den Slots ja keinen Sinn.

*Abhoeren* wuerde bei CSMA im Uebrigen bedeuten, dass die Stationen sehen, ob
ihnen gerade etwas gesendet wird. Sendet eine Station, so breiten sich deren
Informationen natuerlich nicht nur zur Basisstation, sondern zu allen Stationen
aus. Eine Station weiss also, dass ein Medium besetzt ist, wenn es gerade
Nachrichten (die fuer die Basisstation gedacht sind) empfaengt.

Man beachte aber, dass CSMA Kollisionen nur weniger wahrscheinlich macht, aber
nicht vollkommen verhindern kann. Es koennen zum Beispiel mehrere Teilnehmer
gleichzeitig erkennen, dass das Medium frei ist, und gleichzeitg senden. Vor
allem muss man aber bedenken, dass es auch eine Weile dauert, bis das Medium
"besetzt" wird, wenn eine Station einemal sendet. Nehmen wir die folgende
Situation:

1. Eine Station hat Daten anliegen, waecht also auf.
2. Es hoert das Medium ab (empfange ich Nachrichten auf der
   Uebertragungsfrequenz?) und sieht, dass es frei ist.
3. Es sendet die Nachricht (mit Wahrscheinlichkeit $p$ bei $p$-persistentem
   CSMA).
4. Es dauert nun erstmal $t_p$ Sekunden, bis diese Nachricht bei anderen
   Stationen ankommt (bzw. anfaengt anzukommen), das Medium fuer sie also
   *besetzt* wird.
5. Nach $t_p/2$ Sekunden waecht eine andere Station auf und hat auch Daten
   anliegen.
6. Da nach der halben Ausbreitungsverzoegerung die Nachricht von der ersten
   Station noch nicht bei dieser Station angekommen ist, wirkt das Medium fuer
   diese zweite Station als nicht besetzt.
7. Es sendet also, und es kommt trotz CSMA zu einer Kollision.

Es sei wohlgemerkt, dass nach der Ausbreitungsverzoegerung das Medium fuer die
anderen Stationen besetzt waere. Bei CSMA ist das kritische Fenster also genau
$[t_0, t_0 + t_p]$. (Auch sei beachtet, dass $t_p$ bei unterschiedlichen
Distanzen von der sendenden Station auch fuer jede hoerende Station andere Werte
annimmt.) Auch sei angemerkt, dass diese Ueberlegungen nur fuer non-slotted
ALOHA valide sind, da bei slotted ALOHA eine Station erst zum naechsten Slot
senden wuerde, wenn das Medium gerade frei ist. Somit kaeme es nicht zu dieser
Situation. Bei slotted ALOHA gibt es nur mehr die Gefahr, dass zwei Stationen
gleichzeitig merken, dass das Medium frei ist und gleichzeitig zum naechsten
Slot senden. Hierbei hilft $p$-persistentes CSMA eben aus, da bei geringem $p$
nicht beide Stationen dann wirklich senden wuerden. Am guenstigsten scheint im
Uebrigen $0.01$-persistentes und $0.1$-persistentes CSMA zu sein.

\subsubsection{CSMA/CD} 

Wie oben beschrieben verhindert CSMA Kollisionen also nicht. Eine Kollision
wuerde von sendenden Stationen also so bemerkt, dass von der Basisstation keine
Quittierung kommt. Ein anderes Verfahren, um Kollisionen zu *erkennen*, ist die
Verwendung von *JAM* Signalen. Allgemein faellt dies unter das Gebiet von
*Collision Detection* (CD). Insbesondere erlauben JAM Signale ein schnelleres
Erkennen von Kollisionen und schnelleres Abbrechen von kollidierenden
Uebertragungen als fehlende out-of-band Quittierungen.

Wie mit JAM Signalen Kollisionen erkannt wird, sei an diesem Beispiel illustriert:

1. Eine Station $A$ sendet eine Nachricht, weil das Medium gerade frei war.
2. Noch waehrend die Nachricht zu einer anderen Station $B$ propagiert ($t_p$),
   hoert es da Medium ab.
3. Weil fuer Station $B$ das Medium noch frei ist, sendet es auch seine
   Nachricht.
4. Nach einer gewissen Zeit, bevor $B$ mit dem Senden seiner Nachricht fertig
   ist, kommt die Nachricht von $A$ endlich bei $B$ an.
5. Station $B$ erkennt also eine Kollision, und sendet ein spezielles JAM-Signal
   weg.
6. Wenn die Serialisierungszeit der Nachricht von $A$ passend gewaehlt wurde,
   erhaelt Station $A$ dieses JAM Signal noch waehrend es seine Nachricht
   sendet.
7. Beide haben die Kollision also erkannt, und probieren es je nach CSMA
   Verfahren im naechsten Zeitslot nochmal.

Besonders wichtig ist hierbei, dass die Serialisierungszeit von Nachrichten fuer
die Verwendung von JAM Signalen richtig gewaehlt werden muss. Konkret muss
gelten:

$t_{s_{min}} = 2 t_p \iff L_{min} = \frac{2d}{\nu c} r$

Die Serialisierungszeit muss also zwei mal so gross, wie die
Ausbreitungsverzoegerung sein. Wir ueberlegen uns, wieso. Dafuer nehmen wir
gerade den Extremfall an: Station $B$ hoert das Medium infinitesimal kurz vor
dem Zeitpunkt ab, wo die Nachricht von Station $A$ bei $B$ ankommt, und sendet
seine Nachricht, weil das Medium ja frei war. Die Nachricht von Station $A$
trifft bei $B$ also unmittelbar nach Beginn der Serialisierung der
Nachricht von $B$ ein. Sofort sendet Station $B$ ein JAM Signal weg. Nun
ueberlegen wir uns: Es hat $t_p$ Sekunden gedauert, bis der erste Bit der
Nachricht von Station $A$ bei $B$ angekommen ist und die Kollision erkannt
wurde. Es dauert dann genau $t_p$ weitere Sekunden, bis das JAM Signal in die
andere Richtung von $B$ bei $A$ ankommt. Das JAM Signal wuerde also erst nach $2
t_p$ Sekunden bei $A$ eintreffen! Wenn nun also die Serialisierungszeit der
Nachricht von Station $A$ weniger als $2 t_p$ Sekunden dauert, ignoriert Station
$A$ das JAM Signal einfach, weil es zum Zeitpunkt des Eintreffens des JAM
Signals ($2 t_p$) ja keine Nachricht mehr sendet (wieso sollte es also zu einer
Kollision gekommen sein? Anmerkung: Eine Station wird das JAM Signal, wenn es
nicht sendet, ignorieren, weil ja auch andere Stationen waehrend ihrer
Uebertragung das selbe JAM Signal senden, was fuer die Station dann keine
Bedeutung haette).

Es gibt nun aber bei Punkt (7) ein Problem: Die Stationen erkennen also eine
Kollision, und warten bis das Medium wieder frei ist und probieren dann
erneut. Wenn nach der Kollision das Medium frei ist, probieren die Stationen es
also sofort wieder. Es kaeme also schon wieder zu einer Kollision!

Die Loesung dafuer ist, nach einer Kollision eine bestimmte Zeit zu warten. Zum
Beispiel verwendet Ethernet (welches CSMA/CD verwendet) *Binary Exponential
Backoff*. Hierbei wird bei der $k$-ten Wiederholung einer Nachricht (also nach
der $k$-ten Kollision):

1. Eine \_\_zufaellige\_\_ Anzahl an Slotzeiten $n$ aus $\{0, 1, \min(2^{k - 1},
   1024)\}$ gewaehlt.
2. Gerade so viele Slotzeiten gewartet, bis man es wieder probiert.

Maximal probiert man es dabei 15 mal, bevor man die Nachricht verwirft. Das
entspricht also 16 Sendeversuchen. Dadurch wird die Kollisionswahrscheinlichkeit
wieder verringert.

Es ist hierbei uebrigens sehr wichtig, dass eine *zufaellige* Anzahl an
Slotzeiten gewaehlt wird. Wenn $A$ und $B$ dieselbe Slotzeit waehlen wuerden,
kaeme es halt nicht sofort, sondern nach $x$ Slotzeiten zur Kollision.

\subsubsection{CSMA/CA} 

Collision Detection (CD) hat es uns erlaubt, Kollisionen zu erkennen, noch nicht
aber, sie zu *verhindern*. Das ist das Themengebiet der *Collision Avoidance*
(CA). Am einfachsten funktioniert dies mit dem *Request-To-Send (RTS)* /
*Clear-To-Send (CTS)* Prinzip. Hierbei gibt es neben den ueblichen Stationen
noch eine Basisstation, welche sozusagen als Moderator der Kommunikation dient.

Bevor nun eine Station $A$ an eine Station $B$ eine Nachricht senden will,
sendet $A$ zuerst zur Basisstation eine Request-To-Send. Wenn die Basisstation
gerade keinen anderen RTS registriert, akzeptiert die Basisstation diesen
Request und sendet eine Clear-To-Send Nachricht. Dieser CTS geht sowohl an
Station $A$ als auch Station $B$ (in dieser CTS Nachricht muesste stehen, wem
der Kanal nun freisteht, falls beide RTS gesendet haben). Station $A$ darf dann
also senden. Station $B$ weiss dann auch, dass es zuerst eine feste,
vordefinierte Zeitspanne warten muss, bis es ueberhaupt einen RTS an die
Basisstation senden darf.

Die Vorteile davon sind, dass Kollisionen so natuerlich besser vermieden werden
koennen. Kollisionen sind aber dennoch nicht gaenzlich ausgeschlossen, z.B. wenn
Station $B$ das CTS nicht empfangen haette (wir betrachten hier vor allem
drahtlose Netzwerke) und durch einen eigenen RTS die Nachricht von $A$
zerstoert. Auch reduziert RTS/CTS die Datenrate natuerlich.

Ein konkretes Problem, dass CSMA/CA loest, ist das sogenannte *Hidden-Station*
Problem. In dieser Situation haben wir neben der Basisstation zwei Stationen $A$
und $B$, welche sich nicht sehen. Das heisst, sie sind beide gleich weit von der
Basisstation weg, aber auf unterschiedlichen Seiten. Nun kann es eben vorkommen,
dass beide Stationen zur Basisstation senden (da fuer beide das Medium auch frei
waere). Dann wuerden weder Station $A$ noch $B$ eine Kollision erkennen, da die
Nachrichten von $A$ und $B$ nicht zu der jeweils anderen Station
durchkommen. Sie koennten also z.B. kein JAM Signal senden. Dennoch wuerde es
aber spaetestens bei der Basisstation zu einer Kollision kommen. Hier sieht man
also, wo CD (Collision Detection) scheitern wuerde. Daher braucht man eben
Collision Avoidance (CA).

Anmerkungen:

* Grundsaetzlich ist CA nur das Prinzip, bei Funknetzwerken $p$-persistentes
  CSMA zu verwenden. RTS/CTS ist nur eine Erweiterung davon.
* RTS/CTS ist Bestandteil von *Virtual Carrier Sensing*, da das Medium durch ein
  CTS exklusiv fuer eine Zeitspanne reserviert wird.
* Damit RTS/CTS Nachrichten nicht so leicht verloren gehen oder verfaelscht
  werden, werden sie so robust wie moeglich kodiert. Das bedeutet auch die
  hoechste moegliche Redundanz bzw. die kleinste Datenrate.
* RTS/CTS geht auch ohne Basisstation und kann direkt durch die Geraete selbst
  gesteuert werden. Das nennt man dann *ad-hoc Modus*. Die Geraete im *ad-hoc
  Modus* nennt man dann *Service Set*.

\subsubsection{Token Passing} 

Ein weiteres Verfahren, dass zwar nicht CSMA (weder carrier sense) ist, aber
dennoch CA anbietet, ist *Token Passing*. Hierbei werden Stationen zu einem
physikalischen Ring zusammengeschlossen, sodass Stationen also direkt nur mit
ihren zwei Nachbarn kommunizieren koennen. Es zirkuliert dann immer ein *Token*
durch den Ring. Wenn eine Station Daten anliegen hat, wartet es, bis der Token
bei seiner Zirkulation bei ihm ankommt. Es nimmt den Token dann aus der
Zirkulation und reserviert also quasi den ganzen Ring exklusiv fuer sich. Andere
Knoten koennen den Ring also dann gerade nicht mehr erhalten. Dann sendet es
seine Nachrichten und gibt den Ring danach wieder fuer Zirkulation frei (oder
nach einem bestimmten Timeout).

Wenn eine Station eine Nachricht also sendet, zirkuliert diese auch wieder durch
den Ring. Der Empfaenger markiert die Nachricht dann als gelesen und sendet sie
weiter durch den Ring. Kommt die Nachricht so wieder beim Sender, gilt die
Uebertragung als erfolgreich und der Sender nimmt sie vom Netz.

Eine der Stationen im Netz agiert dabei immer als *Monitor Station*. Sie ist
fuer spezielle Aufgaben zustaendig:

* Wenn der Token verloren geht, generiert sie einen Neuen.
* Wenn mehr als ein Token zirkuliert, nimmt es die uebrigen vom Ring raus.
* Wenn Nachrichten endlos zirkulieren, beispielsweise wenn es an gar keinen
  Rechner im Netz richtig adressiert ist, nimmt es dies auch vom Ring.

Faellt diese Monitor Station aus, wird aus den verbleibenden eine neue
ausgewaehlt.

Token Passing hat den Vorteil, ueberhaupt keine kollisionsbedingten
Wiederholungen zu benoetigen, da es nie zu Kollisionen kommen kann. Die
maximale Verzoegerung von Nachrichten ist durch die feste Ringstruktur auch
deterministisch.

Die Nachteile von Token Passing sind aber auch vielfaeltig. Zum einen muss eben
eine Station spezielle Aufgaben uebernehmen. Auch ist es so, dass wenn ein
Knoten im Ring ausfaellt, die *gesamte* Kommunikation im Ring gestoert ist. Die
Uebertragungsverzoegerung ist oft auch groesser als bei CSMA, da der Sender auf
das Token erst warten muss. Auch ist es oft schwer, Stationen im Netzwerk
ueberhaupt zu einem Ring zusammenzuschliessen. Es wird daher heute fast gar
nicht mehr genutzt.

https://en.wikipedia.org/wiki/Token\_ring

\subsection{Rahmen} 

Wir beschaeftigen uns nun mit *Rahmen*, womit wir einfach Nachrichten auf der
Sicherungsschicht meinen. Konkret sehen wir uns an, wie Rahmen gebildet werden,
wie Empfaenger in Rahmen adressiert werden und Fehler in Rahmen erkannt werden
koennen.

\subsubsection{Erkennung von Rahmengrenzen} 

Fuer die physikalische Schicht sind Nachrichten lediglich eine Folge von
Bits. Natuerlich haben Nachrichten, zumindest fuer uns, aber *Grenzen*. Diese
muss man also aus dem Bitstrom heraus irgendwie erkennen koennen.

Hierbei gibt es verschiedene Moeglichkeiten:

* *Laengenangabe* vor der Uebertragung der eigentlichen Daten,
* *Steuerzeichen*,
* *Begrenzungsfelder* und "Bit-Stopfen" oder
* *Coderegelverletzungen*.

Jedenfalls wollen wir die Rahmengrenzen so bilden, dass deswegen nicht weniger
moegliche Daten uebertragbar sind.

\paragraph{Laengenangaben} 

Eine Idee zur Festlegung von Rahmengrenzen waere es, vor jeder Nachricht die
Groesse der Nachricht zuerst zu senden. Wenn man also z.B. $30$ Bit an Daten
sendet, sendet man die Zahl $30$ vor den eigentlichen Daten. Dann weiss der
Empfaenger, wieviele Bits er erwarten kann. Das setzt natuerlich vorraus, dass
dieses Laengenfeld eindeutig aus dem Datenstrom zu erkennen ist (zumindest fuer
die erste Nachricht nach einer Idle-Phase).

\paragraph{Steuerzeichen} 

Schon in der physikalischen Schicht wurde der Begriff von *4B5B Codes*
eingefuehrt, z.B. zur Fehlererkennung bei der Kanalkodierung. Dieses Prinzip,
jeweils $k$ Bits auf $n$ Bits abzubilden um Platz fuer Steuerzeichen zu machen,
koennen wir auch zum Eingrenzen von Rahmen verwenden.

Beispielsweise bedeute ein gesetzter MSB, dass nun ein Steuerzeichen kommt. Dann
koennten wir einen Code waehlen, wobei 10001 das erste Startzeichen ist und
10010 das zweite (zwei, damit ein Fehler in den Daten nicht so leicht zu einer
faelschlichen Startsequenz fuehren kann). 10011 sei dann das erste Stopsymbol
und 10100 das zweite. So koenenn wir also den Beginn und das Ende eines Rahmens
signalisieren.

In der Praxis heissen diese beiden Start-Signale oft $J$ und $K$. Wie gesagt
gibt es *zwei* separate Startsignale $J/K$, nicht weil man zwei *braeuchte*,
sondern einfach, weil es robuster ist (z.B. wegen Synchronisation) zwei
Startsymbole zu senden. Somit kann im ersten Symbol ein Bit umkippen, ohne dass
ein falscher Rahmen erkannt wird, auch wenn das durch den Fehler entstandene
Symbol eines der J/K Symbole ist. Dadurch muss man die Hardware nicht unbedingt
praeziser machen, weil Leitungszeit eigentlich billiger ist als praezise
Hardware (wo weniger Bits umkippen).

FastEthernet (100 MBit/s) benutzt MLT-3 mit 4B5B Codes und signalisiert Start
und Ende durch Steuerzeichen.

Auch auf der Darstellungsschicht (6) werden Steuerzeichen verwendet,
beispielsweise beim ASCII-Code (`\\0`, `\\a`, `\\n` etc.) oder in Emacs <3 (`M-x C-l`).

\paragraph{Begrenzungsfelder und Bit-Stopfen} 

Bei dieser Idee markieren wir den Start und das Ende einer Nachricht durch eine
bestimmte Bitfolge, beispielsweise 01110. Dann muessen wir nur noch dafuer
sorgen, dass diese Bitfolge nicht in den Daten auftritt und faelschlicherweise
das Ende einer Nachricht signalisiert. Das machen wir durch *Bit-Stuffing*. Wir
fuegen dabei (fuer dieses Beispiel) nach zwei Einsen immer eine Null ein
(011010), dann kann in den Daten die spezielle Sequenz nie vorkommen. Der
Empfaenger weiss dann, dasss wenn er zwei Einsen empfaengt und dann:

* eine 1, so handelt es sich um eine spezielle Sequenz (vorausgesetzt die 0
  hinten und vorne war da / kommt noch)
* eine 0, so kommt diese vom Bit-Stopfen und wird wieder entfernt.

Man muss hierbei auch nicht darum kuemmern, die Sequenz 011010 zu escapen. Nach
zwei Einsen wuerde naemlich eben eine 0 eingefuegt (0110010) und beim Empfaenger
wieder entfernt.

\paragraph{Coderegelverletzung} 

Wie schon bei der Leitungskodierung beschrieben, kann man die Tatsache
ausnutzen, dass bestimmte Kodierungsverfahren wie RZ, NRZ und Manchester
Encoding strenge Regeln fuer Signalwerte haben (z.B. niemals Null bei NRZ)
haben. Durch Verletzung dieser Regeln konnen wir den Start oder das Ende einer
Nachricht signalisieren.

Normales Ethernet (10 MBit/s) benutzt diese Methode. Es verwendet dabei
Manchester Encoding als Leitungscode und signalisiert das Ende eines Frames
durch eine Coderegelverletzung (also bei Manchester Encoding einfach Null). Der
Start wuerde auch einfach durch das erste Symbol signalisiert, da davor nur
Nullen kamen und bei Manchester Encoding auch erst der zweite Edge das Symbol (0
oder 1) signalisieren wuerde (der erste Edge bringt das Signal erst einmal auf +
1 oder -1).

\subsubsection{Adressierung} 

Wir interessieren uns nun dafuer, wie Stationen eigentlich adressiert werden
koennen. Dabei betrachen wir vorerst *Direktverbindungsnetze*, wo also jeder
Rechner mit jedem anderen Rechner direkt (durch ein Kabel) verbunden ist. Dabei
findet also keine Vermittlung (Routing) statt. Das kommt erst auf Schicht 3.

Auf der Sicherungsschicht adressieren wir Knoten im Netzwerk durch
*MAC-Adressen*. MAC steht dabei fuer *Media Access Control*. Wir haben folgende
Anforderungen an solche Adressen:

* Sie sollten Rechner in einem Direktverbindungsnetz *eindeutig identifizieren*,
* Es soll eine *Broadcast-Adresse* geben, welche *alle Knoten* im Netzwerk
  adressiert, sowie
* *Multicast-Adressen*, die bestimmte Gruppen von Knoten im Netzwerk ansprechen.

Netzwerkkarten fuer alle IEEE 802 Standards haben ab Werk (Produktion) eine
fixe, festgelegte MAC-Adresse. Diese wird meist im ROM der Netzwerkkarte
gespeichert. MAC Adressen kann man aber auch aendern. MAC-Adressen haben dabei
den folgenden Aufbau:

* Sie sind 48 Bit (6 Byte) lang.
* Sie bestehen aus einem:
  1. 24-Bit *OUI* (*Organziationally Unique Identifier*) und
  2. 24-Bit *NIC* (*Network Interface Controller* bzw. einfach Device ID).
* Der OUI referenziert dabei einen bestimmten Hersteller von Netzwerkkarten,
  beispielsweise Apple.
* Eine MAC-Adresse wird in 6 Paaren von Hexadezimalziffern angegeben, welche
  jeweils durch einen Doppelpunkt getrennt werden. Beispielsweise waere
  `de:ad:be:ef:af:fe` eine gueltige MAC-Adresse.
* Nur Einsen, also `ff:ff:ff:ff:ff:ff` bezeichnet dabei die Broadcast-Adresse.
* Der LSB des ersten Oktetts sagt, ob die Adresse *Unicast* (0) oder *Multicast*
  (1) ist.
* Der erste Bit (links vom LSB) des ersten Oktetts bestimmt, ob die MAC-Adresse
  *globally unique* (0) oder *locally administered* (1) ist. Letzteres ist
  *z.B. bei virtualisierten* Netzwerkadaptern nuetzlich. Ist dieser Bit gesetzt,
  *bedeutet dass eine locally unique Adresse.

MAC Adressen werden von den meisten IEEE 802 Protokollen verwendet
bzw. unterstuetzt, insbesondere Ethernet, WiFi und Bluetooth
(https://en.wikipedia.org/wiki/MAC\_address).

\subsubsection{Fehlererkennung} 

Ein typisches L2 (z.B. Ethernet) Rahmen besteht aus den folgenden Teilen:

1. Destination MAC-Adresse,
2. Source MAC-Adresse,
3. Kontrollinformation (z.B. die Art des Pakets/der Nachricht),
4. Die eigentliche Payload (z.B. einem IP-Paket)
5. Einer *Checksum*, auch *Frame Check Sequence* (*FCS*) genannt.

Mit dem letzten Punkt, der Checksum bzw. der FCS, wollen wir uns nun
beschaeftigen. Die Checksum, zu Deutsch *Pruefsumme*, ist ein
\_\_fehlererkennender Code\_\_. Im Gegensatz zur Kanalkodierung wollen wir hier
Fehler nur *erkennen* und nicht *korrigieren*.

Ein konkretes Beispiel fuer eine Checksum, die z.B. bei Ethernet verwendet wird,
ist der *Cyclic Redundancy Check* (CRC). Das Ziel des CRC ist es:

* Moeglichst viele Bitfehler zu erkennen,
* Wenig Redundanz zu den Nutzdaten hinzufuegen zu muessen,
* Fehler zu erkennen, aber nicht korrigieren zu koennen.

Frage: Welche Fehler kann CRC korrigieren?
Antwort: Gar keine, CRC kann Fehler nur *erkennen*!

Der CRC wird bei Ethernet hierbei auf der ganzen PDU, also inklusive
Ether-Header berechnet.

\paragraph{Theorie} 

Fuer CRC benutzen wir die Eigenschaft von Bit-Sequenzen, auch Polynome
darstellen zu koennen. Ein Datenwort der Laenge $n$ Bit gibt dabei ein Polynom

$a(x) = \sum_{i=0}^{n - 1} a_i x^i$

vom Grad $n - 1$ an, wobei $a_i \in \{0, 1\}$ ($n$ Bit = Grad $n - 1$ weil bei
Grad $n - 1$ auch $x^0$ als konstanter Term enthalten ist). Jeder Term (jede
Potenz) kann also enthalten sein, oder nicht, je nachdem, welchen Wert der
$i$-te Bit hat. Wenn in einer Bit-Sequenz an Stelle $i$ eine Eins steht, so ist
der Term $x^i$ enthalten. Steht an Stelle $i$ eine Null im Datenwort, so ist
$x^i$ nicht enthalten (bzw. $0 \cdot x^i$ wuerde eben Null
ergeben). Beispielsweise entspricht die Bit-Sequenz $10101001_2$ dem Polynom
$a(x) = x^7 + x^5 + x^3 + x^0$. Hierbei ist $x^0$ natuerlich gleich $1$. Wie man
auch sieht ergeben 8 Bit hier ein Polynom von maximalem Grad 7.

Wir koennen nun alle Polynome vom Grad $n$ in einer Menge

$F_q[x] = \left\{a \,|\, a(x) = \sum_{i=0}^{n-1} a_ix^i, a_i \in \{0,
1\}\right\}$

zusammenfassen, welche genau $q = 2^n$ Elemente hat (jeder Term kann enthalten
sein, oder nicht / jeder Bit kann gesetzt sein, oder nicht). Wenn wir nun noch
die Addition und Mulitplikation auf dieser Traegermenge passend definieren,
erhalten wir einen endlichen Koerper $\langle F_q[x], +, \cdot \rangle$.

Die "passende" Addition fuer den Koerper ist dabei die Addition der
Polynomkoeffizienten, also den Bits $\in \{0, 1\}$, nach den Regeln des Galois
Field $GF(2)$, also dem additiv-multiplikativen Koerper modulo $2$. Das heisst
folgendes:

1. $0 + 0 = 0$
2. $0 + 1 = 1 + 0 = 1$
3. $1 + 1 = 2 \equiv_2 = 0$
4. $1 - 1 = 0$
5. $0 - 1 = -1 \equiv_2 1$
6. Addition ist also das selbe wie Subtraktion,
7. Und genau gleich der logischen XOR-Verknuepfung.
8. Der resultierende Grad des Polynoms / Laenge der Bitsequenz ist kleiner oder
   gleich, niemals groesser als die Grade/Laengen der Operanden.

Beispiel: (Achtung: Die Bits sind Koeffizienten von Polynomen, also kein Carry!)

```
10101110
11101010
--------
01000100
```

```
10101110 XOR
11101010
--------
01000100
```

Die "passende" Multiplikation ist etwas komplexer definiert, da die
Multiplikation zweier Bitsequenzen der Laenge $n$ maximal eine Bitsequenz der
Laenge $2n$ gibt, sodass also der Grad des resultierenden Polynoms nicht mehr
kleiner $n$ ist und das Polynom folglich auch nicht in $F_q[x]$ enthalten
waere. Wir muessen also das Produkt zweier Polynome $a(x)$ und $b(x)$ immer
modulo einem *Reduktionspolynom* $r(x)$ vom Grad $n$ rechnen, sodass der Grad
des Produkts auch kleiner $n$ bleibt. Wir multiplizieren $a(x)$ mit $b(x)$ dabei
ganz normal, fuehren dann aber eine Polynomdivision mit $r(x)$ durch. Der Rest
der Division ist dann gerade der Wert von $a(x) \cdot b(x) \mod r(x)$, welchen
wir wollen.

Waehlt man hierbei $r(x)$ als *irreduzibles* Polynom, welches sich also nicht
als Produkt zweier anderer Polynome darstellen laesst (Primzahl im
Polynombereich), erhaelt man einen endlichen Koerper. Das wollen wir aber bei
CRC nicht. Waehlen wir naemlich ein reduzibles Polynom, haeufig $p(x)(x + 1)$ wo
$p(x)$ irreduzibel ist, erhalten wir einen nicht-endlichen Koerper. Dadurch
koennen wir dann alle ungeradezahligen Fehler erkennen (1 Bit falsch, 3 Bit
falsch, ...). Endlichkeit wuerde naemlich bedeuten, dass eben bestimmte Polynome
aequivalent zu einem anderen Polynom sind, was wir nicht wollen (sonst koennten
fehlerhafte Bitsequenzen den urspruenglichen ident sein, was das ganze Konzept
von Fehlererkennung zu Nichte machen wuerde).

\paragraph{Praxis} 

Konkret machen wir bei CRC nun folgendes: Wir wollen also eine $n$-Bit Checksume
(Frame Check Sequence) berechnen, beispielsweise fuer die L2-Payload eines
Ethernet Frames. Die Daten bzw. die Codewoerter der Laenge $N$ Bit sind also
$\in F_q[x]: q = 2^N$. Dann waehlen wir uns ein Reduktionspolyom $r(x)$ vom Grad
$n$ ($n < N$), sodass der Rest der Division unserer Daten mit diesem
Reduktionspolynom also maximal Grad $n - 1$ hat und somit genau $n$ Bit
benoetigt ($x^0, ..., x^{n - 1}$).

Um nun die CRC-Checksum $c(x)$ unserer Daten $d(x)$ zu erhalten, multiplizieren
wir unsere Daten $d(x)$ zunaechst mit $x^n$, wo $n = grad(r(x))$. Das ist
aequivalent dazu, die Bitsequenz $d(x)$ (also unser Datenwort), um $n$ Bit nach
links zu verschieben, da $x^n \equiv 1000...000$ mit $n$ Nullen ist. Das tun
wir, um Platz fuer die Checksum zu machen, welche ja $n$ Bit lang ist.

Dann fuehren wir die Polynomdivision $d(x) div r(x)$ durch, wobei wir uns aber
fuer den Rest $d(x) mod r(x)$ interessieren. Dieses Restpolynom ist naemlich
gerade unsere Checksum $c(x)$. Also $c(x) = d(x) mod r(x)$. Es hat eben maximal
Grad $n - 1$ und benoetigt somit $n$ Bit Platz. Wir addieren diese Checksum
auf das verschobene Polynom drauf bzw. fuehren eine XOR-Operation auf der
verschobenen Bit-Sequenz mit der Rest-Bit-Sequenz durch. Die Nachricht $s(x)$,
welche wir dann senden, ergibt sich also aus $s(x) = d(x) \cdot x^n + c(x)$.

Wir senden diese Nachricht $s(x)$ also. Der Empfaenger macht dann folgendes:

1. Er kennt das selbe Reduktionspolynom $r(x)$.
2. Fuehrt dann auf den Daten wieder eine Polynomdivision durch.
3. Ist der Rest $s(x) mod r(x) = 0$, so ist mit *hoher Wahrscheinlichkeit* kein
   Uebertragungsfehler passiert.
4. Ist der Rest ungleich Null, so ist *sicher* ein Fehler aufgetreten.

Wieso ist das so? Nachdem wir $d(x)$ mit $x^n$ multipliziert haben, bleibt bei
der Division also $c(x)$ Rest. Das bedeutet, dass bei der Polynomdivision unten
$c(x)$ stand. Da wir bei $s(x)$ aber gerade dieses Glied $c(x)$ noch
draufaddiert haben, wuerde bei einer erneuten Polynomdivision von $s(x)$ mit
$r(x)$ dieser Rest sich mit dem identen Teil aus $s(x)$ wegkuerzen, wodurch
sich 0 Rest ergibt. Wir runden das Polynom $d(x) \cdot x^n$ also quasi bis zum
naechsten Vielfachen von $r(x)$ auf.

Tritt ein Fehler bei der Uebertragung auf, wuerde also nicht Null Rest
bleiben. Ausser natuerlich in besonders unguenstigen Faellen. CRC kann bestimmte
Fehler naemlich nicht erkennen. Einen Fehler nicht zu erkennen bedeutet hierbei,
dass der Rest also trotz Veraenderung von $s(x)$ waehrend der Uebertragung Null
ist. Wir betrachten Fehler dabei immer als *Fehlermasken* der selben Laenge wie
$s(x)$, welche auf $s(x)$ draufaddiert/ge-XOR-ed werden. Die Fehler, die es
*nicht verlaesslich* erkennen kann, sind:

* Fehler, die laenger als $n$ sind.
* Fehler, die aus mehreren Bursts (isolierten Gruppen von mehr als einem Fehler-Bit)
  bestehen

Fehler, die auf Grund des Aufbaus von CRC *sicher nicht* erkannt werden koennen,
sind solche, die ein Vielfaches von $r(x)$ sind. Denn dabei wuerde der Rest
weiter gleich bleiben. Hierbei meint Vielfaches eine Mulitplikation mit einem
Term $x^k$ bzw. eine Verschiebung um $k$ Bits. $s(x)$ ist naemlich schon ein
Vielfaches von $r(x)$, also $s(x) = k(x) \cdot r(x)$ fuer ein Polynom
$k(x)$. Wenn wir nun ein weiteres Vielfaches Polynom $f(x) = k'(x) \cdot r(x)$
von $r(x)$ auf $s(x)$ draufaddieren, ist $s(x) + f(x) = (k(x) + l(x)) \cdot
r(x)$ natuerlich auch noch ein Vielfaches von $r(x)$, weswegen der Rest weiter
null waere. Eine Fehlermaske, die von CRC also sicher nie erkannt werden wuerde,
ist $r(x)$ selbst.

CRC erkennt jedoch immer:

1. Alle 1-Bit Fehler,
2. Alle isolierten 2-Bit Fehler,
3. Einige Burst-Fehler, die laenger als $N$ sind.

\paragraph{Beispiel} 

Wir betrachten ein Datenpolynom $d(x) = x^7 + x^5 + x^2 + 1$ von Grad $7$ sowie
ein Reduktionspolynom $r(x) = x^3 + x^2 + 1$ von Grad $n = 3$. Das entspricht
Bitsequenzen $10100101$ fuer die Daten und $1101$ fuer das
Reduktionspolynom. Die Checksum wird drei Bit lang sein. Dann:

1. Wir multiplizieren $d(x)$ mit $x^n = x^3$, verschieben die Bitsequenz also um
   3 Bit nach links und fuellen Nullen nach. Daraus ergibt sich: $10100101000$
2. Dann fuehren wir die Polynomdivision durch. Hierbei verwenden wir die
   XOR-Regeln als Addition (wobei Subtraktion = Addition im GF(2)):

\begin{verbatim}
10100101000 : 1101 = 11010101
1101 (XOR) (da die erste Ziffer im Polynom schon 1, ergibt oben erste 1)
----
0111 (da die 2. Ziffer eine 1 ist, ergibt es oben eine 1)
 1110 (schieben eine Ziffer nach, um genug Bits fuer die Maske zu haben)
 1101
 ----
 001110 (da die 2. Ziffer 0 ist, oben eine 0, und einen Bit nach schieben)
   1101 (dann geht 1 in 1 rein und es kommt oben eine 1 rein)
   ----
   001110
     1101
	 ----
	 001100
	   1101
	   ----
	   0001 = c(x)
```

Nun addieren wir unsere Checksum $c(x) = 1$ auf unser verschobenes Polynom
bzw. XOR-ren 001 auf unsere verschobene Bitsequenz (da diese unteren Bits vorher
Null waren schreiben/OR-en wir sie eigentlich einfach rein):

```
    10100101000 = d(x) * x^n
XOR 00000000001 = c(x)
    -----------
    10100101001 = s(x)
\end{verbatim}

Der Empfaenger macht nun mit $s(x)$ wieder eine Polynomdivision:

```
10100101001 : 1101 = 11010101
1101
----
01110
 1101
 ----
 001110
   1101
   ----
   001110
     1101
	 ----
	 001101 <=== Hier ist der wichtige Unterschied
	   1101
	   ----
	   0000 <=== Kein Fehler
```

So sieht der Empfaenger also, dass kein Fehler aufgetreten ist. Nun geben wir
eine Fehlermaske `01001110001` zur Nachricht hinzu, was beispielsweise bei einer
Uebertragung (leider) passieren koennte:

```
10100101001
01001110001 XOR
-----------
11101011000 <= fehlerhafte Daten
```

Dann pruefen wir beim Empfaenger die Checksum (den ganzzahligen Teil der
Division brauchen wir in der Praxis uebrigens gar nicht zu beachten)

```
11101011000 : 1101 = 1010101
1101
----
001110
  1101
  ----
  001110
    1101
	----
	001100
	  1101
	  ----
	  0001
```

Der Rest ist nicht Null, der Empfaenger erkennt also, dass ein Fehler passiert
ist! Nun waehlen wir eine besonders unguenstige Fehlermaske, zum Beispiel $x
\cdot r(x)$, also `00000011010` (um einen Bit nach links geshiftet) und sehen,
was passiert. Zuerst bestimmen wir die fehlerhaften Daten:

```
10100101001
00000011010 XOR
-----------
10100110011 <= fehlerhafte Daten
```

Und fuehren wieder beim Empfaenger die Polynomdivision durch:

```
10100110011 : 1101
1101
----
01110
 1101
 ----
 001111
   1101
   ----
   001000
     1101
	 ----
	 01011
	  1101
	  ----
	  01101
	   1101
	   ----
	   0000 Rest
```

Wie man sieht, haben wir hier eine Fehlermaske zu den Daten hinzu getan, wobei
der Empfaenger dies gar nicht merken wuerde.

\subsubsection{Links} 

Wir wollen nun die verschiedenen Arten von Verbindungen oder *Links* auf Schicht
2 betrachten. Dabei gibt es grundsaetzlich zwei Varianten:

1. Hubs
2. Switches

Die naechsten Absaetze besprechen diese weiterfuehrend.

\paragraph{Hubs} 

Ein Hub ist ein enorm simpler Baustein in einem Direktverbindungsnetz, an
welchen mehrere Knoten angebunden sein koennen. Sendet ein Knoten $A$ an einen
Hub eine Nachricht, so wird diese an *jeden* anderen Knoten, der an diesem Hub
angeschlossen ist, weitergeleitet. Somit darf zu jedem Zeitpunkt auch immer nur
ein Knoten an einen Hub senden. Treffen mehr als eine Nachricht bei einem Hub
ein, kollidieren diese.

Sprechen wir also von *Kollisions-Domaenen*, so sind alle Knoten, die an einen
Hub angebunden sind, in einer Kollisions-Domaene. Dabei ist eine
Kollisions-Domaene, auch *Segment* genannt, ein Teil eines
Direktverbindungsnetzes, innerhalb welchem immer eine Kollision auftritt, sobald
gleichzeitig mehr als ein Knoten sendet.

Es gibt hier noch zwei Arten von Hubs:

* Aktive Hubs (Repeater): Diese verstaerken (amplifizieren) das Signal auf der
  physikalischen Schicht, ohne dabei die Rahmen irgendwie zu inspizieren. Sie
  sind fuer den Sender also *transparent* (er weiss nicht, dass es ihn gibt).
* Passive Hubs: Solche Hubs sind wirklich nur Sternverteiler -- man keonnte
  genauso gut die einzelnen Adern der Kabel verloeten.

Hubs werden in modernen Netzwerken aufgrund der impliziten Kollisionsdomaene
fast gar nicht mehr verwendet. Ein Switch (siehe unten) ist heutzutage billig
genug, um nur Switches und nicht Hubs zu verwenden.

\paragraph{Switch} 

Die zweite Art von Verbindungsknoten auf Schicht 2 sind *Switches*. Diese sind
um einiges intelligenter als Hubs. Ein Switch hat dabei verschiedene *Ports*,
auf welchen jeweils ein oder viele Hosts (wenn ein Hub angeschlossen ist)
angeschlossen sind. Der Switch hat dann das Ziel, zu lernen, auf welchen Ports
welche Hosts senden. Er wuerde dabei die MAC-Adressen der Hosts in seinen
*Switching-Table* einschreiben. Weiss der Switch erst einmal, wo alle Hosts
leben, kann er einen eingehenden Rahmen nur durch den Port weiterleiten, wo der
Destination Host laut Switching Table angemeldet ist. Dazu braucht der Switch
initial eine *Learning Phase*. Er arbeitet dabei wie folgt:

1. Kommt ein Rahmen bei einem Port an und der Switch hat die Source MAC-Adresse
   noch nicht in seine Switching Table eingetragen, so gibt er einen Eintrag
   fuer diese MAC-Adresse und den Eingangsport in den Table.
2. Dann prueft der Switch, ob die Destination MAC-Adresse in seinem Switching
   Table ist (initial ist dieser leer).
3. Wenn ja, sendet er den Rahmen nur an diesen Port. Das tut er dabei aber nur,
   wenn der Zielhost-Port nicht gleich dem Quellhost-Port ist, da das sonst
   bedeutet, dass in diesem Quell/Ziel Netzsegment irgendwo ein Hub ist und der
   Zielhost den Rahmen sowieso schon ueber diesen Hub bekommen hat (der Switch
   verwirft den Rahmen also einfach).
4. Wenn nicht, *broadcastet* er den Rahmen an alle seine Ports.

Eintraege in der Switching Table werden dabei mit einem Zeitstempel versehen und
periodisch invalidiert.

Was passiert nun, wenn mehrere Hosts zum Switch senden? Sofern diese Hosts alle
an verschiedene Ports des Switch gesendet haben, gar nichts bzw. nur Gutes! Der
Switch kann naemlich auf jedem Port ein Paket abfangen, eventuell kurz speichern
(puffern) und dann weitersenden. Switches haben daher die Eigenschaft,
Kollisiondomaenen zu *unterbrechen*. Betrachten wir beispielsweise einen Switch
mit nur zwei Ports -- auch *Bridge* genannt --- und je zwei Rechner auf beiden
Seiten, die durch je einen Hub an den Switch verbunden sind:

```
A         C
 \ 0   1 /
  H--S--H
 /       \
B         D
```

Es darf nun also auf jeder Seite, Port 0 und Port 1, gleichzeitig ein Host
senden. Der Switch wird die Pakete dann jeweils weiterleiten. Somit unterbricht
der Switch also die Kollisiondomaene in der Mitte:

\begin{verbatim}
     |
     |
     |
A    |    C
 \ 0 | 1 /
  H--S--H
 /   |   \
B    |    D
     |
     |
	 |
\end{verbatim}

Wenn links oder rechts zwei Hosts gleichzeitig senden, z.B. $A$ und $B$, kommt
es noch immer zu einer Kollision. Wenn aber ein Host links und ein Host rechts
sendet, ist das kein Problem.

Ist pro Switchport genau ein Host angeschlossen, so spricht man von
*Microsegmentation* bzw. einem *vollstaendig geswitchem Netz*. Das ist heute der
Regelfall. Dann duerfen also je zwei Hosts gleichzeitig miteinander
kommunizieren (also, jeder mit jedem).

\begin{verbatim}
    |
  A | C
   \|/
----S----
   /|\
  B | D
    |
\end{verbatim}

Switches koennen im Uebrigen auch dazu genutzt werden, Netzteile mit
verschiedenen Zugriffsverfahren zu verbinden! Beispielsweise kann am einen Port
CSMA/CD (Ethernet) verwendet werden und auf dem anderen Port Token Passing
(FDDI). Das ist fuer alle angeschlossenen Rechner bzw. Netzwerke transparent.

Anmerkungen:

* Switches sind wie Hubs fuer Hosts *transparent*, d.h. ein Host weiss nicht,
  dass zwischen ihm und seinem Zielhost ein Switch liegt.
* Sender- und Empfaengeradresse werden nie veraendert (nicht wie bei *Routern*).
* Ein Broadcast Paket wird an alle Ports weitergeleitet.

Es gibt auch noch zwei verschiedene Arten von Switches:

1. *Store-and-Forward*: Eingehende Rahmen werden vollstaendig empfangen und
   *deren FCS geprueft* (sowie dann gegebenfalls zurueckgesendet). Ist der
   Ausgangsport zum Zielhost belegt, koennen Rahmen gepuffert werden. Hierbei
   wird der vollstaendige Rahmen empfangen, bevor er weitergeleitet wird.
2. *Cut-Through*: Rahmen werden sofort, ohne Inspektion der FCS,
   weitergesendet. Insbesondere wird ein Rahmen schon weitergeleitet, sobald die
   Destination Address Bytes des L2-Headers (ersten 6 Byte) erhalten wurden.

https://en.wikipedia.org/wiki/Network\_switch\#Layer\_2

Weitere Anmerkung:

In einem normalen (billigen) Switch laeuft alles out-of-the-box und man kann
auch weiter nichts konfigurieren. Man muss sich also nicht mit ihnen verbinden
koennen weil man mit ihnen auch gar nicht kommunizieren kann
(bzw. muesste). Teure high-end Switches haben ein eigenes Betriebssystem und
Management-Protokoll, sodass man sich einloggen kann (z.B. ueber SSH) um sie zu
konfigurieren (z.B. um FCS Pruefung ein/auszuschalten oder Puffer-Groessen
festzulegen). Solche Switches, auch *Managed Switches* genannt, haetten dann
auch eine eigene IP-Adresse.

\section{OSI Modell: Layer 3} 

Auf Schicht 3, der *Vermittlungsschicht* bzw. dem *Network Layer*, beschaeftigen
wir uns nun schon mehr mit se Interwebs. Wir betrachten zunaechst
Vermittlungsarten, dann Adressierung im Internet und letztlich Routing.

\subsection{Vermittlungsarten} 

Es gibt drei verschiedene Moeglichkeiten, Daten von $A$ nach $B$ zu senden:

1. Leitungsvermittlung: Reserviere einen exklusiven Kanal fuer Vermittlung von
   $A$ nach $B$.
2. Nachrichtenvermittlung: Sende Nachrichten individuell.
3. Paketvermittlung: Teile eine Nachricht in kleinere *Pakete* auf und sende
   diese individuell, unabhaengig von den anderen.

Wir werden diese drei Arten nun naeher beleuchten. Dabei beschreiben wir die
Verfahren an Hand des folgenden Beispielpfades:

```
(s) --- (i) --- ... --- (j) --- (t)
```

Wobei $s, t$ Quell- bzw. Zielknoten sind und $i, ..., j$ Vermittlungsknoten $n$
darstellen. Dabei uebertragen wir immer eine Nachricht mit $L$ Bits ueber eine
Gesamtdauer ($s \rightarrow t$) von $T$. Die Knoten $s$ und $t$ liegen dabei $d$
Meter auseinander und alle Knoten sind aequidistant plaziert.

\subsubsection{Leitungsvermittlung} 

Bei der Leitungsvermittlung reservieren wir einen exklusiven Kanal von $s$ nach
$t$. Die Uebertragung besteht dabei aus drei Phasen:

1. Verbindungsaufbau: Hierbei tauschen $s$ und $t$ zunaechst
   Signalisierungsnachrichten aus, um die Verbindung zu initialisieren. In
   diesem Schritt wird auch schon die letztendliche Wegwahl von $s$ nach $t$
   genau ueber Knoten $i, ..., j$ getroffen.
2. Datenaustausch: Daten koennen nun ueber diesen exklusiven Kanal von $s$ nach
   $t$ und retour ausgetauscht werden. Das bedeutet, dass nur diese beiden
   Knoten diesen Kanal verwenden duerfen. Da der Kanal nur $s$ und $t$
   verbindet, kann die Adresseriung der Nachrichten waehrend der Uebertragung
   auch einfach ausfallen.
3. Verbindungsabbau: Signalisierungsnachrichten werden wieder auf beiden Seiten
   ausgetauscht, um nun das Ende der Uebertragung zu signalisieren. Danach
   duerfen also wieder neue Knoten $s'$ und $t'$ diesen Kanal (exklusiv) nutzen.

Wie hoch ist hier nun die Uebertragungszeit? Um sie zu bestimmen, nehmen wir
zunaechst an, dass:

1. Die Serialisierungszeit fuer *Signalisierungsnachrichten* vernachlaessigbar
   klein ist.
2. Die Wartezeiten fuer Nachrichten in jedem Vermittlungsknoten
   vernachlaessigbar klein sind.

Dann senden wir also zuerst eine Signalisierungsnachricht von $s$ nach $t$ sowie
retour, um die Verbindung zu starten (1). Dies kostet gerade $t_p = d/(\nu c)$
Zeit. Danach wird die eigentliche Nachricht versendet. Die serialisierten Daten
werden dabei von Knoten zu Knoten ohne Verzoegerung versendet. Also brauchen wir
nur die Serialisierungszeit der $L$ Bit bei einer Datenrate $r$ betrachten: $t_s
= L/r$. Der letzte Teil dieser Nachricht sei dabei die Signalisierung, dass die
Verbindung nun abgebrochen werden soll. Insgesamt kommt hier nun fuer die
serialisierte Nachricht und der Signalisierung noch einmal $t_p$ viel Zeit
hinzu. Letztlich "acknowledged" $t$ noch den Verbindungsabbruch mit einem
Signal, was wiederum $t_p$ Sekunden braucht, um von $t$ nach $s$ zu
kommen. Insgesamt haben wir also:

$$T_{LV} = 2t_p + t_s + 2t_p = t_s + 4t_p = \frac{L}{r} + 4\frac{d}{\nu c} .$$

Der einzige wirkliche Vorteil von Leitungsvermittlung ist, dass die Verbindung
nach Verbindungsaufbau *konstant schnell* ist. Nachrichten gehen naemlich immer
ueber den selben Weg gehen, weswegen keine aufwendigen
Vermittlungsentscheidungen zwischendrin getroffen werden muessen.

Nachteile sind jedoch:

* Ressourcenverschwendung, da die Leitung exklusiv reserviert wird (hohe
  Verzoegerung von Nachrichten anderer Knoten, die den Kanal waehrenddessen
  nicht nutzen koennen).
* Verbindungsaufbau kann komplex sein und lange dauern.

Leitungsvermittlung wird heute eher selten genutzt.

Analogie: Wir wollen ein Haus von Berlin nach Wien senden. Dafuer reservieren
wir die ganze Autobahn von Berlin nach Wien und senden unser Haus, Stein fuer
Stein, ueber diese Autobahn.

\subsubsection{Nachrichtenvermittlung} 

Bei der Nachrichtenvermittlung wird zwischen Sender und Empfaenger kein
exklusiver Kanal mehr allokiert. Stattdessen muss jede einzelne Nachricht von
$s$ nach $t$ einzeln adressiert werden und einzeln an den Empfaenger gesendet
werden. Dabei muss also auch immer fuer jede Nachricht ein neuer, womoeglich
anderer, Weg gewaehlt werden. Was wir uns aber erhoffen wuerden, ist, Zeit fuer
Verbindungsaufbau und -abbau zu sparen. Auch soll der Kanal fairer genutzt
werden.

Da ein Kanal (z.B. eine Reihe von Switches bzw. Routern) nicht mehr exklusiv
fuer Sender und Empfaenger reserviert ist, kann es auf einem Kanal also mehrere
moegliche Empfaenger geben. Insofern muss fuer jede Nachricht auch irgendwie
klargestellt werden, an welchen von diesen Empfaengern sie denn nun gehen
soll. Wir muessen die Nachricht also *adressieren*. Dafuer wird der $L$-bittigen
Nachricht ein Header der Laenge $L_H$ vorangestellt, welcher exakt diese
Adressinformationen beinhaltet. Das waere bei einem Brief beispielsweise eben
die Addressangabe des Empfaengers. Hierbei ist diese Adresse dann meist auch
global eindeutig.

Wir wollen die Uebertragungszeit einer Nachricht von $s$ nach $t$
herausfinden. Hierbei sei angemerkt, dass bei jedem Knoten auf dem Weg von $s$
nach $t$ immer die gesamte serialisierte Nachricht ankommen muss, bevor sie zum
naechsten Knoten weitergesendet werden kann. Das bedeutet dann also, dass wir
fuer $n$ Vermittlungsknoten die Nachricht $n + 1$ mal vollstaendig senden und
erhalten muessen. Dabei senden wir die Nachricht $n$ mal an einen
Vermittlungsknoten, und am Ende vom letzten Vermittlungsknoten zum Zielknoten
$t$ (deswegen $n + 1$). Da die Nachricht wie gesagt jedes mal vollstaendig
ankommen muss, dauert es immer $t_s$ Sekunden, bis die Nachricht beim vorherigen
Knoten serialisiert ist. Man beachte, dass hier neben der Nachricht natuerlich
auch noch der Header mitgesendet werden muss! Da wir aber nur die Nachricht
senden, und keine Signalisierungsnachrichten, haben wir die
Ausbreitungsverzoegerung $t_p$ nur einmal. Hierbei genuegt es, die
Ausbreitungsverzoegerung ueber die gesamte Distanz von $s$ nach $t$ zu
betrachten und nicht von jedem Knoten zu jedem (intermediaeren) Knoten, da diese
sich sowieso aus den Ausbreitungsverzoegerungen zwischen den Knoten ergeben
wuerde. Das ergibt dann fuer die Uebertragungszeit $T_{NV}$ bei der
Nachrichtenvermittlung:

$$T_{NV} = (n + 1) \cdot t_s + t_p = (n + 1) \cdot \frac{L_H + L}{r} +
\frac{d}{\nu c}$$

Da ein Kanal bei der Nachrichtenvermittlung nicht mehr exklusiv genutzt wird,
koennen also mehrere Knoten nacheinander Nachrichten ueber denselben Kanal
senden. Das entspricht dem *Zeitmultiplexing* (*Time Division Multiplex,
TDM*). Es koennen beispielsweise zwei Knoten $s_1$ und $s_2$ gleichzeitig eine
an $t$ adressierte Nachricht an einen Switch $i$ senden, der diese Nachrichten
dann *nacheinander* weitersenden wird (mit Betonung auf *nacheinander*).

Somit sind die Vorteile von Nachrichtenvermittlung zusammenzufassend:

* Flexibles *Zeitmultiplexing* von Nachrichten,
* Bessere Ausnutzung der Kanalkapazitaet (bei Leitungsvermittlung koennte der
  Kanal waehrend einer Pause gar nicht genutzt werden),
* Keine Verzoegerung durch Verbindungsaufbau beim Senden der ersten Nachricht.

Nachteile von Nachrichtenvermittlung sind:

* Nachrichten muessen fuer TDM moeglicherweise bei den Vermittlungsknoten
  gepuffert werden.
* Sind die Puffer der Vermittlungsknoten in ihrer Kapazitaet beschraenkt,
  koennen Nachrichten so verloren gehen.
* Die \_\_ganze Nachricht\_\_ muss $n + 1$ mal serialisiert werden!!!!!!!

\subsubsection{Paketvermittlung} 

Bei der Nachrichtenvermittlung ist das groesste Problem, dass wir die *gesamte*
Nachricht bei jedem Knoten neu serialisieren muessen. Diesem Problem koennen wir
durch *Paketvermittlung* entgegenwirken. Hierbei werden Nachrichten jeweils in
kleine Stuecke, *Pakete* genannt, aufgeteilt. Jedes Paket erhaelt dann einen
eigenen Header der Laenge $L_H$ und kann einzeln, unabhaengig von den anderen an
den Empfaenger gesendet werdet. Dieser Header enthaelt dabei nicht nur
Adressierungsinformation, sondern auch Hinweise darauf, wie die vollstaendige
Nachricht aus den einzelnen Paketen wieder reassembliert werden kann (z.B. mit
einer Nachrichten-ID). Die einzelnen Pakete muessen dabei im Allgemeinen nicht
gleich gross sein, haben aber eine minimale Groesse $p_{min}$ und maximale
Groesse $p_{max}$.

Wir interessieren uns wieder fuer die Uebertragungszeit einer $L$-bittigen
Nachricht. In diesem Szenario gehen wir davon aus, dass alle Pakete dieselbe
Groesse $p_{max}$ haben, sodass die Nachricht also ein ganzzahliges Vielfaches
von $p_{max}$ ist. Das essentielle hierbei ist, dass sobald das erste Paket der
Nachricht nach $p_{max}/r$ Sekunden serialisiert wurde und der letzte Bit nach
$t_p$ Sekunden angekommen ist, dieses beim naechsten Knoten im Pfad auch
*sofort* weitergesendet werden kann. D.h. dass wenn eine Nachricht in 1000
Paketen gesendet wird, wir zwischen dem Quellknoten $s$ und dem ersten
Vermittlungsknoten $i$ lediglich eine Verzoegerung von einem Paket haben, bevor
$i$ weitersenden kann. Bei Nachrichtenvermittlung haetten wir naemlich bei
Knoten $i$ auch noch auf die anderen 999 Pakete warten muessen, bevor wir die
1000 Pakete (bzw. die Nachricht) weitersenden haetten koennen. Das waeren fuer
*je zwei Knoten* also 999 mal mehr Paketserialisierungszeiten. So haben wir
zwischen je zwei Knoten lediglich eine Paketserialisierungszeit, die wir warten
muessen. Danach werden Pakete naemlich auch laufend weiter gesendet. Waehrend
Knoten $i$ naemlich das erste Pakete fuer $i + 1$ serialisiert, kommt bei $i$
schon das naechste Paket von $s$ an. Das heisst dann, dass wir fuer $n$
Vermittlungsknoten $n$ mal die extra Paketserialisierungszeit fuer das erste
Paket zahlen muessen. Dazu dann noch $L/p_{max}$ mal diese Zeit, fuer die erste
Serialisierung beim Quellknoten sowie letztlich natuerlich noch die
Ausbreitungsverzoegerung $t_p$. Insgesamt also:

$$T_{PV} = n \cdot \frac{p_{max} + L_H}{r} +
(\left\rceil\frac{L}{p_{max}}\right\rceil \cdot L_H + L) + \frac{d}{\nu c}$$

Durch Paketvermittlung koennen wir also nun nicht nur Nachrichten bzw. den Kanal
multiplexen, sondern Pakete. Das hat den Vorteil, dass die Aufspaltung der Zeit
(TDM) noch flexibler sein kann und dass wenn ein Paket verloren geht, auch nur
dieses erneut gesendet werden muss. Veraendert sich beispielsweise ploetzlich
die Qualitaet eines Kanals, so kann fuer das naechste kleine Paket eine andere
Entscheidung getroffen werden. Man muss nicht die Uebertragung der *gesamten*
Nachricht ueber diesen nun moeglicherweise schlechteren oder ausgelasteten Kanal
abwarten. Auch ist die Pufferung kleiner Pakete einfacher als die Pufferung
ganzer Nachrichten.

Die Nachteile von Paketvermittlung gegenueber anderen Verfahren sind:

* Jedes Paket benoetigt seinen eigenen Header, was viel Overhead bringt.
* Der Empfaenger muss Pakete wieder reassemblieren, was natuerlich zusaetzlich
  Zeit und Aufwand kostet.

Paketvermittlung wird in den meisten modernen Datennetzen und insbesondere durch
das Internet Protokoll (IP) verwendet.

\subsection{Adressierung im Internet} 

Die Sicherungsschicht bietet uns Moeglichkeiten, Knoten direkt zu
adressieren. Das geht natuerlich nur, wenn Knoten auch direkt verbunden
sind. Wir wollen uns nun ansehen, wie Rechner global adressiert und organisiert
werden. Dazu beschaeftigen wir uns mit *IP Adressen* und zwar Versionen 4
und 6. IP steht dabei fuer *Internet Protocol*.

Es ist vielleicht zuerst nuetzlich, zu besprechen, wieso IP Adressen ueberhaupt
notwendig sind. Wir haben doch schon durch MAC Adressen die Moeglichkeit, jeden
Rechner der Welt eindeutig zu identifizieren. Wieso brauchen wir dann noch diese
redundanten IP Adressen?

Betrachten wir hierzu das folgende Beispielnetzwerk mit vier Hosts und drei Switches:

```

01:23:45:67:89:AB A           C 11:36:45:CF:89:AB
                   \   0 1   /
                   S1---S---S2
                   /         \
74:FA:BB:01:15:80 B           D DE:AD:BE:EF:AF:FE
```

Der Switch $S$ haette jetzt in seiner Switching-Table bei Port 0 Eintraege fuer
Hosts $A$ und $B$ und bei Port 1 Eintraege fuer Hosts $C$ und $D$. Was ist nun
aber, wenn wir das Netzerk auf Port 1 irgendwie identifizieren wollen? Die
MAC-Adressen selbst haben ueberhaupt keine (hierarchische) Struktur, die dies
erlauben wuerden. Auch ist "Port 1" von Switch $S$ sicher keine gute oder
skalierbare Adressierungsmoeglichkeit, da sie insbesondere nicht global
eindeutig ist (Switches haben auch keine MAC Adressen). Was wir moechten ist
eine global eindeutige Identifizierung der Hosts auf Port 1, sowie dann von
Hosts innerhalb dieses Sub-Netzwerks.

Genau dafuer gibt es nun eben IP-Adressen. Sie erlauben es naemlich, Netzwerke
global eindeutig zu identifizieren und dann auch Hosts innerhalb von diesen
eindeutigen Netzwerken. Sie haben also eine *hierarchische* Struktur, die uns
die Organisation von Rechnern weltweit viel einfacher macht. Auch sei angemerkt,
dass es sehr einfach ist, seine MAC Adresse zu verfaelschen, sodass sie nicht
mehr global eindeutig ist.

\subsubsection{IPv4} 

Bei IPv4 geben wir jedem Host auf der Welt eine eindeutige *IP-Adresse*. Eine
*IP-Adresse* ist dabei eine 32-Bit lange Zahl, die aus 4 Bytes (genauer:
*Oktetten*) besteht. Diese Oktette trennt man zur Veranschaulichung mit einem
Punkt, wobei man die Byte-Werte selbst in Dezimalschreibweise angibt. Das nennt
man dann *Dotted-Decimal* Notation.

Eine IP-Adresse besteht immer aus zwei Teilen:

1. Einem Netzwerkteil, der das Netzwerk des Hosts identifiziert. Dieser ist
   immer in den oberen Bits einer Adresse.
2. Einem Hostteil, der den Host selbst innerhalb des Netzwerkes identifiziert.

Ein Beispiel fuer eine IP-Adresse waere: `192.168.1.1`. Hierbei koennten
beispielsweise die ersten drei Oktette, also `192.168.1` das Netzwerk
identifizieren und das letzte Oktett, also `.1`, den (ersten) Host innerhalb des
Netzwerkes `192.168.1`.

Um nun eine Nachricht (bzw. ein Paket, in welches eine Nachricht aufgeteilt
wurde) an einen Host mit einer bestimmten IP-Adresse zu senden, muss man diese
Nachricht an diesen Host mit seiner IP-Adresse adressieren. Der Nachricht wird
also wiederum ein Header vorangestellt, also der *IP-Header*. Der IP-Header
sowie seine Payload, also die PDU der hoeheren Schicht (Schicht 4), bilden also
die Payload eines Schicht 2 Pakets, z.B. einem Ethernet Paket.

Wir wollen einen solchen IP-Header naeher untersuchen.

\paragraph{IP-Header} 

IP-Header haben eine minimale Laenge von *20 Byte*. Sie enthalten neben Quell- und
Ziel-IP-Adresse noch viele weitere Informationen:

1. Bits 0-3: *Version*. Da es neben IPv4 noch IPv6 gibt, gibt dieses Feld die
   Version des verwendeten IP (Internet Protocol) an.
2. Bits 4-7: *IHL* (*IP-Header-Length*). Dieses Feld gibt in 4 Bit die Laenge
   des IP-Headers in Vielfachen von *32 Bit bzw. 4 Byte* an (ueblicherweise
   5). Das ist notwendig, das IP-Header noch optionale Felder enthalten koennen,
   dessen Anzahl variabel ist. Ein IP-Header hat somit auch eine maximale
   Laenge, naemlich $15 \cdot 4 = 60$ Byte bzw. 480 Bit.
3. Bits 8-15: *TOS* (*Type-of-Service*). Dient zur Klassifizierung
   bzw. Priorisierung von Nachrichten. Kommt es beispielsweise bei einem Router
   zum Stau, werden Nachrichten mit hoeherer Prioritaet, z.B. zeitsensitiven
   Nachrichten (Sprachuebertagung), schneller durchgelassen.
4. Bits 16-31: Gibt die Laenge des IP-Pakets, also Header plus Daten, in Bytes
   an. Somit kann ein ganzes IP-Paket also maximal 65535 Byte gross sein. Manche
   Schicht 2 bzw. Schicht 1 Protokolle haben aber eine *Maximum Transmission
   Unit* (MTU), die geringer als diese 65535 Byte sind. Beispielsweise ist die
   MTU bei FastEthernet 1500 Byte. Ist das IP-Paket groesser als diese MTU,
   wuerde sie *fragmentiert* werden, also in kleinere Pakete aufgespalten.
5. Bits 32-47: *Identification*. Ein fuer jedes IP-Paket zufaellig gewaehlter 16
   Bit Wert. Dieses wird genutzt, um Pakete einer fragmentierten Nachricht beim
   Empfaenger wieder sammeln und reassemblieren zu koennen.
6. Bits 48-50: *Flags*. Diese drei Bits stehen fuer drei Flags, naemlich:
   1. Bit 48: Reserviert und momentan noch auf 0 gesetzt.
   2. Bit 49: *Don't Fragment* (*DF*) Bit. Ist diese Flag gesetzt, darf ein
      Paket nicht fragmentiert werden. Ist das Paket dann groesser als die MTU,
      wird es einfach verworfen.
   3. Bit 50: *More Fragments* (*MF*) Bit. Dieses Bit signalisiert, ob mehr
      Fragmente folgen (= 1) oder ob dies das letzte bzw. einzige (wenn nicht
      fragmentiert wurde) Fragment ist (= 0).
7. Bits 51-66: *Fragment Offset*. Dieser 16-Bit Wert gibt die absolute Position
   der Daten in diesem Fragment bezogen auf die *unfragmentierte Nachricht* in
   ganzzahligen Vielfachen von 8 Byte an. Diese Information, zusammen mit der
   Identification und MF-Bit, ermoeglicht dann die Reassemblierung von Paketen
   beim Empfaenger.
8. Bits 67-74: *TTL* (*Time-To-Live*). Diese Zahl gibt an, wie viele Hops ein
   Paket im Netzwerk maximal machen darf. Ein *Hop* ist dabei eine Kante im
   Netzwerkgraphen, also eine L2 Direktverbindung (Ethernet, WLAN). Erhaelt ein
   Router ein IP-Paket, dekrementiert er als Erstes den TTL um 1. Ist der TTL
   dann 0, so verwirft er das Paket und sendet eine *ICMP Time-Exceeded*
   Nachricht an den Absender. Bleibt es danach positiv, sendet er es weiter. Ein
   TTL von 0 wuerde also bedeutet, dass das Paket nur an den selben Host (sich
   selbst) gesendet werden darf. Ein TTL von 1 bedeutet, dass das Paket nur an
   Hosts im selben Direktnetzwerk (Subnetz) gesendet werden darf, also
   insbesondere nie an einen Router kommen sollte. Ein TTL ist naemlich gleich
   einem Hop, also keinem Router dazwischen (nur z.B. einem odere mehreren
   Switches). Durch die TTL wird sichergestellt, das Pakete nicht endlos durch
   das Internet kreisen koennen. Der Linux Kernel setzt diesen Wert auf 64.
9. Bits 75-82: *Protocol*. Identifiziert das Protokoll auf Schicht 4, das fuer
   die Daten in der Payload genutzt wird. Gueltige Werte sind beispielsweise TCP
   (0x06) und UDP (0x08).
10. Bits 83-98: *Header Checksum*. Eine einfache, schnelle Pruefsumme, nur fuer
    den IP-Header (also nicht den Daten). Die Pruefsumme ist so konzipiert, dass
    eine Dekrementierung des TTL, welche ein Router ja durchfuehren muss, einer
    einfachen Inkrementierung dieser Pruefsumme entspricht. Somit muss also
    nicht die ganze Pruefsumme neu berechnet werden. Wie bei CRC ist nur
    Fehler*erkennung*, nicht Fehler*korrektur* moeglich.
11. Bits 99-130: *Source Address*. Die IP-Adresse des Absenders.
12. Bits 131-162: *Destination Address*. Die IP-Adresse des Empfaengers.
13. Bits 163-...: *Options* oder *Padding*. Das IP unterstuetzt eine Reihe von
    Optionen, welche als optionale Felder an den IP-Header angefuegt werden
    koennen. Eine moegliche optionale Information waere zum Beispiel ein
    Zeitstempel. Nicht alle diese Optionen sind Vielfache von 4 Byte lang. Da
    ein Paket aber auf 4 Byte "aligned" sein muss (siehe IHL), muss
    gegebenenfalls gepadded werden.

Der 49. Bit eines IP-Header ist reserviert und auf 0 gesetzt. Reservierte Bits
muessen immer auf 0 standardisiert werden. Die Alternative waere naemlich zu
sagen, dass der Wert des Bits egal ist. Wenn man sich dann aber doch eine
Bedeutung fuer dieses Bit ausdenkt, dann koennte das zu Problemen fuehren, wenn
man vorher gesagt haette, dass der Bitwert "egal" ist. Manche Dienste haetten
diesen Bit dann z.B. zufaellig auf Eins gesetzt (oder was halt in Memory lag)
oder fuer eigene Zwecke genutzt. Kriegt der Bit dann ploetzlich eine Bedeutung,
ist es schlecht, wenn der Bit bei manchen eben schon (unwissentlich) gesetzt
wird.

\paragraph{Adressaufloesung} 

Stellen wir uns vor, wir wissen die IP-Adresse eines Hosts. Damit kommen wir
selbst noch nicht sehr weit. Liegt der Zielhost im selben Netzwerk brauchen wir
wohl seine MAC-Adresse. Liegt er in einem anderen Netzwerk, muessen wir unsere
Nachricht zu irgendjemandem senden, der unsere Nachricht zum Netzwerk des Ziels
vermittelt.

Je nachdem, ob das Ziel nun im selben oder einem anderen Netzwerk liegt, muessen
wir also anders handeln. Wie finden wir aber ueberhaupt raus, ob wir im selben
Netzwerk sind, wie der Zielhost? Dazu koennen wir eben die hierarchische
Organisation von IP-Adressen nutzen. Wir wissen immer, wieviele Bits der
IP-Adresse das Netzwerk identifizieren, und wieviele den Host (sofern
Subnetzmaske bekannt). Daher muessen wir also nur pruefen, ob der Netzwerkteil
der Ziel-IP-Adresse gleich dem Netzwerkteil unserer eigenen IP-Adresse ist. Je
nach Resultat, gehen wir anders vor.

\paragraph{Selbes Netzwerk} 

Im ersten Fall, wo Quell- und Zielhost also im selben Direktnetzwerk liegen,
muessen wir von der IP-Adresse ausgehend nun irgendwie die MAC-Adresse des Ziels
herausfinden. Dazu koennen wir das *Address Resolution Protocol* (*ARP*)
verwenden. Dabei *broadcastet* der Quellhost, der die MAC-Adresse seines Ziels
wissen moechte, einen *ARP-Request* an sein lokales Netzwerk ("Who has <ip>?
Tell <my ip>). Er sendet also einen passenden ARP-Request an die
MAC-Broadcast-Adresse `ff:ff:ff:ff:ff:ff`. Jeder Switch, der diese MAC-Adresse
im Ethernet Header sieht, leitet das Paket also an alle seine Ports weiter
(ausser jenem, von welchem der Request kam), sodass auch jeder Host im
(Direkt)Netzwerk am Ende diesen Request bekommt. Liest der Zielhost dieses Paket
und erkennt, dass nach seiner MAC-Adresse gefragt wird, so sendet er einen
*ARP-Reply* ("<ip> is at <mac>") als MAC-Unicast zurueck an den Quellhost (durch
den ARP-Request waere die MAC-Adresse der Quelle ja bekannt). Der Quellhost
erhaelt somit also die Ziel-MAC-Adresse und kann seine Nachricht danach in einem
neuen Ethernet-Paket *direkt* dorthin senden.

Der genaue Aufbau eines ARP-Pakets (welcher also in der Payload eines L2 Pakets
gesendet wuerde), ist wie folgt:

1. *Hardware Type*: z.B. Ethernet.
2. *Protocol Type*: z.B. IPv4.
3. *Hardware Address Length*: die Laenge der benutzten Hardware Adressen in
   Byte, z.B. 6 fuer MAC-Adressen.
4. *Protocol Address Length*: die Laenge dre benutzten Protokoll Adressen in
   Byte, z.B. 4 fuer IPv4.
5. *Operation*: Request (0x0001) oder Reply (0x0002).
6. *Sender Hardware Address [0:31]*: Obere 32 Bit der Sender Hardware (MAC) Adresse.
7. *Sender Hardware Address [32:47]*: Untere 16 Bit der Sender Hardware (MAC) Adresse.
8. *Sender Protocol Address [0:15]*: Obere 16 Bit der Sender IP-Adresse.
9. *Sender Protocol Address [16:31]*: Untere 16 Bit der Sender IP-Adresse.
10. *Target Hardware Address [0:15]*: Obere 16 Bit der *Target* Hardware
    Address.
11. *Target Hardware Address [16:48]*: Untere 32 Bit der *Target* Hardware
    Address.
12: *Target Protocoll Address*: Die vollstaendige IP-Adresse des Ziels.

Der Host, der die MAC-Adresse eines anderen Hosts wissen moechte, wuerde die
*Operation* im ARP-Header also auf *Request* (0x0001) stellen und alle Felder,
bis auf die Target Hardware Adresse (welche er ja wissen moechte),
ausfuellen. Der Request wird dann eben an alle Hosts gebroadcastet. Der
Ziel-Host sendet dann, nachdem er diesen Request erhalten hat, einen Reply
zurueck. Dabei vertauscht er die Inhalte fuer Sender und Target, und fuellt
zusaetzlich noch das vorhin leere Target (nach Vertauschen dann Sender) Hardware
Address Feld aus. Die Operation wird dann noch auf Reply (0x0002) gesetzt und
via Unicast (unterstes Bit des ersten Oktetts der Target MAC-Adresse im Ethernet
Header auf Null gestellt) an den Absender zurueckgeschickt.

Das sieht dann beispielsweise so aus:

Request: "Who has 192.168.1.2? Tell 192.168.1.1"
Reply: "192.168.1.2 is at \_\_04:0c:ce:e2:c8:2e\_\_".

So weiss der Quellhost nun also, wohin er sein Paket im lokalen Netzwerk senden
muss. Er wird diesen Eintrag dann natuerlich auch noch fuer einige Zeit (5 - 10
Minuten) cachen, um nicht fuer jedes Paket erneut einen ARP-Request machen zu
muessen. Diesen Cache nennt man passend *ARP-Cache*, welcher auf Linux mit dem
`arp -a` Befehl angezeigt werden kann. Mit diesem Befehl sieht man auch, dass
eigentlich nicht nur die MAC Adresse des Ziels gespeichert wird, sondern auch
der Name des Interfaces des Zielhosts, fuer welches diese IP Adresse steht. Auch
wird versucht, den Domain Name des Zielhosts zu speichern. Komplett waere ein
ARP-Cache Eintrag also:

\begin{verbatim}
[domain name] <ip> at <mac> [on <interface>]
\end{verbatim}

Der Zielhost kann seinen ARP-Reply auch als L2-Broadcast verschicken, sodass
alle Hosts in seiner Broadcast Domain den Reply erhalten. Abhaengig vom
Betriebssystem werden derartige unaufgeforderten ARP-Replies (*unsolicited ARP
replies*) oft auch im ARP-Cache gespeichert. Das machen beispielsweise Router
oft, sodass Hosts im Direktnetzwerk die MAC Adresse ihres Default Gateways
(insbesondere nach Aenderungen) bestenfalls immer im ARP-Cache haben.

\paragraph{Anderes Netzwerk} 

Wenn ein Host eine Nachricht an einen anderen Host im selben Netzwerk senden
will, und nur dessen IP-Adresse kennt, macht er also einen ARP-Request. Was ist
nun aber, wenn der Zielhost in einem anderen Netzwerk liegt? Dann kann er gar
keinen ARP-Request mehr machen, da eine Broadcastdomaene ja nur bis zum
naechsten Router geht.

Hierfuer hat jeder Host immer einen sogenannten *Default Gateway*. Das ist jener
Router, an welchen ein Host ein Paket sendet, wenn der Zielhost in einem anderen
Netzwerk liegt. Ein Host kennt dabei genauer gesagt die IP-Adresse des Default
Gateway Routers. Will ein Host also nun ein Paket an einen Host in einem anderen
Netzwerk senden, so muss der Quellhost das Paket an den Default Gateway
leiten. Kennt er dessen MAC-Adresse schon von einem vorherigen ARP-Request,
sodass die MAC-Adresse gecached wurde, kann er sein Paket direkt an den Router
senden. Ansonsten macht er mit der IP-Adresse des Routers eben vorher noch einen
ARP-Request.

Jedenfalls sendet der Quellhost dann ein Ethernet-Paket an den Router. Die
L2-Adresse (die Target MAC-Adresse im Ethernet Header) ist dabei also jene des
*Routers*. Die L3-Adresse ist aber die IP-Adresse des Target Host.

Wichtig ist hierbei, dass die MAC-Adresse, welche also nur zur Adressierung
innerhalb eines Direktverbindungsnetzes dient, beim Routing *veraendert*
wirdd. Der Router wird naemlich die Target Adresse des an ihn gerichteten
Ethernet-Headers durch eine neue ersetzen. Diese neue MAC-Adresse ist dann
entweder schon die MAC-Adresse des Zielhosts, wenn dessen Netzwerk an einem
anderen Interface des Routers haengt, oder die MAC-Adresse des Default-Gateways
des Routers selbst. Die letztere wuerde der Router eintragen, wenn der
Netzwerkteil der Target-IP-Adresse mit keinem der Netzwerke an seinen Interfaces
uebereinstimmt.

IP-Adressen, welche zur globalen End-zu-End Adresserierung zwischen mehreren
Direktverbindungsnetzen (Netzwerken), dienen, werden beim Routing aber
natuerlich \_\_nicht\_\_ veraendert (sofern NAT nicht beachtet wird).

\paragraph{ICMP} 

*ICMP* steht fuer *Internet Control Message Protocoll* und erlaubt den Austausch
von speziellen Signalnachrichten im Internet. Beispielsweise ist das notwendig,
wenn der TTL bei einem Router auf Null geht. Dann sendet der Router naemlich
eine ICMP *Time-Exceeded* Nachricht zurueck an den Absender. Andere moegliche
Signale waeren z.B.:

* *ping* (*echo request / echo reply*) um die Erreichbarkeit von Hosts zu
  ueberpruefen.
* *redirect*, um Pakete umzuleiten. Diese Nachricht wird von einem Router an
  einen Absender geschickt, wenn der Router meint, es gaebe einen besseren
  (effizienteren) Gateway als sich selbst.

Eine ICMP Nachricht wird dabei als Payload eines IP-Pakets gesendet. Es enthaelt
die folgenden Felder:

1. *Type*: Art der ICMP Nachricht (z.B. *echo request* oder *destination unreachable*)
2. *Code*: Subklassifizierung des Typs, z.B. wenn Type "destination unreachable"
   ist, dann sagt ein Code von 0, dass das Target-Netzwerk nicht erreichbar war,
   Code von 1, dass der Target-Host nicht erreichbar war etc.
3. *Checksum*: Eine Checksum fuer die ICMP Nachricht mit Payload
   (https://tools.ietf.org/html/rfc1071).
4. 32 Bit die vom Typ der ICMP Nachricht abhaengen.
5. Weitere variable Daten.

\paragraph{`ping`} 

Ein Beispiel von ICMP Nachrichten sind *Echo Request* und *Echo Reply*
Nachrichten. Hierbei kann ein Host einem anderen Host einen Echo Request senden
(ihn `ping`en). Erhaelt der Zielhost diese Nachricht, sendet er einen Echo
Reply zurueck. Kommt der Request nicht beim Host an, sendet irgendein Router
eine TTL Exceeded Nachricht zurueck.

Die Echo Request Nachricht wuerde dabei so aussehen:

1. Type: 0x08
2. Code: 0x00
3. Checksum
4. *Identifier*: Ein "grober" Identifier (z.B. fuer einen Stream von Requests;
   ein `traceroute`)
5. *Sequence Number*: Ein "feiner" Identifier (z.B. fuer einen konkreten Request
   in einem Stream; ein request in einem `traceroute`)

Linux Systeme geben zum Beispiel jedem `ping` Prozess einen eindeutigen
*Identifier*, und dann jedem Request ausgehend aus einem bestimmten `ping`
Prozess eine aufsteigende Sequence Number.

\paragraph{`traceroute`} 

Was man mit ICMP Echo Request/Reply Nachrichten zum Beispiel machen kann, ist
den `traceroute` Befehl zu implementieren. Dieser berichtet, welche Hops ein
Paket von Host $A$ zu Host $B$ macht. Er wird wie folgt implementiert:

1. Host $A$ will den Weg zu Host $B$ wissen.
2. Er sendet dabei Echo Request Nachrichten mit aufsteigender TTL. Initial ist
   dieser 1 (Hop).
3. Fuer jeden TTL Wert kommt das Paket maximal genau so viele Hops weiter. Wird
   der TTL bei einem Router Null, so sendet dieser Router dem Sender eine ICMP
   Time Exceeded Nachricht zurueck. Dann weiss der Sender also, dass nach so
   vielen Hops also dieser Router (mit dieser IP Adresse) kommt.
4. Dann erhoeht er die TTL fuer den naechsten Echo Request, sodass die Nachricht
   also einen Router weiter kommt. Das geht so lange, bis das Paket beim Ziel
   ankommt. Dann kommt eben der Echo Reply zurueck.

So kann Host $A$ also genau nachvollziehen, welchen Weg ein Paket zu $B$ gehen
wuerde.

\paragraph{IP-Zuteilung} 

Jeder Host wo gibt hat also eine global eindeutige IP-Adresse. Aber woher
bekommt er diese ueberhaupt? Es gibt zwei Moeglichkeiten:

1. Ein Netzwerkadministrator verteilt die Adressen statisch, per Hand (aus dem
   Pool von Adressen den ihn sein ISP fuer sein Subnetz gegeben hat)
2. Dynamisch, von einem *DHCP-Server*.

Vor allem die zweite Variante ist interessant. DHCP steht fuer *Dynamic Host
Configuration Protocol* und erlaubt es, in einem Netzwerk dynamisch IP-Adressen
zu verteilen. Moechte ein Host (Client) bei DHCP eine IP-Adresse haben, passiert
folgendes:

1. Er broadcastet einen *DHCP-Discover* Paket (L2-Broadcast), um einen
   DHCP-Server zu finden.
2. Ein (oder viele) DHCP-Server erhaelt diesen Rahmen und sendet einen
   *DHCP-Offer* mit einer moeglichen IP-Adresse zurueck an den Client.
3. Passt die Adresse dem Client, sendet er dem DHCP-Server wiederum einen
   *DHCP-Request* um zu signalisieren, dass er die Adresse haben moechte. Das
   ist so, weil es mehrere DHCP-Server in einem Netzwerk geben kann. Dann
   haetten auch mehrere Server den Discover Rahmen erhalten und auch mehrere
   einen Offer gesendet. Der Client wird aber nur einen Offer
   akzeptieren. Dafuer spezifiziert er in einem Teil des Request Rahmens einen
   ID fuer den Server, dessen Adresse er akzeptiert hat. Der Request, welcher
   dann eben wiederum gebroadcastet wird, kommt dann bei allen Servern an. Diese
   koennen dann pruefen, ob der ID im Request ihr eigener ist. Wenn ja, dann hat
   der Client also den Offer von diesem Server akzeptiert. Wenn nein, dann war
   es wohl ein anderer Server. Server, dessen Offer nicht akzeptiert wurden,
   geben ihre angebotenen Adressen dann eben wieder in ihren Addresspool.
4. Der DHCP-Server antwortet dann mit einem *DHCP-ACK*, um zu signalisieren,
   dass er die angeforderte IP-Adresse fuer den Client freigibt, oder einen
   *DHCP-NACK*, wenn er die Adresse (doch) nicht hergeben will (z.B. weil
   zwischendurch ein anderer Host die Adresse schon genommen hat).

Eine IP-Adresse, die ein Client von einem DHCP-Server erhaelt, wird auch *Lease*
bezeichnet. Ein Lease ist immer auf einen bestimmten Zeitraum beschraenkt,
welcher dem Client im DHCP-Offer bekanntgegben wird. Nach der Haelfte dieses
Zeitraums muss ein Client also vom (selben) DHCP-Server eine neue Adresse
anfordern. Dann kann der Lease entweder erneuert werden, oder ein neuer Lease
(eine neue IP Adresse) vergeben werden. Eine weitere interessante Eigenschaft
bei DHCP ist es, dass, besonders in kleinen Netzwerken, oftmals *Router* die
Aufgaben eines DHCP-Servers uebernehmen.

Oben wurde schon angemerkt, dass es auch mehrere DHCP-Server in einem Netzwerk
geben kann. Diese muessen dann aber entweder disjunkte Adressraeume verwalten,
oder, wenn nicht, ihren gemeinsamen Adressraum auf irgendwelche Art und Weise
synchronisieren ("Ich habe gerade diese Adresse vergeben; tu sie auch aus deinem
Adresspool").

Ein DHCP-Paket (L3) hat immer dieselbe Struktur. Interessant dabei sind drei
Felder:

* *Client-IP-Address*: Falls der Client noch keine IP-Adresse hat, dann sind
  hier nur Nullen. Wenn er schon eine hat, kann er sie hier reinschreiben,
  z.B. fuer Erneuerung (er wuerde fuer ein Renewal einen Request zu seinem
  DHCP-Server direkt unicasten, nicht zuerst einen Discover broadcasten.)
* *Your-IP-Address*: Wenn das DHCP-Paket ein Offer oder ein ACK ist, kommt hier
  die vergebene IP-Adresse rein.
* *Server-IP-Adress*: Wird vom Server bei Offern und ACKs entsprechend
  ausgefuellt (damit der Client dann zu diesem Server fuer ein Renewal unicasten
  kann)

Neben der IP Adresse und der Subnetzmaske liefern DHCP Server dann noch oft
weitere Informationen, zum Beispiel:

* DNS-Resolver zur Aufloesung von Domain-Namen
* Hostname (FQDN)
* Statische Routen, insbesondere den Default-Gateway
* MTU im Direktnetzwerk
* NTP Server zur Zeitsynchronisation

\paragraph{Adressklassen} 

Wie schon besprochen sind bestimmte Anteile einer IP-Adresse immer fuer das
Netzwerk designiert, und andere fuer Hosts. Wo genau die Trennlinie zwischen
Netzwerk- und Hostteil ist, haengt von der *Adressklasse* der IP-Adresse ab. Es
gibt naemlich folgende Adressklassen:

\begin{tabular}{|l|l|l|l|l|}
Klasse & 1. Oktett & Anteile & \# Netze & \# Hosts \\\hline
   A    & 0xxxxxxx  & N.H.H.H &  128   &  $2^24$ \\
   B    & 10xxxxxx  & N.N.H.H &  $2^14$  &  $2^16$ \\
   C    & 110xxxxx  & N.N.N.H &  $2^21$  &  $2^8$  \\
   D    & 1110xxxx  &        Multicast          \\
   E    & 1111xxxx  &        reserviert         \\
\hline
\end{tabular}

Im Jahre 1981 war diese Aufteilung mit Sicherheit logisch. Zum Einen konnte sich
niemand vorstellen, dass man mit 32 Bit niemals auskommen wuerde. Zum Anderen
dachte man dass es deswegen OK waere, bestimmten grossen Organisationen wie
Apple, MIT oder AT\&T so gigantische Adressbereiche zu geben.

Heutzutage merkt man aber zwei Sachen:

1. 32 Bit reichen nicht aus. Der letzte IPv4 Adressblock wurde am 3.2.2011
   vergeben.
2. Die Aufteilung in solche Adressklassen ist generell vollkommen ineffizient.

Deswegen wurde 1998 auch IPv6 eingefuehrt, womit wir uns spaeter beschaeftigen
werden.

\paragraph{Subnetting} 

Man hat schnell gemerkt, dass Adressklassen (*Classful Routing*) nicht effizient
oder effektiv sind. Man brauchte einen neue Moeglichkeit, Netzwerke zu
identifizieren und zu strukturieren. Hierfuer hat man sich das Konzept von
*Subnetzen* (*Classless Routing*) ausgedacht.

Ein Subnetz ist im Prinzip einfach nur ein Netzwerk. So erhaelt nun jedes
Interface nicht nur eine IP-Adresse, sondern auch eine 32-Bit lange
*Subnetzmaske*. Eine Subnetzmaske ist eine logische Maske, welche eine Adresse
in einen Netzanteil und einen Hostanteil teilt (so wie man es vorher durch die
obersten Bits eines Oktetts beim classful routing gemacht haette). Eine logische
1 an einer Bitstelle sagt dabei aus, dass diese Bitstelle in einer IP-Adresse
zum Netzanteil gehoert, und eine 0 dass diese Stelle im Hostanteil steht. Diese
logischen Einsen muessen dabei aneinander grenzend (contiguous) sein (man kann
also Netzanteil und Hostanteil nicht mehrmals mischen).

Will man nun konkret von einer IP-Adresse den Netzteil haben, so fuehrt man
einfach eine logische AND-Verknuepfung zwischen der IP-Adresse und der
Subnetzmaske durch. Z.B.:

```
11000000.10101000.00000000.10110010 (192.168.0.127)
11111111.11111111.11111111.00000000 (255.255.255.0)
-----------------------------------
11000000.10101000.00000000.00000000 (192.168.0.0)
```

Man sieht also hier an der Subnetzmaske, dass dieses Netzwerk 24 Netzwerkbits
und 8 Hostbits hat. Diese aus der AND-Verknuepfung resultierende Adresse, welche
in den Host Bits nur Nullen hat, nennt man die *Netzadresse* dieses
Subnetzes. Sie identifiziert nie einen konkreten Host, sondern immer das ganze
Netzwerk an sich (symbolisch). Setzt man nun alle Host Bits auf 1, erhaelt man
die (L3) *Broadcastadresse* des Subnetzes. Fuer das obige Beispeilsubnetz waere
die Broadcastadresse:

```
11000000.10101000.00000000.11111111 (192.168.0.255)
```

Es ist hierbei wichtig anzumerken, dass diese Broadcast Adresse auf Schicht 3
gilt und insofern anders als die L2-Broadcast (MAC) Adresse ist. So koennen
beispielsweise Hosts in einem Subnetz in verschiedenen Direktnetzwerken liegen
(mit mehreren Routern dazwischen). Jeder Router unterbricht dabei eine
Broadcast-Domain, sodass L2-Broadcasts nur innerhalb einer solchen Domain
gesendet wuerden. Aber ein L3 Paket, das an die L3 Broadcast Adresse gerichtet
ist, wuerde jeden Host in einem IP-Subnetz addressieren, unabhaengig von ihrer
physischen Position und ihrer Zugehoerigkeit zu einem Direktnetzwerk.

Weiss man nun also, dass ein Subnetz $N$ Netzwerkbits und $H$ Hostbits hat, so
weiss man auch, dass man in diesem Netzwerk (maximal) $2^H - 2$ Hosts
adressieren kann. Die minus zwei sind wegen der Netz- und Broadcastadresse.

Man fasst eine Subnetzmaske im Uebrigen oftmals durch die *Praefix-Notation*
zusammen. Hierbei schreibt man nicht die volle Subnetzmaske aus, sondern notiert
einfach neben einer IP-Adresse, nach einem Schraegstrich, die Anzahl an
Netzwerkbits in der Subnetzmaske. Oben hatten wir beispielsweise 24
Netzwerkbits. Dann wuerden wir die IP-Adresse als *192.168.0.255/24* notieren.

\paragraph{Supernetting} 

Wir koennen also einen groesseren Adressraum durch Subnetting in kleinere,
logisch gruppierte Netzwerke aufteilen. Das gleiche koennen wir natuerlich auch
in die andere Richtung machen und mehrere, kleine Netzwerke zu einem groesseren
zusammenzufassen (und dann als solches addressieren). Dieses Prinzip nennt man
*Supernetting*. Man kann zwei Subnetze aber nur dann supernetten, wenn sie sich
*nur im letzen Netzwerkbit* ihrer IP Adresse unterscheiden. Dann kann man diese
beiden Subnetze mit $N$ Netzwerkbits naemlich eindeutig durch ihr
uebergeordnetes Netz mit $N - 1$ Netzwerk identifizieren.

Nehmen wir beispielsweise diese beiden Netzadressen:

```
11000000.10101000.00000010.00000000 / 24
11000000.10101000.00000011.00000000 / 24
```

Sie unterscheiden sich also nur im letzen Netzwerkbit (Subnetzmaske ist
255.255.255.0). Das bedeutet, wir koennen diese beiden Subnetze supernetten und
durch die folgende /23 Adresse eindeutig identifizieren:

```
11000000.10101000.00000010.00000000 / 23
```

Da der ISP uns die beiden Subnetzraeume gegeben hat, hat er uns ja im Prinzip
dieses Supernet gegeben. Also gehoert dieses /23 Supernet auch sicher uns. Falls
uns das andere /24 Netz nicht gehoert, duerfen wir das natuerlich nicht machen.

\paragraph{Besondere Adressbereiche} 

Es gibt einige Adressbereiche in IPv4, die eine besondere Bedeutung haben,
naemlich:

* 0.0.0.0/8: *Hosts in diesem Netzwerk* (die *unspezifizierte Adresse*)
  - Wird als Quell-Adresse bei DHCP verwendet, wenn der DHCP-Client noch keine
    IP-Adresse hat.
  - Bedeutet bei Serveranwendungen: "jede Adresse die auf meinem Host verfuegbar
    ist"
  - Adressen aus diesem Bereich werden nicht geroutet (nicht von Routern
    weitergeleitet).
* 127.0.0.0/8: *Loopback-Adressen*
  - Identifiziert den lokalen Rechner (\_\_localhost\_\_): 127.0.0.1 (127.0.0.0 ist
    die Netzadresse, keine Host-Adresse!)
  - Wird nicht geroutet.
  - Solche Pakete werden nie gesendet, sondern einfach \_\_gelooped\_\_
    (*loopback*). Sprich, es wird sozusagen vom "Send-Buffer" einfach in den
    "Receive-Buffer" am selben Rechner gelegt.
* 10.0.0.0/8, 172.16.0.0/12, 192.168.0.0/16: *Private Adressbereiche*
  - Adressen hieraus kann es viele, viele Male in privaten Netzwerken auf der
    Welt geben (vor allem wegen NAT).
  - Adressen innerhalb dieser Bereiche duerfen nur zwischen privaten Netzwerken
    geroutet werden.
  - Duerfen nicht in oeffentliche Netze geroutet werden (wenn der Router keinen
    Eintrag fuer das Netzwerk hat, darf er es nicht zu seinem Gateway into se
    WWW senden)
* 255.255.255.255/32: *Global Broadcastadresse*
  - Identifiziert alle Hosts im momentanen Netzwerk. Es ist sozusagen die
    Broadcastadresse von 0.0.0.0/8.
  - Wird von Linux automatisch auf die L2 (Ethernet) Broadcastadresse gesendet.
  - Wird nicht geroutet.

\subsubsection{IPv6} 

IPv4 wird auf Grund seiner Adressknappheit langsam aber sicher obsolet. Auch ist
es von seiner Header Struktur einfach nicht effizient genug. Deswegen wurde 1995
als Nachfolger von IPv4 ein neues Internet Protokoll, IPv6, vorgeschlagen. Es
wurde dann 1998 auch standardisiert und ist seitdem im Einstatz.

Neuerungen bei IPv6 gegenueber IPv4 sind:

* Vergroesserung des Addressraums von $2^{32}$ auf $2^{128}$ Adressen. Das
  entspricht ca. $6.67 \cdot 10^{24}$ Adressen pro Quadratmeter auf der Erde.
* Vereinfachung des Headerformats (ergibt effizientere Verarbeitung auf Routern)
* Aenderungen bei der IP-Fragmentierung
* Stateless Address Autoconfigruation (SLAAC) mittels ICMPv6
* ...

Wir werden nun die Eigenheiten von IPv6 detaillierter betrachten.

\paragraph{Adressformat} 

IPv6 Adressen sind 128 Bit lang, was 16 Bytes entspricht. In Dotted-Decimal
Notation, wie sie bei IPv4 benutzt wird, waeren also 16 Gruppen zur
Repraesentation noetig. Das ist natuerlich nicht praktisch. Deswegen notiert man
IPv6 Adressen zu je 16 Bit in Hexadezimal. Jede Gruppe von 2 Byte bzw. 4 Hex
Ziffern wird dabei durch einen Doppelpunkt von seinen Nachbarn separiert. Eine
Beispieladresse waere `2001:0db8:0000:0000:0001:0000:0000:0001`.

Solche IPv6 Adressen sind dabei in drei Teile aufgespalten:

1. Oktette 0 bis 5 (48 Bit; 3 Gruppen): Praefix
2. Oktette 6 bis 7 (16 Bit): Subnet Identifier
3. Oktette 8 bis 15 (64 Bit): Interface Identifier

Es gibt dabei noch einige Konventionen zur Notation. Wir betrachten dazu die
obige Beispieladresse.

1. Fuehrende Nullen in Bloecken (je 16 Bit) konnen immer weggelassen werden:
  ```
  2001:0db8:0000:0000:0001:0000:0000:0001 => wird zu =>
  2001:db8:0:0:1:0:0:1
  ```
2. *Hoechstens eine* Gruppe konsekutiver Bloecke von nur Nullen darf durch einen
  leeren Block abgekuerzt werden:
  ```
  2001:db8:0:0:1:0:0:1 => wird zu =>
  2001:db8::1:0:0:1
  ```
3. Gibt es mehrere Moeglichkeiten fuer (2), so waehlt man die laengste Sequenz
   von Nullbloecken. Wenn es mehrere gleich lange Bloecke gibt, so waehlt man
   den ersten. Wenn man naemlich mehr als eine solche Gruppe von Bloecken
   abkuerzt, weiss man nicht mehr, wie lange die einzelnen Gruppen waren.
   Oben waere also falsch gewesen:
   ```
   X 2001:db8:0:0:1::1
   X 2001:db8::1::1
   ```
4. Ein einzelner 0 Block darf *nicht* mit :: abgekuerzt werden:
   ```
   2001:db8:0:1:1:1:1:1 => darf nicht werden zu =>
   2001:db8::1:1:1:1:1
   ```

Die Loopbackadresse in IPv6 besteht aus nur Nullen und hinten einer Eins:

```
0000:0000:0000:0000:0000:0000:0000:0001/128
```

Man kann sie also wie folgt abkuerzen:

```
::1/128
```

Die ganze Adresse ist hierbei der Praefix.

\paragraph{Headerformat} 

Eine der Neuheiten bei IPv6 gegenueber IPv4 ist ein neues Headerformat. Es ist
viel einfacher als bei IPv4 und besteht aus:

1. Bits 0-3: *Version*. Gibt wie bei IPv4 Headern die IP Version an (dieses Feld
   ist das Einzige, welches bei IPv4 und IPv6 gleich ist,
   notwendigerweise). Gueltige Werte sind 4 oder 6.
2. Bits 4-11: *Traffic Class*. Aequivalent zum Type-Of-Service (TOS) Feld bei
   IPv4. Es priorisiert Pakete (*Quality-of-Service*; QoS)
3. Bits 12-31: *Flow Label*. Ein Identifikator fuer einen *Flow*. Das war
   urspruenglich gedacht, um Echtzeitanwendungen hoehere Prioritaet zu
   geben. Heute dient es dazu, Routern zu sagen, dass Pakete aus dem selben Flow
   gleich behandelt werden sollen. D.h. sie sollen ueber den selben Pfad gesendet
   werden.
4. Bits 21-36: *Payload Length*. Die Laenge der Payload des Pakets, in Bytes.
5. Bits 37-45: *Next Header*. Gibt den Typ des naechsten (inneren) Headers an,
   der am Ende des IPv6 Headers folgt. Das waere bei L4-Payloads zum Beispiel
   TCP oder UDP, oder aber auch ICMPv6.
6. Bits 46-53: *Hop Limit*. Das Selbe wie TTL bei IPv4.
7. Bits 54-181: *Source Address*. Die 128 Bit lange IPv6 IP-Adresse des Senders.
8. Bits 182-309: *Destination Address*. Die 128 Bit lange IPv6 IP-Adresse des Empfaengers.

\paragraph{Extension Header} 

Was neben TCP/UDP oder ICMP noch im *Next Header* Feld eines IPv6 Headers stehen
kann, ist *Extension Header*. Eine IPv6 Extension Header ist ein Zusatz zum
urspruenglichen Header.

Es gibt verschiedene Arten von Extension Header, welche durch den konkreten Wert
im Next Header Field des IPv6 Headers bestimmt ist, z.B.:

* Fragment (44): Information bezueglich Fragmentierung von IPv6 Paketen.
* Hop-By-Hop (0): Informationen, die von jedem Hop auf einem Pfad inspiziert
  werden muessen.
* Destination Options (60): Informationen, die nur vom Target inspiziert werden
  sollen.

Deswegen haengt die eigentliche "Payload" eines Extension Headers vom konkreten
Typ ab. Abstrakt besteht ein Extension Header aber aus:

1. *Next Header*: Den Typ des naechsten (inneren) Header. Das kann selbst wieder
   ein Extension Header, oder aber der eigentliche L4 PDU Header sein.
2. *Header Extension Length*: Laenge dieses Headers. Dessen Sinn ist zweifaltig.
    Erstens, dient es natuerlich dazu, die Laenge der Extension Payload zu
    bestimmen, da diese variabel ist bzw. von der Art des Headers
    abhaengt. Zweitens dient es aber auch dazu, dass wenn ein Knoten nicht
    weiss, wie es mit einer bestimmten Art von Extension Header umgehen soll, es
    diesen auch ueberspringen kann. Waere nur der Extension Header Typ
    angegeben (im Next Header Feld des IPv6 Headers), muesste der Knoten wissen,
    wie lange so ein Extension Feld von diesem Typ ist.
3. *Daten*: Die Payload des Extension Headers, welche vom Typ der Extension
   abhaengt.

Ein Beispiel fuer einen Extension Header waere der, der fuer Fragmentierung
genutzt wird. Schicht-2 Protokolle wie Ethernet haben naemlich immer eine
MTU. Ueberschreitet die Groesse der Payload eines IP-Pakets diese MTU-Groesse,
muss das Paket in kleinere Pakete fragmentiert werden. Hierfuer gab es bei IPv4
den Fragment Offset und das More-Fragments Bit/Flag. Bei IPv6 gibt es hierfuer
einen eigenen Extension Header. Die Payload dieses Extension Headers besteht
aus:

1. Bits 16-28: *Fragment Offset*
2. Bits 29-30: Einem reservierten Bereich von 2 Bit.
3. Bit 31: Einem *More-Fragments* (*MF*) Bit.
4. Bit 32-63: *Identification*

Diese Teile sollten von IPv4 Fragmentierung noch bekannt sein. Es sei aber
angemerkt: \_\_Fragmentierung passiert bei IPv6 nur beim Sender\_\_! Der Sender muss
also wissen, was die MTU auf dem Weg von sich selbst bis zum Sender ist. Diese
Information kann ihm beispielsweise durch einen DHCP-Server (oder Router)
mitgeteilt werden.

\paragraph{Besondere Adressen} 

Wie bei IPv4 gibt es wieder bestimmte *spezielle* Adressen:

* ::1/128 -- *Loopback Adresse*
  - Adressiert den lokalen Rechner.
  - Werden an den selben Rechner "zurueckgeloopt" (verlassen ihn also nie).
  - Wird nicht geroutet.
  - Analog zu 127.0.0.1/8 bei IPv4.
  - Man bemerke dass die ganze Adresse der Praefix ist.
* ::/128 - *Nicht-spezifizierte Adresse*
  - Wird nicht geroutet.
  - Analog zu 0.0.0.0/8 bei IPv4.
* fe80::/10 - *Link-Local Adressen*
  - Jedes IPv6 Interface hat eine Link-Local Adresse.
  - Sie wird aus dem Interface-Identifier (MAC Adresse) generiert.
  - Hat nur innerhalb des lokalen Links (Direktnetzwerk) Gueltigkeit.
  - Kann es in verschiedenen Netzwerken mehrmals geben (nicht global eindeutig)
  - Wird daher nicht geroutet.
  - Aehnlich zu privaten Adressen bei IPv4 (aber einzeln vergeben).
* fc00:/7 - *Unique-Local Unicast Adressen*
  - Global eindeutige Adressen, die nur privat geroutet werden duerfen.
  - Aehnlicher zu privaten Adressen bei IPv4 (das sind hier wirklich ganze Bloecke).
* ff00:/8 *Multicast-Adressen*
  - Adressen, die eine bestimmte Gruppe von Hosts adressieren.
  - Werden geroutet.

\paragraph{Multicast} 

Ueberfliegen wir noch einmal, welche Arten von Adressierung es gibt:

1. Unicast
   * Pakete (Rahmen), die an ein *einzelnes Ziel* adressiert sind.
   * Alle anderen Knoten im Netzwerk verwerfen solche Rahmen bzw. leiten sie
     lediglich zum Ziel (Router).
2. Broadcast
   * Pakete, die an alle Stationen im Netzwerk adressiert sind.
   * Hierfuer gibt es spezielle Broadcastadressen.
   * Auf Schicht 3 sind diese meist auf das lokale Netzsegment (Direktnetzwerk)
     beschraenkt.
3. Multicast
   * Pakete, die an eine *bestimmte Gruppe von Knoten* adressiert sind.
   * Passiert mittels spezieller Multicast-Adressen (mit ff00 Praefix).
   * Auf Schicht 2 werden Multicasts haeufig wie Broadcasts behandelt.
   * Auf Schicht 3 gibt es spezielle Protokolle, die Multicast-Adressierung auch
     ueber das lokale Netzsegment hinaus ermoeglichen.
4. Anycast
   * Pakete, die an eine beliebige Station einer bestimmten Gruppe adressiert
     sind.
   * Zum Beispiel alle DNS oder alle DHCP Server.

Wie entscheidet ein Host aber, ob ein Multicast Paket an ihn adressiert ist? Die
Antwort ist, dass es bestimmte, vordefinierte *Multicast Grupppen* gibt. Jeder
Host ist immer in zumindest einer, aber moeglicherweise auch vielen Multicast
Gruppen. Erhaelt ein Host ein IPv6 Multicast Paket, muss er nur pruefen, ob die
Multicast Gruppe, an welche das Paket adressiert ist (die IPv6 Adresse des
Pakets) eine der Gruppen ist, in welcher sich der Host befindet. Ist der Host
nicht in dieser Gruppe, braucht er es auch gar nicht weiter zu bearbeiten
(z.B. auf L3). Die folgenden Adressen sind Beispiele fuer Multicast-Gruppen:

* ff02::1 - *All Nodes*
  - Adressiert alle Knoten auf dem lokalen Link.
  - Man bemerke: Es gibt in IPv6 keine explizite Broadcast Adresse
    (\_\_insbesondere ist die hoechste Adresse in einem Subnetz eine valide IPv6
    Adresse!\_\_), sondern nur diese spezielle Art von Multicast.
  - Jeder Host, der IPv6 spricht, ist in dieser Gruppe. Das ist also die oben
    angesprochene Gruppe, in welcher mit Sicherheit jeder Host ist.
* ff02::2 - *All Routers*
  - Adressiert alle Router auf dem lokalen Link.
* ff02::1:2 - *All DHCP-Agents*
  - Adressiert alle DHCP-Server auf dem lokalen Link.
* ff02::1:ff00:0/104 - *Solicited-Node Adress*
  - Wird im Neighbor Discovery Protocol (NDP) verwendet, welches zur
    Adressaufloesung (wie ARP) verwendet wird.
  - Wird aus einer normalen IPv6 Adresse gebildet, indem man die unteren 24 Bit
    (6 Ziffern / 3 Byte) auf den Prefix ff02::1:ff00:0 addiert.

Wie man sieht, fangen diese bestimmten Multicast Adressen alle mit `ff02`
an. Ist das Zufall? Nein. Wie oben beschrieben haben alle IPv6 Multicast
Adressen den Prefix `ff` (bzw. `ff00::/8`). Was konkret danach kommt, bestimmmt
den *Scope* des Multicast. Das Scope kann dabei sein:

* `ff02::`: *Link-Local*: Adressen im selben Direktnetzwerk (LAN).
* `ff05::`: *Site-Local*: Adressen, z.B. im selben Gebaeude (wobei es mehrere
  Links bzw. Subnetze in diesem Gebaeude gaebe)
* `ff08::`: *Global*

Die obigen Multicast Adressen sind also alle *link-local*.

Es gibt nun des Weiteren noch spezielle Protokolle fuer Multicast. Bei IPv4
hiess dies *Internet Group Management Protocol* (*IGMP*) und bei IPv6 heisst es
nun *Multicast Listener Discovery* (*MLD*). Mit diesen Protokollen arbeiten
Router, um Multicasts ueber Netzwerke hinweg routen zu koennen. Bei MLD haelt
jeder Router dabei eine Tabelle von Hosts auf seinem Interface (Netzwerk), und
den jeweiligen Multicast Gruppen, in welchen der Host ist. Der Router kann dann
eingehende Multicast Nachrichten entsprechend filtern. Wird ein Host nun Teil
einer neuen Multicast Gruppe, z.B. wenn er zum ersten Mal von einem DHCP-Server
eine IPv6 Adresse erhaelt und so automatisch Teil der *all nodes* Multicast
Gruppe wird, so informiert er den Router dabei durch einen *MLD Report*
(`exclude Host A from the filter for this multicast group`). Das selbe tut er
auch, wenn er nicht mehr Teil der Gruppe wird. MLD ist uebrigens Teil von
ICMPv6.

Zusaetzlich dazu gibt es dann noch das Konzept des *MLD Snooping* (oder IGMP
Snooping). Hierbei hoeren Switches im Netzwerk die MLD Nachrichten ab, um zu
erfahren, auf welchen Ports bestimmte Multicast Gruppenmitglieder present oder
nicht sind. Ein Switch muss dann Multicast Pakete nur an jene Ports senden, wo
der Switch weiss, dass es dort einen Host in dieser Gruppe gibt. Fuer *all
nodes* waere das natuerlich jeder Port. Da bei MLD ueber MLD Reports immer
gemeldet wird, wenn ein Host Teil einer neuen Gruppe wird, kann der Switch aber
auch fuer andere Gruppen immer genau wissen, ob auf einem Port Mitglieder dieser
Gruppe sind, oder nicht.

\paragraph{Solicited Node Address} 

Es seien noch einige Details zu *Solicited Node Adressen* (*SNA*)
angemerkt. Zuerst wollen wir uns ansehen, wann und wie eine solche Adresse
generiert wird. Grundsaetzlich erhaelt eine Host fuer jegliche IPv6 Adresse,
egal ob link-local oder global, immer eine entsprechende Solicited Node
Address. Geben wir einem Host beispielsweise die globale IPv6 Adresse
`2001:0db8:1ee7:2ea2:0921:2e11:d2c6:938b`, so erhaelt er automatisch auch eine
Solicited Node Adresse. Diese wird erzeugt, inde man die letzten drei Byte (24
Bit; 6 Ziffern) der urspruenglichen IP Adresse an den 104-Bit Praefix
`ff02::1:ff00:0` hinten anhaengt. Also:

\begin{verbatim}
2001:0db8:1ee7:2ea2:0921:2e11:d2c6:938b
                                ^^^^^^^     Relevant
ff02:0000:0000:0000:0000:0001:ff00:0000  (ff02::1:ff00:0)
---------------------------------------
ff02:0000:0000:0000:0000:0001:ffc6:938b
=>
ff02::1:ffc6:938b (Solicited Node Address)
\end{verbatim}

Wir haben also wie man hier sieht, `ff02::1:ffc6:938b` als Solicited Node Adress
erhalten. Was bedeutet das aber, genau? Bzw. was macht man mit dieser Adresse?

Das wichtige hier ist, dass diese SNA keine Unicast, sondern eine Multicast
Adresse ist. Sie identifiziert also eine *Multicast* Gruppe (im link-local
`ff02::` scope). Sobald ein Host also eine IP-Adresse bekommt, erhaelt er auch
diese SNA und faellt in dies Multicast Gruppe.

Jetzt kommt das allerwichtigste: Bei Neighbor Discovery will ein Host nun also
die L2-Adresse zur IP Adresse `2001:0db8:1ee7:2ea2:0921:2e11:d2c6:938b`
finden. Bei IPv4 haetten wir jetzt einen ARP Request an *alle Hosts im Netzwerk
gebroadcastet*. Das heisst, dass jeder Rechner, jeder Router, jeder Switch,
jeder DHCP-Server etc. im lokalen Link diesen Request erhalten, entpacken und,
bis auf einen Rechner, dann verwerfen muss. Bei IPv6 muss der Quellhost sein
Neighbor Discovery Paket (genauer: Neighbor Soliciation; siehe unten) aber nur
an diese SNA Multicast Adresse senden, die genau fuer diesen Zweck konzipiert
wurde. Im einfachen Fall, wo Switches kein MLD Snooping machen, erhaelt dann
zwar jeder Host das Paket. Sie muessen aber einfach schauen, ob sie in dieser
Multicast Gruppe sind. Wenn nicht, brauchen sie das Paket nach L2/L3 aber nicht
weiter verarbeiten (also die Neighbor Soliciation nicht weiter betrachten). Das
spart viel Arbeit. Wenn die Switches dann durch MLD Sooping sogar noch wissen,
an welchen Ports welche Multicast Pakete erwartet werden, kommt der ganze
Traffic dann sogar gar nicht erst zu allen Hosts.

Es ist vielleicht noch nicht ganz klar, wieso mehrere Hosts in der selben SNA
Multicast Gruppe enthalten sein koennen. Das ist aber vor Allem bei link-local
Adressen denkbar, welche ja aus der MAC-Adresse gebildet werden. Es koennte
beispielsweise zwei Hosts nebeneinander in einem Subnetz geben, die folgende MAC
Adressen haben:

```
de:ad:be:ef:af:fe (A)
de:ad:00:ef:af:fe (B)
```

Diese wurden passend so gewaehlt, dass ihre letzten 24 Bit, also ihre Device
Identifier (NIC) gleich sind. Z.B. koennte (A) ein Computer von Apple sein und
(B) ein Router von Cisco, sodass ihre oberen 24 Bit (OUI) verschieden
waeren. Nun bilden wir via SLAAC ihre link-local IPv6 Adressen:

```
fe80::dcad:beff:efef:affe (A)
fe80::dcad:00ff:efef:affe (B)
```

Man bemerke hier:

1. Wir haben `fe:ef:` zwischen dem OUI und dem Device Identifier eingefuegt.
2. Wir haben das zweit-linkste Bit (global/local) im linksten Byte des OUI
   geflipped (`de` => `dc`)

Jetzt haben wir also unsere link-local Adressen. IPv6 erfordert nun die
Generierung von SNA Multicast Adressgruppen, damit wir ueber Neighbor Discovery
die Hosts an diesen IP-Adresse durch Multicasts erreichen koennen. Wir bilden
also die SNA Gruppen, indem wir die unteren 24 Bit der IPv6 Adressen an den
Praefix `ff02::1:ff00:0/104` knuepfen:

\begin{verbatim}
ff02::1:ff00:0             +
fe80::dcad:beff:efef:affe
                  ^^^^^^^
-------------------------
ff02::1:ffef:affe        (A)


ff02::1:ff00:0             +
fe80::dcad:beff:efef:affe
                  ^^^^^^^
-------------------------
ff02::1:ffef:affe        (B)
\end{verbatim}

Wie man sieht haben beide Hosts die selbe SNA Multicast Adresse fuer diesen
lokalen Link. Sendet ein Host also ein Neighbor Discovery Paket an diese SNA, um
die MAC Adresse zu einer der beiden link-local Adressen zu erhalten, werden
diese beiden Hosts es sicher bekommen und verarbeiten. Sie muessen dann also
auch als einzige Hosts (unter der Annahme dass es nicht noch mehr Hosts in
dieser SNA Gruppe gibt) das Neighbor Discovery Paket, also den Neighbor
Soliciation Request, inspizieren. Gibt es noch 10,000 weitere Hosts im lokalen
Link, sind das 9,998 Hosts weniger. In der Neighbor Soliciation steht dann die
genaue 128-Bit Target Adresse drin, welche natuerlich nur entweder (A) oder (B)
gehoeren wird. Nur einer der beiden Hosts wuerde im Weiteren dann einen Neighbor
Advertisement senden.

\paragraph{Multicast Abbildung} 

Um ein IPv6 Multicast Paket zu senden, muss auch auf Schicht 2 eine
entsprechende Multicast Adresse gebildet werden. Hierfuer gibt es einen
speziellen MAC-Adressraum, der nur fuer IPv6 Multicast reserviert wurde. Bei
diesem Adressraum sind die ersten zwei (der insgesamt sechs) Oktette
`33:33`. Wir erinnern uns, was das bedeutet:

1. Der unterste Bit des ersten Oktetts ist 1 $\rightarrow$ *Multicast* (und nicht
   Unicast)
2. Der zwei-unterste Bit des ersten Oktetts ist 1 $\rightarrow$ *locally
   administered* (nicht globally unique)

D.h. wenn ein Switch diese spezielle, locally administerd Multicastadresse
bekommt, wird er sofort wissen, dass es sich um einen Multicast handelt. Woher
aber weiss man aber, was die Multicastgruppe ist? Wir haben oben ja verschiedene
Multicastgruppen bei IPv6 eingefuehrt, z.B. *All Nodes* (`ff02::1`) oder *All
DHCP-Agents* (`ff02::2:1`). Fuer jede dieser IPv6 (Schicht 3) Adressen gibt es
auch eine passende MAC/Ethernet (Schicht 2) Adresse. Diese wird gebildet, indem
man zu dem `33:33` Praefix der IPv6 Multicast MAC-Adresse noch die unteren vier
Oktette (32 Bit) der IPv6 Multicastadresse hinzugibt.

Moechte man also einen Multicast an alle Router im lokalen Link senden, so
wuerde man diese an die `ff02::2` IPv6 Multicast Adresse senden wollen -- auf
Schicht 3. Auf Schicht 2 wuerde man die entsprechende MAC-Multicast Adresse dann
so bilden:

1. Man nimmt den `33:33` Prefix fuer IPv6 Multicast MAC Adressen.
2. Und gibt die unteren 32 Bit (4 Oktette; zwei Gruppen) der IPv6 Multicast
   Adresse hinzu.

Das ergibt dann also `33:33:0:0:0:2` fuer den All-Routers Multicast.

\paragraph{Hypothese} 

Wieso werden die unteren 32-Bit der IPv6 Adresse auf die MAC-Adresse abgebildet?
Dazu zwei Beobachtungen:

* Da wir hier von Level 2 Multicasts sprechen, sprechen wir auch nur von
  Link-Local Multicasts auf Level 3. Insofern kommt auch nur der `ff02` Praefix
  fuer die MAC-Multicast Adressen in Frage (heisst auch, dass wir ihn weglassen
  koennen).
* Wir erlauben dann eben bis zu $2^{32}$ verschiedene solche Link-Local
  Multicasts mit Praefix `ff02`. Deswegen die unteren vier Oktette der IPv6
  Multicast Adresse. Dann muessten die 80 Bit zwischen `ff02` und diesen unteren
  32 Bit also eben immer Null sein. Scheint auch so zu sein:
  https://en.wikipedia.org/wiki/Multicast\_address.

\paragraph{SLAAC} 

IPv6 hat ein Feature, welches ermoeglicht, dass sich Hosts eigene Link-Local
IPv6 Adressen aus ihrer MAC Adresse generieren koennen. Dieses Feature heisst
*stateless address auto-configuration* (*SLAAC*). Man nennt sie *stateless*,
weil diese Adressen nicht von einem DHCP-Server vergeben werden. Eine solche
durch SLAAC generierte Adresse sieht dabei wie folgt aus:

1. Der Praefix ist `fe80::/10`, also eben fuer Link-Local Adressen (2 Oktette).
2. Danach kommen 54 Oktette nur Nullen fuer den Subnet Identifier (brauchen wir
   hier nicht, da Link Local Adressen sowieso nur fuer das lokale Subnetz/Link
   gueltig sind).
3. Danach kommt die MAC Adresse ins Spiel:
   1. Zuerst kommen die 24 Bit des *OUI* (Organizationally Unique
      Identifier).
   2. Dann kommen die 16 Bit `ff:fe`, um die 48 Bit MAC Adresse auf die vollen
      64 Bit des IPv6 Interface Identifiers zu padden.
   3. Und letztlich die restlichen 24 Bit der MAC Adresse, also der Device
      Identifier (*NIC*).

Hierbei ist zu beachten, dass beim OUI Teil der zweite Bit von rechts, also die
locally administered / globally unique Flag, *invertiert* wird. Wenn bei der MAC
Adresse dieser Bit also Null ist (globally unique), dann muss fuer die
entsprechende link-local Adresse dieser Bit auf Eins gesetzt werden (um das
selbe zu meinen). Das hat den Sinn, dass man, wenn man Link-Local Adressen
manuell verteilt, diesen Bit nicht explizit setzen muss um einen locally
administered link-local Adresse zu signalisieren. Wuerde man sonst naemlich die
MAC Adresse aus einer (vermeintlich) mit SLAAC konfigurierten Link-Local Adresse
rauslesen, so wuerde das auf eine globally-unique MAC Adresse hinweisen, wobei
wir das aber gar nicht so wollten. Als Loesung muesste man also bei manuell
vergebenen IPv6 Adressen selbst das locally-administered Bit setzen, was die
Adressen weniger uebersichtlich machen wuerde, da dieses Bit ziemlich weit links
ist. Wie oft es aber vorkommt, dass man MAC Adressen aus Link-Local Adressen
rausliest, sei dahingestellt.

Es ist im Weiteren sogar auch moeglich, *globale* Adressen automatisch zu
konfigurieren. Hierbei muss das Geraet noch die Praefixe seines Subnetzes
wissen, welche er durch das Neighbor Discovery Protocol von Routern im Netzwerk
in Form von unsolicited Router Advertisements erfahren kann. Solche globalen
Adressen mit eingebetteter MAC Adresse sind aber nicht unbedingt sicher, da man
so einen Host immer verfolgen kann. Hierzu gibt es IPv6 Privacy Extensions.

\paragraph{ICMPv6} 

Es gibt auch bei IPv6 wieder ICMP Nachrichten. ICMPv6 Header sind aehnlich wie
jene bei IPv4 aufgebaut:

1. Bits 0-7: *Type*. Die Art der ICMP Nachricht.
2. Bits 8-15: *Code*. Praezisiert den Typ der Nachricht, z.B. ob es sich um
   einen Echo *Request* oder Echo *Reply* handelt, wenn der Typ *Echo*
   ist. Anmerkung: Bei ICMPv4 waren das noch zwei verschiedene *Typen*.
3. Bits 16-31: *Checksum* Eine einfache Checksum fuer das ganze ICMPv6 Paket,
   inklusive Payload.
4. Bits 32-: *Payload*. Eine Nachrichtenpayload variabler Groesse (keine
   bestimmte Padding Vorschrift). Sie wird durch den Type bestimmt.

Fuer die Checksum wird ein sogenannter *Pseudo-IPv6-Header* verwendet. Hierbei
wird die Checkusumme ueber einen IPv6-Header berechnet, welche aber nur drei
Felder enthaelt:

* Next Header.
* Source IPv6 Address.
* Destination IPv6 Address.

Diese Felder sind naemlich gerade jene, welche sich bei der Uebertragung durch
se Interwebs nicht verandern. Dadurch kann man zwar eine Checksumme auf dieser
hoeheren Schicht haben (neben der FCS auf Schicht 2 und der Kanalkodierung auf
Schicht 1), was nochmals hoehere Sicherheit bringt, da kein
Fehlererkennungsverfahren optimal ist. Man muss aber Router auch nicht unnoetig
belasten. Wuerde man in diesen Pseudo Header beispielsweise noch den Hop Limit
einfuegen, so muesste man natuerlich nach jedem Hop die Checksum neu berechnen.

\paragraph{NDP} 

Das *Neighbor Discovery Protocol* (NDP) ist Bestandteil von ICMPv6 und bietet
eine Reihe von Funktionalitaeten an, wofuer man bei IPv4 teilweise eigene
Protokolle brauchte (z.B. ARP). Genauer gesagt definiert NDP *fuenf* ICMPv6
Paket Typen, naemlich:

1. Neighbor Solicitation (Type 135): Wie ARP-Requests.
1. Neighbor Advertisement (Type 135): Wie ARP-Replies.
3. Router Solicitation (Type 133): Nachfrage nach einem Router.
4. Router Advertisement (Type 134): Ich bin ein Router, hallo.
5. Redirect (Type 137): Ein Router teilt dem Quellhost mit, dass es einen
   besseren Gateway fuer ein Paket gibt, als sich selbst.

Diese fuenf Nachrichtenarten werden bei NDP fuer verschieden Zwecke genutzt:

* *Adressaufloesung* (NDP ersetzt ARP): Durch Neighbor Solicitations und
  Advertisements.
* *Router Discovery*: Durch Router Solicitations und
  Advertisements.
* Fuer SLAAC
* *Neighbor Unreachability Detection* (NUD). Funktioniert auch ueber Neighbor
  Solicitations.
* *Duplicate Address Detection*: Ein Knoten kann ueberpruefen, ob seine IPv6
  Adresse im Link schon benutzt wird (wie mit ARP bei IPv4). Das ist
  z.B. notwendig, wenn man sie sich durch SLAAC selbst generiert hat.
* *Parameter Discovery*: Hosts koennen Parameter fuer einen Pfad finden,
  z.B. die MTU oder den Hop Limit eines Routers auf dem lokalen Link. Bei IPv6
  muessen Hosts ja selbst, bevor dem Senden, ihre Nachricht
  fragmentieren. Router machen das bei IPv6 nicht mehr. Wenn die MTU fuer das
  gewaehlte Interface eines Routers kleiner ist als ein eingehendes IPv6 Paket,
  verwirft dieser das Paket und sendet eine ICMPv6 *Packet Too Big* (Type 2)
  Nachricht zuruck an den Quellhost. Was ein Host bei IPv6 also macht, ist
  anfangs anzunehmen, dass die MTU fuer den gesamten Pfad gleich der MTU auf dem
  lokalen Link ist, welche es eben durch diese Parameter Discovery herausfinden
  kann (oder durch DHCP). Kommt das Paket dann zu einem Router auf diesem Pfad,
  dessen MTU kleiner ist als das Paket, sendet es eben eine solche *Packet Too
  Big*-Nachricht zurueck an den Host. In dieser ist dabei auch die MTU dieses
  Routers enthalten. Der Quellhost kann sein Paket dann also entsprechend dieser
  MTU fragmentieren und probiert dann die Fragmente nochmals wegzusenden. Das
  macht er so lange, bis die Fragmente klein genug sind.

Betrachten wir konkret die Nutzung von NDP fuer Adressaufloesung, was bei IPv4
noch mit ARP ging. Die Situation ist wie bei ARP die folgende: Ein Host kennt
die IP (v6) Adresse eines Hosts auf dem lokalen Link, benoetigt aber nun noch
seine MAC (Schicht 2) Adresse, um ihm eine Nachricht zu senden.

Als Erstes wuerde dieser Host, der die MAC Adresse seines *Neighbors* finden
will, eine *Neighbor Solicitation*. Diese Neighbor Soliciation (welche also eine
ICMPv6 Nachricht ist) besteht aus den folgenden Teilen:

1. Dem ICMPv6 Header, bestehen aus Type (135 fuer Solicitations) und Code (0)
   sowie Pseudo-Header Checksum.
2. 4 Byte nur Nullen, als Padding.
3. Die 128 Bit IPv6 Adresse des Interface, dessen MAC Adresse man finden
   moechte. Erst ueber diese Adresse weiss ein Host in der Solicited Node
   Multicast Adressgruppe, ob wirklich nach seiner MAC Adresse gefragt ist, oder
   jener eines anderen Hosts in der selben Multicast Gruppe. D.h. das ist
   wirklich die global eindeutige IPv6 Adresse des Hosts.
4. Einer NDP *Option*. NDP unterstuetzt naemlich auch wieder verschiedene
   Optionen, welche aufgebaut sind aus:
   1. *Type* (1 = Source Link Layer Address)
   2. *Length* (hier 1) in \_\_Vielfachen von 8 Byte\_\_ fuer \_\_die
      ganze Option\_\_. Das passt bei Solicications z.B. eben genau: einen Byte
      fuer den Type, einen Byte fuer diese Length und dann 6 Bytes fuer:
  3. *Source Link Address*: Die MAC-Adresse des Quellhosts (der dieses NDP Paket
     absendet).

Diese Solicitation wuerde dann also an alle Hosts in der Multicast Gruppe der
Solicited Node Adresse fuer die gewuenschte Adresse (die in der Solicitation
drin) gemulticastet werden. Diese Solicited Node Address Multicast Adresse wird
dann auch auf die entsprechende MAC Multicast Adresse mit `33:33` Praefix
abgebildet. Wenn Switches kein MLD Snooping machen und nicht wissen, an welchen
Ports welche Multicast-Gruppenmitglieder sind, artet das dann auch
gewissermassen zu einem Broadcast aus. Ein Host der so ein Solication IPv6 Paket
wuerde dann also:

1. Pruefen, ob er Mitglied der Solicited Node Multicast Gruppe ist, an welche
   das Paket adressiert ist. Wenn der Switch MLD Snooping macht, waeren das
   gerade alle Hosts die in dieser SNA Gruppe sicher drin sind. Ansonsten sind
   das alle Hosts im lokalen Link. Das kann hierbei im Uebrigen schon auf
   Schicht 2 ueber die Multicast-Adresse geschehen.
2. Wenn der Host nicht in der Multicast Gruppe ist, verwirft er das Paket.
3. Wenn er schon drin ist, entpackt er das IPv6 Paket um an die ICMPv6 Nachricht
   bzw. die Neighbor Soliciation in der Payload des Pakets zu kommen. Dann
   prueft er genau ob die gewuenschte 128 Bit Adresse ihm gehoert, oder nicht.

In Retour kommt vom passenden Host dann ein *Neighbor Advertisement*
Paket. Dieses Paket ist dann *Unicast* an den Quellhost. Es ist ein wenig anders
aufgebaut als die Soliciation:

1. ICMPv6 Header:
   1. Type (136 fuer Advertisement)
   2. Code (0)
   3. Pseudo-Header Checksum
2. Drei Flag-Bits:
   1. Router-Flag (R): gestetzt, wenn der antwortende Knoten ein Router ist.
   2. Solicited-Flag (S): gibt an, ob dieses Advertisement eine Antwort auf eine
      Solication ist, oder ob es sich um ein *unsolicited Advertisement*
      handelt. Das machen zum Beispiel Router, um ihre MAC Adressen
      unaufgefordert bekannt zu machen.
   3. Override-Flag (O): gesetzt, wenn das Advertisement einen Eintrag im Cache
      des Ziels aktualisieren soll. Z.B. bei unsolicited Advertisements von
      Routern sinnvoll.
3. Target Adress: selbe IPv6 Adresse (des Targets), die auch schon in der
   Solicitation geschickt wurde. Sie bleibt also gleich.
4. Die NDP Option fuer die gesuchte MAC Adresse:
   1. Type: 2 ( = Target Link Layer Adress)
   2. Length: 1
   3. Target Link Adress: Die gesuchte MAC Adresse.

\subsection{Routing} 

Wir haben nun in genuegender Laenge uns damit beschaeftigt, wie Pakete innerhalb
von Direktnetzwerken zirkuliert werden. Nun interessieren wir uns aber noch
dafuer, wie wir ein Paket auch an den Nordpol senden koennen, wo wir
hochstwahrscheinlich keinen Direktlink per Ethernet haben. Dafuer muessen wir
uns mit *Routing* von Nachrichten auseinandersetzen. Hier gibt es zwei
Varianten:

* Statisches Routing
* Dynamisches Routing

\subsubsection{Routing Tabellen} 

Bei Routing sprechen wir allgemein immer von *Routing Tabellen*. Wenn wir ein
Paket an einen Router senden, entscheidet er anhand dieser Tabellen, wohin er
das Paket denn *routen* soll. Bei statischem Routing legen wir diese Tabelle per
Hand fest. Bei dynamischem Routing werden wir ueber Protokolle sprechen, die das
Aufstellen und Aktualisieren von diesen Tabellen automatisch machen.

Eine Routing Tabelle hat standardmaessig folgende Spalten:

1. Die *Netzadresse* eines dem Router bekannten Netzwerkes.
2. Die *Laenge des Praefixes*, also des Netzanteils einer IP-Adresse in diesem
   Netzwerk. Bei IPv4 bestimmt das die Subnetzmaske.
3. Den *Next-Hop* vom Router aus in Richtung dieses Netzwerkes. Diesen Next-Hop
   nennt man auch *Gateway*.
4. Das *Hardware-Interface* des Routers, ueber welches man zum Gateway gelangt
   (z.B. `eth0`).
5. Eine Metrik fuer die *Kosten* bis zum Ziel.

Nehmen wir diese Tabelle als Beispiel:

|  Destination  | Praefix |    Gateway     | Costs | Interface |
|:--------------|:--------|:---------------|:------|:----------|
| 192.168.255.8 |    30   | 0.0.0.0        |   0   |    eth2   |
| 192.168.255.0 |    29   | 0.0.0.0        |   0   |    eth1   |
| 192.168.0.0   |    24   | 0.0.0.0        |   0   |    eth0   |
| 172.16.1.0    |    24   | 192.168.255.3  |   1   |    eth1   |
| 172.16.0.0    |    23   | 192.168.255.3  |   1   |    eth1   |
| 0.0.0.0       |    0    | 192.168.255.10 |   0   |    eth2   |

Wir koennen hier viele interessante Dinge beobachten:

1. Unter *Destination* haben wir immer die *Netzadresse* eines Netzwerkes.
2. Ist das Netzwerk direkt von diesem Router aus erreichbar, liegt es also an
   einem Interface des Routers, so ist der Gateway 0.0.0.0. Diese Adresse
   identifiziert das lokale Direktnetzwerk (local link). Das bedeutet, dass der
   Router ein Paket, dass in ein Netzwerk mit diesem eingetragenen Gateway gehen
   soll, direkt per L2 Unicast senden kann.
3. Liegt das Netzwerk nicht direkt am Router angeschlossen, so ist der Gateway
   ein anderer Router.

Und noch zwei weitere Eigenschaften, die uns zum Thema des *longest prefix
matching* fuehren und die eigentliche Vorgehensweise beim Routing erklaeren:

1. Die Adressen sind nach absteigender Praefix-Laenge sortiert.
2. Die letzte Netzwerkadresse in der Destination-Spalte ist 0.0.0.0/0.

\paragraph{Longest Prefix Matching} 

Ein Router erhaelt nun also ein Paket. Wie routet er es? Er geht so vor:

1. Er betrachtet die Eintraege in der Tabelle in absteigender Reihenfolge,
   laengste Praefixe also immr zuerst.
2. Er berechnet dann fuer die Zieladresse des Pakets die Netzadresse, indem er
   eine logische AND-Operation der Zieladresse mit der durch die Praefixlaenge
   angegebene Subnetzmaske durchfuehrt.
3. Die erste Netztadresse, die so genau matched, wird genommen. Das ist also die
   Adresse, mit dem *longest matching prefix*.
4. Das Paket wird dann an den Next Hop (Gateway) gesendet. Ist dieser 0.0.0.0,
   so sieht er in seinem ARP-Cache nach um die MAC-Adresse des Hosts zu finden
   (muss eventuell einen ARP-Request machen) und sendet es direkt ueber den
   lokalen Link. Der Interface Eintrag sagt dem Router, ueber welches seiner
   Interfaces er den Next Hop dann erreichen kann, also von wo er das Paket
   weitersenden soll. Ist der Gateway nicht diese Default-Adresse, sendet er es
   also an den Next Hop, welcher ebenso ein Router ist (und bei sich fuer seine
   Routing Tabelle dasselbe macht). Hierbei muss zum Weitersenden des IP Pakets
   an den naechsten Gateway natuerlich womoeglich auch ein ARP Request gemacht
   werden.
5. Passt keine der nicht-`0.0.0.0` Destination Netzwerkadressen, passt am Ende
   sicher die `0.0.0.0` Adresse. Dies hat naemlich Praefix Null und eine Verundung
   mit nur Nullen ergibt auch sicher diese Nur-Nullen Adresse. Der Next Hop fuer
   diese Default-Route wird dann auch *Default Gateway* genannt.

Betrachten wir ein Beispiel. Der Router mit der obigen Tabelle erhaelt nun ein
Paket, dass an die IPv4 Adresse `172.16.1.23` adressiert ist. Dann geht der
Router so vor:

1. Er versucht die Adresse gegen den ersten Eintrag in der Tabelle zu
   matchen. Dazu ver-UND-et er diese eingehende Adresse mit der Subnetzmaske,
   die 30 Einsen (Praefixlaenge) und 2 Nullen hat, also `255.255.255.252`.
   ```
   10101100.00010000.00000001.00010111 (172.16.1.23)
   11111111.11111111.11111111.11111100 (255.255.255.252)
   -----------------------------------
   10101100.00010000.00000001.00010100 (172.16.1.20)
   ```
   Diese Adresse passt nicht zur Destination Adresse in der Tabelle
   (`192.168.255.8`). Also gehen wir eins weiter.

2. Wir probieren es mit dem zweiten Eintrag in der Tabelle:
   ```
   10101100.00010000.00000001.00010111 (172.16.1.23)
   11111111.11111111.11111111.11111000 (255.255.255.248)
   -----------------------------------
   10101100.00010000.00000001.00010000 (172.16.1.16)
   ```
   Passt wieder nicht zu `192.168.255.0`.

3. Probieren es mit dem dritten Eintrag:
   ```
   10101100.00010000.00000001.00010111 (172.16.1.23)
   11111111.11111111.11111111.00000000 (255.255.255.0)
   -----------------------------------
   10101100.00010000.00000001.00000000 (172.16.1.0)
   ```
   Passt auch nicht zu `192.168.0.0`.

4. Wir probieren weiter, nun mit dem vierten Eintrag:
   ```
   10101100.00010000.00000001.00010111 (172.16.1.23)
   11111111.11111111.11111111.00000000 (255.255.255.0)
   -----------------------------------
   10101100.00010000.00000001.00000000 (172.16.1.0)
   ```
   Ein Match! Diese Netzadresse passt zu der Destination Adresse in Zeile 4.

Wir haben nun also die Zeile fuer das Destination Netzwerk erkannt. Jetzt leiten
wir das Paket also an den in dieser Zeile spezifizierten Gateway weiter. In
diesem Fall waere das der Router (eigentlich allgemein ein Host)
`192.168.255.3`. Auch wissen wir, dass wir diesen naechsten Router ueber
Interface `eth1` erreichen koennen.

Wuerden wir die Adresse `123.123.123.123 ` routen wollen, so waere der einzige
Match in der letzten Zeile:

```
1111011.1111011.1111011.1111011 (123.123.123.123)
0000000.0000000.0000000.0000000 (0.0.0.0)
-------------------------------
0000000.0000000.0000000.0000000 (0.0.0.0)
```

Matched also (natuerlich) mit der Netzadresse in der letzten Zeile. Dann wuerden
wir dieses Paket an unseren *Default Gateway* weitersenden. Das geht so lange,
bis ein Router fuer die Adresse gar keine Moeglichkeit mehr hat. Das ist
z.B. bei Routern ganz ganz weit oben in der Hierarchie des Internet, wo es gar
keine Default Gateways mehr gibt. Diesen Raum nennt man deswegen auch *Default
Free Zone*. Solche Router muessen also immer weiter wissen.

\subsubsection{Dynamisches Routing} 

Bisher haben wir noch nicht besprochen, wie Eintraege in unsere Routing Tabelle
ueberhaupt reinkommen. Die einfachste Variante waere natuerlich, sie manuell zu
konfigurieren. Das wird ab mehreren tausend Eintraegen sehr lustig. Viel eher
wollen wir, dass unsere Routing Tabellen *dynamisch* eingerichtet und
aktualisiert wreden. Hierzu gibt es verschiedene Protokolle, welche wir uns nun
ansehen wollen. Jedenfalls ist hierbei immer die Kommunikation von Routing
Tabellen und Informationen zwischen Routern ein integraler Bestandteil, sowie
auch die Methoden, durch welche kuerzeste Pfade gefunden werden.

\paragraph{Routing Protokolle} 

Es gibt zwei verschiedene, grundlegende Arten von Routing Protokollen:
Distanz-Vektor Protokolle und Link-State Protokolle. Diese haben folgende
Eigenschaften:

1. Distanz-Vektor Protokolle:
   * Hier kennen Router nur die Richtung (Vektor; der Gateway) und Kosten (Distanz)
     zu einem Ziel (Destination).
   * Router haben keine Information ueber die Netztopologie, sehen also nicht
     ueber ihre unmittelbaren Nachbarn hinaus.
   * Router tauschen untereinander lediglich \_\_akkumulierte Kosten\_\_ aus.
   * Lassen sich verteilt implementieren.
   * Finden kuerzeste Wege mittels dem Algorithmus von *Bellman-Ford*, welcher
     nur Informationen ueber die unmittelbaren Knoten des Nachbars benoetigt.
2. Link-State Protokolle:
   * Router informieren einander nicht ueber die akkumulierten Kosten, sondern
     nur ueber die Nachbarschaftsbeziehungen jeden Knotens. D.h., zu welchen
     Knoten sie verbunden sind und wie hoch die Kosten zu diesen Knoten sind.
   * Jeder Router erhaelt also von allen anderen Routern (Knoten) deren
     Nachbarschaftsbeziehungen (Connectivity Information). Somit kennt der
     Router also letztendlich die *gesamte Netztopologie*.
   * Anders gesagt: Jeder Router erfaehrt uber den *State* jeden *Links* im
     Netzwerk.
   * Aus dieser errechnet sich dann jeder Router selbst den kuerzesten Pfad
     zu einem Ziel durch das gesamte Netzwerk.
   * Das steht im Gegensatz zu Distanz-Vektor Protokollen, wo man davon ausgeht,
     dass der naechste Router schon den besten Pfad zum Ziel kennt.
   * Lassen sich nicht verteilt implementieren.
   * Verwendet den *Dijkstra Algorithmus* um kuerzeste Wege zu finden.

Der grundlegende Unterschied zwischen Distanz-Vektor und Link-State Protokollen
ist, dass Knoten bei Distanz-Vektor Protokollen davon ausgehen, dass
Nachbarknoten den besten Pfad zu einem Ziel schon kennen. Jeder Knoten hat also
sozusagen die akkumulierten Informationen ueber einen Pfad bis zu diesem
Knoten. Erhaelt ein Knoten von seinem Nachbarn seine akkumulierten Kosten zu
einem Ziel, vertrauen wir ihm, dass diese Pfade die bestmoeglichen sind und
inkrementieren den Hop Count einfach um einen Hop. Bei Link-State Protokollen
kann man sagen, dass Knoten einenander nicht so sehr vertrauen. Ein Knoten
vertraut seinem Nachbarn nicht, dass er wirklich den besten Pfad zu einem Ziel
kennt. Deswegen will er nicht die *akkumulierten Kosten* (Hop Counts) eines
Routers zu einem Ziel haben (Routing Table), sondern einfach nur dessen
Nachbarschaftsbeziehungen (Kanten und *direkte Kosten*). Kennt der Knoten
erstmal die gesamte Netztopologie, rechnet er sich *selbst* den kuerzesten Pfad
durch das *gesamte Netzwerk* zu jedem Ziel aus. Deswegen kann man Link-State
Protokolle auch nicht verteilt implementieren, da ja jeder Knoten das gesamte
Netzwerk kennen moechte (beim Dijkstra Algorithmus: deswegen hat jeder Knoten
einen Heap fuer das gesamte Netzwerk).

https://keepingitclassless.net/2011/10/link-state-vs-distance-vector-the-lowdown/

\paragraph{Bellman-Ford} 

Der Algorithmus von Bellman-Ford laesst sich wie folgt implementieren:

* Wir haben ein *Parent-Array* `parent`, welches immer den Vorgaenger eines Knoten im
  Graphen enthaelt.
* Wir speichern immer die Kosten von der Wurzel, also jenem Knoten, wo wir
  anfangen, in einem Kosten- bzw. Distanzarray `distance`.
* Ein vordefiniertes Array `costs` mit den Kosten `costs[i, j]` jeder Kante $(i, j)$.
* Wir haben also einen Wurzelknoten $s$.

Dann liest sich der Algorithmus in Pseudo-Code wie folgt:

\begin{verbatim}
#Initialisierung
for n in Nodes:
	parent[n] = -1
	distance[n] = 0 if n == s else infinity

T = {s} # Menge erreichbarer Knoten

while distance[n] has changed for some n in N:
	S = {} # Updated nodes
	for i in T:
		for j in adjacent_to(i):
			if distance[i] + costs[i, j] < distance[j]:
				distance[j] = distance[i] + costs[i, j]
				parent[j] = i
				S.put(j)
	T = S

\end{verbatim}

Praktisch gesehen machen wir im Prinzip Breadth-First-Search. Noch weitere
konkrete Implementierungsdetails:

* Der Graph sei intern durch eine Adjazenzliste dargestellt.
* Kanten seien durch Destination, Kosten und eindeutigem ID charakterisiert.
* Knoten seien durch Vorgaenger, akummulierte Distanz und eindeutigem ID
  charakterisiert.
* Wir markieren Kanten, nicht Knoten. Wir koennen Knoten naemlich mehrmals
  betrachten, um i
* Wir enqueuen dennoch Knoten.

\begin{verbatim}
struct Node {
	var distance,
	var parent,
	var id
}

struct Edge {
	var cost,
	var destination,
	var id
}

function bellman-ford(graph, root):
	visited = LookupTable()
	queue = Queue()
    queue.enqueue(root.id)

	while not queue.is_empty():
		node = queue.dequeue()
		for edge in graph.adjacent(node):
			if not visited.contains(edge.id):
				visited.put(edge.id)
				if node.distance + edge.cost < edge.destination.distance:
            		queue.enqueue(edge.destination.id)
					edge.destination.distance = node.distance + edge.cost
					edge.destination.parent = node
\end{verbatim}

Wichtig hierbei ist, dass jeder Knoten nur seine unmittelbaren Nachbarn zu
kennen braucht bzw. die Kosten zu ihnen.

\paragraph{Dijkstra} 

Beim Dijkstra Algorithmus, welcher fuer Link-State Protokolle verwendet wird,
haben wir den folgenden Setup:

* Wir haben wieder ein Parent Array `parent` und akkumulierte Distanz Array
  `distance` sowie Kantenkosten Array `costs`.
* Nun haben wir noch eine Priority Queue (Heap), in welchem immer der billigste
  Knoten im Netzwerk steht. Dieser soll hier eine Art Dictionary Semantik haben,
  sodass (Knoten, Kosten)-Paare gespeichert werden koennen. Dieses Paar soll
  durch den Knoten identifiziert sein, wobei der Heap nach den Kosten geordnet
  ist (MinHeap).

\begin{verbatim}
for n in Nodes:
	parent[n] = -1
	distance[n] = 0 if n == s else infinity
	queue.put((n, distance[n]))

T = {} # Menge der bearbeiteten Knote (visited)

while T != N:
	i = queue.popMin()
	T.put(i)
	for j in adjacent_to(i):
		if distance[i] + costs[i, j] < distance[j]:
			queue.decreaseKey(j, distance[i] + costs[i, j])
			parent[j] = i
\end{verbatim}

\paragraph{Vergleich} 

Wir wollen nun noch diese beiden Algorithmen bezueglich ihrer Eigenschaften
vergleichen:

1. Bellman-Ford:
   * Im $n$-ten Durchlauf betrachten wir gerade alle Pfade Laenge hoechstens
     $n$.
   * Keine komplexen Datenstrukturen notwendig.
   * Verteilte Implementierung moeglich.
   * Laufzeit in $O(|N| \cdot |E|)$
2. Dijkstra:
   * Es werden immer Pfade ueber den im jeweiligen Schritt am guenstisten Knoten
     gesucht.
   * Wurde ein Knoten abgearbeitet, so ist garantiert, dass der kuerzeste Pfad
     zu diesem Knoten gefunden ist.
   * Ressourcenintensiver als der Algorithmus von Bellman-Ford, da komplexe
     Datenstruktur (Priority Queue).
   * Wegen dem ersten Punkt muss jeder Knoten die vollstaendige Netztopologie
     kennen und die selbe grosse Priority Queue im Speicher halten. Laesst sich
     also schlechter verteilen.
   * Asymptotisch bessere Laufzeit: $O(|E| + |N| \cdot |N|)$

\paragraph{RIP} 

*RIP* steht fuer *Routing Information Protocol* und ist ein sehr einfaches,
altes Routing Protokoll, das bereits 1988 eingefuehrt wurde. Es dient dazu,
Routing Tabellen auszutauschen. Es hat nur ein einzige Metrik fuer Verbindungen:
den Hop Count (Anzahl an Routern zum Ziel). Das waere also so, als ob man allen
Kanten im Graphen die selben Kosten geben wuerde. Diese eine Metrik ist
natuerlich nicht sehr realistisch, da beispielsweise eine optische Leitung viel
schneller ist als eine alte Kupferleitung.

Die Funktionsweise ist bei RIP wie folgt:

1. Router senden in regelmaessigen Abstaenden (Standardwert: 30 Sekunden) den
   Inhalt ihrer Routingtabelle an die spezielle IPv4 Multicast Adresse
   `224.0.0.9`.
2. Alle Geraete in dieser Multicast-Gruppe (Router) akzeptieren das
   Update.
3. Sie inkrementieren alle Kosten in der erhaltenen Tabelle (um den Hop vom
   Quellrouter der Tabelle zu intergrieren). Gilt fuer einen Eintrag in dieser
   Tabelle dann, dass:
   * Es eine fuer diesen Router noch unbekannte Route ist, so wird sie sofort in
     die eigene Routingtabelle integriert.
   * Es eine Route ist, die der Router schon kennt, aber mit niedrigeren Kosten
     (nach dem Inkrementieren!), so uebernimmt er diese neue Route.
   * Es eine Route zu einem Ziel ist, wo der Router der seine Tabelle sendet
     gerade der Gateway ist, wird die Route entsprechend aktualisiert. Also wenn
     Router $A$ Router $B$ seine Tabelle sendet, mit einem Eintrag nach Router
     $C$ wo $A$ bei $B$ Gateway ist, dann aktualisiert $B$ seinen Eintrag nach
     den neuen Informationen von $A$, da die Informationen von $B$ fuer dieses
     Ziel $C$ ja nur relativ zu $A$ sind.
   * Andernfalls wird die vorhandene Route beibehalten und dieser Eintrag
     uebersprungen.
4. Bleiben *fuenf* aufeinanderfolgende Updates (also 2:30 Minuten bei 30
   Sekunden) von einem bestimmten Router aus, so gilt der Router als tot. Andere
   Router, die dies detektieren, schmeissen dann alle Eintreage mit diesem nun
   toten Router als Next Hop aus ihrer Tabelle raus.
5. RIP hat eine vordefiniertes Hop Count Limit von 15. Weiter entfernte Ziele
   sind ueber RIP also gar nicht erreichbar.

Betrachten wir wieder ein Beispiel. Wir nehmen hierfuer ein Netzwerk mit Routern
$A, B, C$. Anfangs sind ihre Tabellen leer, mit Ausnahme eines einzigen
Eintrags: der Route vom Router zu sich selbst.

```
    |Dst|GW|Cost|
    |---|--|----|
    | A |xx|  0 |
	      A
	    /   \
	   /     \
	  /       \
	 /         \
	/           \
   /             \
  B ------------- C
|Dst|GW|Cost|   |Dst|GW|Cost|
|---|--|----|   |---|--|----|
| B |xx|  0 |   | C |xx|  0 |
```

Dann sendet beispielsweise A zuerst seine Routing Tabelle an $B$ und
$C$. Dadurch erhalten $B$ und $C$ zum ersten Mal Routen fuer $A$:

```
    |Dst|GW|Cost|
    |---|--|----|
    | A |xx|  0 |
	      A
	    /   \
	   /     \
	  /       \
	 /         \
	/           \
   /             \
  B ------------- C
|Dst|GW|Cost|   |Dst|GW|Cost|
|---|--|----|   |---|--|----|
| B |xx|  0 |   | C |xx|  0 |
| A |xx|  1 |   | A |xx|  1 |
```

Der Gateway war in diesem Fall eben wieder das lokale Netzwerk, weil alle
Rechner hier direkt miteinander angeschlossen sind. Nun betrachten wir noch, was
passiert, wenn ein Router einen Eintrag nicht akzeptiert. Das waere der Fall,
wenn nun $B$ seinen Table multicastet. Dann wuerden $A$ und $B$ den Eintrag fuer
$B$ zum Ersten mal erhalten und folglich sofort integrieren. Aber: $C$ hat schon
einen Eintrag fuer $A$. Wenn $C$ nun also den Hop Count aller von $B$ erhaltenen
Eintraege inkrementiert und so eine Route fuer $A$ mit Kosten $2$ erhaelt,
obwohl $C$ eine Route zu $A$ mit Kosten $1$ kennt, dann wird $C$ diesen Entrag
nicht in seine Tabelle uebernehmen.

```
    |Dst|GW|Cost|
    |---|--|----|
    | A |xx|  0 |
	| B |xx|  1 |
	      A
	    /   \
	   /     \
	  /       \
	 /         \
	/           \
   /             \
  B ------------- C
|Dst|GW|Cost|   |Dst|GW|Cost|
|---|--|----|   |---|--|----|
| B |xx|  0 |   | C |xx|  0 |
| A |xx|  1 |   | A |xx|  1 |
                | B |xx|  1 |
```

\paragraph{Probleme} 

Wir wollen zwei Probleme betrachten, die es bei RIP bzw. Distanz-Vektor
Protokollen allgemein, geben kann.

\paragraph{Langsame Updates} 

Das erste Problem ist, dass es bei diesen rundenweisen, synchronen Updates oft
sehr lange dauern kann, bis Updates zu jedem Router durchsickern. Da das
maximale Hop Count Limit bei RIP 15 ist und standardweise alle 30 Sekunden ein
Update von einem Router weggesendet wird, kann es also bis zu $15 \cdot 30s =
7.5 min$ dauern, bis ein Update vom einen Ende des Pfades zum anderen kommt:


```
    30s      30s      30s      30s
(s) ---> (i) ---> ... ---> (j) ---> (t) 7,5 min!
```

Eine einfache Loesung hierfuer sind sogenannte *Triggered Updates*. Die Idee
dafuer ist einfach, dass ein Router neue Aenderungen an seiner Tabelle sofort
weitersagt. So fuehrt ein einziges Update von einem Router also zu einer "Welle"
von Updates im ganzen Netzwerk. Das hat den grossen Nachteil, dass das Netzwerk
in dieser Zeit sehr stark belastet sein kann (denkbar, wenn Routing Tabellen
mehrere hundert tausend Eintreage haben und es viele hundert Router in einem
Netzwerk geben kann).

Man spricht bei der Zeit, bis alle Knoten im Netz eine einheitliche Sicht auf
das Netzwerk haben, auch von der *Konvergenzzeit*. Sind also alle Router im
Netzwerk nach einer veraenderten Route "stabil", haben sie *Konvergenz*
erreicht. Bei RIP ist die Konvergenzzeit eben sehr hoch.

\paragraph{Count to Infinity} 

Betrachten wir fuer dieses Problem das folgende Netzwerk (mit eingezeichneten
Kantenkosten):

```
      (C)
     /   \
  1 /     \ 100
   /       \
  /         \
(A) ------- (B)
       1
```

Nun faellt der Link von $A$ nach $C$ aus:

```
      (C)
     /   \
  1 X     \ 100
   X       \
  /         \
(A) ------- (B)
       1
```

Host $A$ hatte vorher einen Eintrag nach $C$ in seiner Tabelle. Da es nach fuenf
fehlenden Updates merkt, dass der Host tot ist, loescht $A$ dann $C$ aus seiner
Routing Tabelle. $B$ hatte zuvor auch einen Eintrag fuer Host $C$ in seinem
Routing Table, naemlich mit einem Hop ueber Host $A$. Nun nehmen wir an, dass
$B$ seine Routing Tabelle nach dem RIP Protokoll multicastet (da die Reihenfolge
von Updates ja unspezifiziert ist). Es sendet dabei als Distanz-Vektor Protokoll
nur den Vektor, also die Adresse von seinen Eintraegen, sowie dessen Distanz
(Kosten). Hier sei der Fokus darauf, dass eben nicht der Gateway mitgesendet
wird. $A$ erhaelt nun also den (2-spaltigen) Table von $B$ und sieht den Eintrag
fuer den Pfad nach $C$. Da $A$ nach den fuenf fehlenden Updates keinen Eintrag
fuer $C$ mehr hat, nimmt es diesen Eintrag in seine Tabelle und inkrementiert
den Hop Count von 1 auf 2. Etwas spaeter sendet $A$ seine Tabelle nach $B$. Da
$B$ den Router $A$ als Gateway zu $C$ eingetragen hat, muss $B$ seine Route nach
$C$ entsprechend dem Eintrag von $A$ aktualisieren. D.h. $B$ inkrementiert
seinen Hop Count nach $C$ ueber $A$ um eins. Dann sendet $B$ seine Tabelle
wieder nach $A$, und $A$ sieht das sein Gateway nun einen Pfad der Laenge $2$ zu
$C$ hat, also muss $A$ seinen Hop Count nach $C$ ueber $B$ auf drei
inkrementieren, dann sendet $A$ seine Routing Tabelle zu $B$ ...

Es ist also ein Zyklus zwischen $A$ und $B$ entstanden. Die Hop Counts wuerden
so lange upgedatet, bis der Hop Count (nach RIP) das Maximum von $15$
ueberschreitet. Danach, mit einem Hop Count von $16$, gilt der Knoten sowohl von
$A$ als auch $B$ unerreichbar. Dieses Problem bei Distance-Vektor Protokollen
wird deswegen auch *Count to Infinity* genannt. Es sei aber angmerkt, dass
"Count to Infinity" nur ein fancy Name fuer einen Routing Loop ist. Das Problem
oben war ja, dass es eine Schleife zwischen $A$ und $B$ gab, weil sich diese
Knoten gegenseitig als Gateways eingetragen hatten.

Es gibt hierfuer drei Loesungen:

1. Split Horizon
   * "Sende dem Nachbarn, von dem du die Route zu $X$ gelernt hast, keine Route
     zu $X$"
   * "Sende deinem Gateway keine Routen ueber ihn selbst"
   * $B$ haette $A$ also nach dem Ausfallen von $C$ erst gar nicht eine Route zu
     $C$ geschickt, weil die Route von $B$ ueber $A$ ging.
   * Wenn es aber noch mehr Knoten im Netzwerk gibt, loest Split Horizon das
     Problem auch nicht. Wichtig hierbei ist insbesondere, dass es nicht
     spezifiziert ist, in welcher Reihenfolge Knoten ihre Tabellen
     austauschen. So koennten also dennoch ungueltige Routen verteilt werden.
2. Poison Reverse
   * "Sende dem Nachbarn, von dem du die Route zu $X$ gelernt hast, nur Routen
     zu $X$ mit unendlicher Metrik"
   * *Unendliche Metrik* bedeutet bei RIP also Metrik von 16.
   * Aehnlicher Ansatz wie Split Horizon, insbesondere da unendliche Metrik
     mehr oder minder dasselbe wie gar kein Eintrag ist.
   * Loest das Problem also insbesondere bei mehreren involvierten Knoten nicht.
3. Path Vector
   * Sende bei Updates nicht nur Ziel (Vektor) und Kosten (Distanz), sondern
     auch den vollstaendigen Pfad, ueber den das Ziel erreicht wird.
   * Ein Router prueft dann vor Installation einer Route in seiner Tabelle, ob
     er wohl nicht selbst in dieser Route drin ist.
   * Somit werden Zyklen im Graph also vermieden.

\paragraph{Echte Routing Protokolle} 

Wir haben schon RIP als ein echtes Routing Protokoll kennengelernt. Es wird aber
heute fast nirgends mehr benutzt, weil es schon sehr alt ist und viele
Schwaechen hat (z.B. Count to Infinity). Wir wollen uns noch andere Beispiele
fuer echte Protokolle ansehen:

* Distanz-Vektor Protokolle
  - *RIP* (*Routing Information Protocol*): Sehr einfaches Protokoll mit Hop
    Count als einzige Metrik. Geeignet fuer eine geringe Anzahl von Netzen.
  - *IGRP* (*Interior Gateway Routing Protocol*): Proprietaeres Routing
    Protokoll von Cisco, welches komplexere Metriken unterstuetzt als RIP.
  - *EIGRIP* (*Enhanced Interior Gateway Routing Protocol*): Verbesserte Form
    von IGRP, mit verbesserten Konvergenzeigenschaften.
* Link-State Protokolle
  - *OSPF* (*Open Shortest Path First*): Industriestandard fuer mittlere bis
    grosse Netzwerke.
  - *IS-IS* (*Intermediate System to Intermediate System*): Seltener
    eingesetztes, leistungsfaehiges Protokoll, welches unabhaengig von IP ist,
    da es sein eigenes L3-Protokoll mitbringt. Es wird insbesondere fuer grosse
    ISP Netzwerke genutzt. https://en.wikipedia.org/wiki/IS-IS

IGRP hat beispielsweise fuenf verschiedene Metriken:

* Hop count,
* Bandwidth,
* Delay,
* Load,
* Reliability.

Es benutzt einen maximalen Hop Count von 100 und multicastet Routing Tabellen
alle 90 Sekunden.

https://en.wikipedia.org/wiki/Interior\_Gateway\_Routing\_Protocol

\subsubsection{Autonome Systeme} 

Die oben vorgestellten Protokolle betrachten alle Pfade im Netz objektiv nach
ihren Metriken (Hop Count, Bandbreite, Delay etc.). Sie bieten aber keine
Moeglichkeit, Routen direkt zu beeinflussen. In Realitaet wollen wir aber
oftmals genau das machen. Uns interessieren dann zum Beispiel:

* Geldkosten, um ein Paket ueber eine bestimmte Leitung zu senden.
* Die Identitaet der Netze, ueber welche Datenverkehr (Traffic) weitergeleitet
  werden soll. Zum Beispiel in welchen Laendern diese Netze sind.
* Infrastrukturentscheidungen (z.B. Belastung einzelner Router)

Routingentscheidungen auf Basis solcher nicht-objektiver Kriterien bezeichnet
man auch als *Policy-Based Routing*.

Policy-Based Routing ist vor allem im Kontext von *Autonomen Systemen*
relevant. Ein Autonomes System (AS) bezeichnet ein Netzwerk oder eine Menge von
Netzwerken, die unter einheitlicher administrativer Kontrolle stehen (z.B. durch
die deutsche Telekom). Ein AS wird durch einen 16 Bit Identifier, der
sogenannten AS-Nummer, identifiziert. Dieser wird von der IANA vergeben.

Bei den oben angefuehrten Beispielen fuer Routing Protokollen wurde auch das
Interior Gateway Routing Protocol (IGRP) angesprochen. Weswegen heisst es
*Interior* GRP? Bzw. was ist das Gegenteil von Interior im Kontext von Routing?
Die Antwort ist:

* Innerhalb eines autonomen Systems werden *Interior Gateway Protocols* (IGPs)
  wie RIP, OSPF oder EIGRP eingesetzt.
* Zum Austausch von Routen zwischen autonomen Systemen wird ein *Exterior
  Gateway Protocol* (EGP) verwendet. Das einzige in der Praxis genutzte EGP ist
  das *Border Gateway Protocol* (BGP).

Autonome Systeme sind meist kommerzielle Gruppen, wie AT\&T oder
Vodafone. Hierbei gibt es dann noch drei Schichten, auch *Tiers*, genannt. Tier
3 AS sind die groessten, monolithischten Internet Service Provider (ISPs) wie
AT\&T. Auf Tier 2 sind kleinere AS (Comcast, Tele2) und auf Tier 3 wiederum.

Damit Knoten im einen AS mit Knoten in einem anderen AS kommunizieren koennen,
muessen AS irgendwie miteinander verbunden werden. Hierbei gibt es zwei
Moeglichkeiten:

1. *Peering Verbindungen*: Vereinbarungen zwischen AS, Traffic untereinander zum
   gegenseitigen Nutzen beider Systeme kostenlos auszutauschen. Solche
   Verbindungen sind meist horizontal.
2. *Customer-Provider* (*C2P*) Verbindungen: Kommerzielle Vertraege zwischen
   einem Customer-AS und einem Provider-AS, damit der Customer seinen Traffic in
   das Netz des Providers leiten kann und umgekehrt. Solche C2P Verbindungen
   werden meist zwischen niedrigeren Tiers als Customer und hoeheren Tiers als
   Provider geschlossen. Die Provider werden deswegen oft auch *Upstream
   Provider* genannt. Man kann sich diese Verbindungen meist vertikal
   vorstellen.

Peerings werden meist bei *Internet Exchange Points* (*IXP*s) eingerichtet. Hier
schliessen sich mehrere ISPs zusammen um Traffic untereinander kostenlos
austauschen zu koennen. Der groesste IXP der Welt ist der DE-CIX in Frankfurt.

Die Border-Router eines AS (welche also ein EGP wie BGP einsetzen wuerden)
announcen Netzadressen und Praefixe an seine Peerings und
Upstream-Provider. Wenn ein Tier 2 AS mit seinem Traffic nicht weiter weiss,
leitet er es an seinen Default Gateway weiter, welcher ein Router eines
Upstream-Providers sein wird. Kommt es in eine Tier 1 AS und kann auch dort
nicht verarbeitet werden, so wird das Paket verworfen. Tier 1 AS muessen
naemlich *immer* wissen, wo sie Traffic hinsenden koennen, da sie keine Default
Gateways mehr haben (welche bei einem niedrigeren Tier eben Router in hoeheren
Tiers waeren). Deswegen nennt man Tier 1 auch die *Default Free Zone*.

%# OSI Model: Layer 4

Die vierte Schicht des OSI Modells wird auch *Transportschicht* bzw. *Transport
Layer* genannt. Sie behandelt Multiplexing von Datenströmen zwischen zwei Knoten
im Netzwerk, sowie spezifiziert zwei Datenübertragungsprotokolle, *TCP* und
*UDP*. Weiters geschieht auf dieser Schicht auch *Network Address Translation*
(*NAT*). Das ist ein Mechanismus, wodurch private, zwischen Netzwerken
wiederverwendbare Adressen auf globale Adressen abgebildet werden.

### Multiplexing

Wir haben durch Schicht 1 (physikalische Schicht), 2 (Sicherungsschicht) und 3
(Vermittlungsschicht) nun die Möglichkeit, Daten zwischen Rechnern im Internet,
egal wo sie in der Welt physikalisch stehen, auszutauschen. Jedoch hat ein
Rechner meist nicht nur eine Anwendung laufen, sondern viele. Diese empfangen
und senden alle verschiedene Arten von Daten. Beispielsweise könnte der Rechner
einen Web-Server laufen lassen und muss also HTTP Requests und Responses senden
und empfangen können. Gleichzeitig hat er aber auch einen Email Dienst, welcher
mit SMTP Nachrichten arbeitet. Wenn nun also ein IPv4 oder IPv6 Paket bei
unserem Rechner ankommt, dann müssen wir ja irgendwie entscheiden, an welche
Anwendung wir ein bestimmtes Paket weiterleiten. Wir müssen die SMTP Nachrichten
an den Mail Client und die HTTP Requests and den Web-Server weitergeben. Der
springende Punkt ist also, dass wir einem Paket noch detailliertere
Informationen geben müssen, um die jeweilige Verbindung bzw. den konkreten
Kommunikationskanal für eine Anwendung eindeutig zu identifizieren.

Dieses Konzept nennt sich auch *Multiplexing*. Ein Rechner selbst bzw. der Kanal
zwischen Empfänger und Sender ist ja selbst neutral gegenüber den Daten, welche
auf ihm transferiert werden. Am Empfänger müssen die Daten auf dem Kanal dann
aber auf einen von vielen Kanälen für die jeweilige Anwendung weitergeleitet
weden. Gleichzeitig muss der Sender die Daten von seinen verschiedenen
Anwendungen alle auf denselben Kommunikationskanal übertragen. Der Sender wird
dadurch also zu einem *Multiplexer* und der Empfänger zu einem *Demultiplexer*.

Um Multiplexing zu realisieren, gibt es auf der Transportschicht nun das Konzept
von *Ports*. Ein Rechner mag zwar nur eine IP Adresse haben, kann dann aber
viele Ports haben. Jeder Port ist dann sozusagen der Eingangspunkt für eine
bestimmte Art von Paket, für eine bestimmte Anwendung. Man denke beispielsweise
an die Übertragung von Briefen zwischen zwei Apartmentblöcken. Es gibt nur eine
Straße (Kommunikationskanal) zwischen diesen Blöcken. Ebenso haben diese Blöcke
nur eine (IP) Adresse. Es würde jedoch nie ein ganzer Apartmentblock an den
ganzen jeweiligen anderen Block einen Brief senden. Viel eher würde ein Mensch
in einem Apartment im Block an einen anderen Menschen in einem Apartment im
anderen Block senden. Wir können die Briefübertragung also nicht nur durch ein
$(\text{Quelladresse}, \text{Zieladresse})$ Tupel darstellen, sondern bräuchten
zusätzlich noch die Apartmentnummern innerhalb der Blöcke, um den Kanal
eindeutig zu identifizieren. Analog dazu braucht man bei Netzwerken eben noch
Ports, um die genaue Verbindung über einen geteilten Kanal erkennbar zu machen.

Im Weiteren gibt es auf der Transportschicht nun noch verschiedene
Übertragungsprotokolle. Analog kann man einen Brief auch per Luft oder Land von
$A$ nach $B$ versenden. Neben Zieladresse und -port bzw. Quelladresse und -port
brauchen wir also noch Information über die Übertragungsart bzw. das
Protokoll. Somit wird eine Verbindung auf der Transportschicht letztendlich
identifiziert durch ein 5-Tupel aus:

$$(\text{Quell-IP, Quell-Port, Ziel-IP, Ziel-Port, Protokoll})$$

Während auf der Vermittlungsschicht Daten noch in Pakete aufgeteilt wurden,
sprechen wir bei der Aufsplittung von Daten auf der Transportschicht von
*Segmenten*. Jedes Segment erhält dann einen eigenen Header, wodurch also die
Protocol Data Unit (PDU) der Transportschicht entsteht. Jedes dieser Segmente
mit Header wird dann selbst wieder an einen IP Header konkateniert und in ein IP
Paket gewickelt, dann in einen Ethernet Rahmen und letztendlich verschickt. Der
Sinn hinter diesen verschiedenen Bezeichnungen fuer eine Nachricht, also
Segment, Paket und Rahmen ist im Uebrigen, dass ein Segment in viele IP Pakete,
ein IP Paket dann in viele Rahmen fragmentiert werden kann. Auf der
Anwendungsschicht haben wir dann noch *Datagramme* (z.B. eine E-Mail), welche
also selbst in viele Segmente aufgeteilt werden koennen.

Konkret ist ein Port meist ein *16 Bit* großer Wert. Das oben genannte 5-Tupel
ist auch genau das, was ein Socket im Kernel repräsentiert. Mit diesem kann dann
über einen File Deskriptor und den `read()` und `write()` Systemaufrufen Daten
versendet bzw. empfangen werden.

## Protokolle

Es gibt auf der Transportschicht zwei Protokolle: TCP, was
*verbindungsorientiert* ist und UDP, was *verbindungslos* ist. Wir werden zuerst
UDP bzw. allgemein verbindungslose Protokolle betrachten und dann TCP
bzw. allgemein verbindungsorientierte Protokolle. Jedenfalls bemerke man
unbedingt die Parallelität zwischen verbindungslosen und -orientierten
Protokollen auf der Transportschicht, und Paket- bzw. Leitungsvermittlung auf
der Vermittlungsschicht.

### Verbindungslose Übertragung

*Verbindungslose* Übertragungen sind so konzipiert, dass jedes versendete
Segment unabhängig von allen anderen ist. Aus Sicht der Transportschicht sind
die Pakete also sozusagen *zustandslos*. Daher muss zunächst mal der Header
solcher Segmente jeweils alle Informationen enthalten, die die Verbindung
identifizieren. Auf der Transportschicht sind das die Ports -- die Header von
Segmenten verbindungsloser Protokolle enthalten also zumindest mal Quell- und
Zielport.

Was wir dabei sogleich merken, ist dass wenn Segmente den Empfänger nicht
erreichen, sie einfach verloren gehen und grundsätzlich weder Empfänger noch
Sender etwas davon wissen. Man spricht hier also von *ungesicherter*
Kommunikation.

Da die Segmente auch alle unabhängig sind und so verschickt werden, können sie
also auch verschieden geroutet werden und insbesondere in beliebiger Reihenfolge
(out-of-order) beim Empfänger ankommen. Gibt es also eine logische Verbindung
zwischen den Daten der Segmente, so muss man sich überlegen, wie man die
Reihenfolge der Segmente erhält. Weil Segmente also eigentlich jeweils für sich
ganze Nachrichten darstellen, nennt man verbindungslose Protokolle
*nachrichtenorientiert*. Das Gegenteil davon wäre *Stromorientierung*, wofuer
TCP ein Beispiel ist.

#### UDP

Praktisch gesehen ist verbindungslose Übertragung heutzutage Synonym mit *UDP*,
dem *User Datagram Protocol*. Es ist eigentlich nur ein leichter "Wrapper" um
seine Payload. Es fügt dem IP-Paket einfach genau die minimal notwendige
Information der Transportschicht hinzu, um die Adressierung durch IP auf der
Vermittlungsschicht zu ergänzen. Die im Header enthaltenen Daten sind also genau
Quellport, Zielport, Länge von Header und Payload und letztlich noch eine
Checksum. Die Länge gibt dabei die Anzahl an Bytes von Header *und* Daten
an. Die Checksumme ist bei IPv4 optional, bei IPv6 dann aber doch
notwendig. Wird sie für IPv4 nicht berechnet, wird sie einfach mit Nullen
gefüllt. Wird sie schon berechnet, wird hierfür wie bei NDP Nachrichten ein
Pseudo-Header verwendet. Dieser Pseudo-Header berechnet sich aus dem UDP Segment
(Header + Payload) sowie folgenden Informationen aus dem IP Paket:

* IP Adresse der Quelle und des Ziels
* Ein 8-Bit Feld mit nur Nullen
* Die ID des verwendeten Protokolls, also immer UDP mit ID `0x11` (17)
* Die Länge von Header und Payload

Die Vorteile von UDP sind nun wie folgt zusammenzufassen:

* UDP Segmente haben geringen Overhead, da der Header klein ist (leichter
  Wrapper).
* Keine Verzögerung durch Verbindungsaufbau wie bei verbindungsorientierten
  Protokollen. Auch keine Verzögerung durch Retransmits.
* Gut geeignet für Anwendungen, wo es nicht so schlimm ist, wenn gelegentlich
  ein Paket verloren geht. Das sind vor allem Streaming Anwendungen wie VoIP
  oder Online-Spiele.
* Keine Beeinflussung der Daten durch Fluss- oder Staukontrolle (siehe unten)

Die Nachteile von UDP sind aber:

* Keine Zusicherung irgendeiner Form von Dienstqualität, da die Nachrichten
  *ungesichert* sind und die Fehlerrate somit beliebig hoch sein kann.
* Datagramme können ungeordnet (out-of-order) ankommen, beispielsweise wenn es
  mehr als einen Pfad zu einem Ziel gibt
* Keine Flusskontrolle: der Empfänger kann überrumpelt werden.
* Keine Staukontrolle: der Sender kann das Netzwerk überlasten, wodurch die
  Verlustrate steigt, wenn Pakete in Transit verloren gehen (z.B. wenn bei einem
  Router im Puffer kein Platz mehr ist).

### Verbindungsorientierte Übertragung

Bei *verbindungsloser* Übertragung war jedes Segment eine unabhängige
Nachricht. Bei *Verbindungsorientierter* Übertragung stellen wir zuerst einen
Kanal zwischen zwei Endpunkten (IP + Port) her. Dieser Kanal ist dann also die
dedizierte Verbindung zwischen Quell- und Zielknoten. Dieser Kanal muss dann
natürlich auch entsprechend verwaltet, also auf- und abgebaut, werden. Es gibt
also bei verbindungsorientierter Übertragung immer drei Schritte:

1. Verbindungsaufbau (*Handshake*)
2. Datenübertragung
3. Verbindungsaufbau (*Teardown*)

Während der Datenübertragung werden bei verbindungsorientierter Übertragung
einzelne Segmente, welche also nun einen Strom von Daten und nicht einzelne
Nachrichten darstellen, linear durchnumeriert. Jedes Segment erhält also eine
*Sequenznummer*. Diese erlaubt uns:

* Bestätigung erfolgreich übertragener Segmente,
* Dadurch dann die Identifikation von fehlenden Segmenten, sowie
* Erneutes Anfordern von fehlenden Segmenten und
* Zusammensetzen von Segmenten in der richtigen Reihenfolge bezüglich ihrer
  Sequenznummern

Während dem Handshake müssen sich Empfänger und Sender dann *synchronisieren*,
was dem Austausch von initialen Sequenznummern entspricht. Während der
Übertragung muss dann der *Zustand* der Verbindung gehalten werden. Mit
*Zustand* ist hierbei die aktuelle Sequenznummer sowie die bereits bestätigten
Sequenznummern gemeint.

#### Verbindungsaufbau und -abbau

Betrachten wir nun den Aufbau einer bidirektionalen Verbindung zwischen Knoten
$A$ und $B$. Knoten $A$ (o.B.d.A.) muss hierzu zuerst ein Synchronisationspaket
senden. Dieses beeinhaltet Informationen darüber, dass das Paket zum initialen
Verbindungsaufbau, also der Sychronisations, dient (`SYN` Flag). Auch wird die
initiale Sequenznummer $x$ von $A$ mitgeschickt. Dann kommt die entsprechende
Antwort auf die Synchronisation von Knoten $B$. Sie enthält wieder entsprechende
Flags fuer den Handshake, die auf die Natur des Pakets hinweisen. Dann enthält
es aber noch zwei weitere Daten:

1. Die initiale Sequenznummer $y$ der Segmente von Knoten $B$,
2. Die Bestätigung der initialen Sequenznummer $x$ von $A$, sodass also das
   Segment von $A$ mit Sequenznummer $x + 1$ angefordert wird.

Nach diesen beiden Segmenten kommt noch eine letzte Antwort von $A$ für dieses
zweite Paket, das von $B$ kam. Diese Antwort enthält einzig und alleine die
Bestätigung der Sequenznummer $y$ von $B$, sodass also $y + 1$ angefordert wird.

```
A           B
|---\___    | (1) SYN, SEQ = x
|       \---|
|   ____/---| (2) SYN, SEQ = y, ACK = x + 1
|--/        |
|---\___    |
|       \---| (1) ACK = y + 1
v           v
```

Diesen drei-stufigen Verbindungsaufbau (Handshake) nennt man auch *3-Way
Handshake*. Für den Verbindungsabbau wird dasselbe Verfahren verwendet, wobei
aber nicht Synchronisationspakete mit der gesetzen `SYN` Flag gesendet werden,
sondern *Finish*-Pakete mit dem `FIN` Flag.

#### Datenübertragung

Bei der Datenübertragung interessieren wir uns vor allem dafür, wann und auf
welche Weise Segmente *quittiert*, also bestätigt, werden. Die einfachste
Variante der Quittierung ist, dass einfach jede Nachricht einzeln quittiert wird
und der Sender erst auf die Bestätigung für ein Segment vom Empfänger wartet,
bevor er sein nächstes Segment wegschickt. Diese Methode ist zwar simpel, aber
ebenso ineffizient. Während nämlich ein Segment von $A$ nach $B$ propagiert und
$A$ dann auf die Bestaetigung von $B$ wartet, könnten weitere Segmente
nachgeschickt werden. Die *Round Trip Time* (RTT) wird also nicht gut
ausgenutzt, wodurch viel Bandbreite ungenutzt bleibt. Dieses Verfahren nennt man
auch __Stop and Wait__ (sende, stop and wait, ack, sende, ...).

Eine effizientere und flexiblere Steigerung dieses simplen Prinzips ist es, dem
Sender ein *Fenster* (also eine Anzahl) an Segmenten mitzuteilen. Der Sender
darf alle Segmente im Fenster gleichzeitg senden, ohne auf individuelle
Bestätigungen warten zu müssen. So kann die Übertragungszeit effizienter
(dichter) genutzt werden. Bevor der Sender aber in das nächste Fenster übergeht,
muss er zuerst alle Segmente im momentanen Fenster bestätigt bekommen
haben. Außerdem kann man durch Anpassung der Fenstergröße die Übertragungsrate
auf zwei Weisen steuern:

* Da der Empfänger die maximale Fenstergröße festlegt, kann er *Flusskontrolle*
  durchführen. Er kann also je nach seiner eigenen Überlastung die
  Übertragungsrate des Senders steuern.
* Merkt der Sender, dass das Netzwerk momentan überlastet ist, kann er durch
  Modifikation der Fenstergröße *Staukontrolle* machen.

Wir betrachten im Weiteren folgendes Modell von Schiebefensterprotokollen:

* Sender und Empfänger haben denselben *Sequenznummernraum* $\mathcal{S} = \{0, 1,
  ..., N - 1\}$ mit $N := |\mathcal{S}|$
* Der Sender hat ein *Sendefenster* (Send Window) $W_s \subset \mathcal{S}$ mit
  $w_s = |W_s|$
* Der Empfänger hat ein *Empfangsfenster* (Receive Window) $W_r \subset
  \mathcal{S}$ mit $w_r = |W_r|$
* Bestätigungen sind im Weiteren immer kumulativ. Das bedeutet, dass ein `ACK
  = x + 1` nicht mehr nur das Segment mit Sequenznummer $x$ bestätigt, sondern
  alle Sequenznummern $\leq x$.
* Der Sender darf sein Fenster immer dann um eine Sequenznummer
  weiterschieben, wenn er das erste Segment in seinem momentanen Fenster
  bestätigt bekommen hat.
* Der Empfänger darf sein Fenster immer dann weiterschieben, wenn er die erste
  Sequenznummer in seinem momentanen Fenster erhalten und dieses Segment
  bestätigt hat (unabhängig davon, ob die Bestätigung eigentlich ankommt.)

##### Segmentverlust

Wir interessieren uns nun dafür, wie bei Schiebefensterprotokollen mit
Segmentverlusten umgegangen wird. Hierfür gibt es zwei Verfahren: *Go-Back-N*
und *Selective Repeat*. Wir wollen sie näher untersuchen. Jedenfalls ergibt sich
durch den Einsatz von Schiebefenstern die interessante Situation, dass
Verwechslungen zwischen Fenstern auftreten können. Es ist hierbei vor allem
wichtig zu verstehen, dass sich die Fenster beider Seiten während der
Übertragung von Segmenten eines Fensters verschieben können (nach oben
angefuehrten Bedingungen).

Betrachten wir beispielsweise den Fall, dass sowohl Sender und Empfänger den
gesamten Sequenznummernaum $\mathcal{S}$ als Fenster wählen. Nun darf der Sender
also $N$ Segmente gleichzeitig senden. Erhält er eine Bestätigung für das erste
Segment, würde er sein Fenster um eine Position weiterschieben dürfen. Nun
erhält der Empfänger also alle $N$ Segmente, für welche er dann anschließend
Bestätigungen wegsendet. Da er die Segmente erhalten und bestätigt hat, darf der
Empfänger sein Fenster (um 360 Grad) drehen. Er nimmt nämlich an, dass seine
Bestätigungen alle ankommen, und dieselben Sequenznummern (da sich das Fenster
im Kreis, also garnicht, bewegt hat) nun Segmente aus dem *nächsten* Fenster
designieren. Nun kann es aber vorkommen, dass alle Quittierungen in Transit
verloren. Dann erhält der Sender also keine einzige Besätigung, und darf sein
Fenster auch nicht weiterschieben. Nach einem Timeout sendet er nun alle
Segmente aus seinem unveränderten Fenster nochmals an den Empfänger. Der Sender
denkt ja, dass der Empfänger die Segmente nicht erhalten hat, weil sie nicht
bestätigt wurden. Wenn diese $N$ Segmente mit Sequenznummern gerade
$\mathcal{S}$ nun beim Empfänger ankommen, denkt dieser aber nun, dass das die
Segmente des *nächsten* Fensters sind. Der Empfänger hat die alten ja alle
bekommen und quittiert. Hier ist nun also eine Verwechslung aufgetreten. So
könnte aus diesem Datenstrom nun eine falsche Nachricht rekonstruiert werden.

Diese Art von Problem gibt es bei Go-Back-$N$ und Selective Repeat, wie wir
gleich sehen werden. Es gibt dann aber feste Grenzen für die Fenstergrößen,
unter welchem keine Verwechslungen auftreten können.

Es sei noch angemerkt, dass wenn Sende- und Empfangsfenster beide eins sind, man
wieder Stop and Wait erhält. Eine Steigerung von Stop and Wait erhält man erst,
wenn man *das Sendefenster* erhöht. Die Größe des Sendefensters ist also gerade
das essentielle, was diese beiden Methoden von Stop and Wait unterscheidet. Wenn
das Sendefenster eins ist aber das Empfangsfenster groesser eins, dann bringt
das nichts, weil nie mehr als ein Segment gesendet wird. Wenn das
Empfangsfenster eins ist, muss das Sendefenster eins sein und wir sind wieder
bei diesem Fall. Das Sendefenster ist also eben ausschlaggebend.

###### Go-Back-$N$

Die erste Möglichkeit zur Segmentverlustbehandlung ist das *Go-Back-$N$*
Verfahren. Die Idee hierbei ist es, dass wenn der Empfänger ein Segment nicht
erhält, er darauf verharrt, es zu bekommen. D.h. er akzeptiert stets nur die
nächste von ihm erwartete Sequenznummer. Segmente mit höheren Sequenznummern
werden schlicht und einfach verworfen.

Hat der Sender beispielsweise ein Sendefenster von vier Sequenznummern $\{0, 1,
2, 3\}$, dann sendet er also erstmal alle los. Nun geht das erste (1) Segment in
Transit verloren. Der Empfänger erhält also erstmal das Segment mit
Sequenznummer $0$. Dann sendet er also ein `ACK = 1` Segment zurück, da er ja
das nun erste Segment erwartet und somit das null-te bestätigt. Gleichzeitg
erhält er nun die höheren Sequenznummern $2$ und $3$. Diese verwirft der
Empfänger nun aber einfach! Er wartet nämlich noch immer auf das erste. Konkret
erhält der Empfänger diese höheren Segmente zwar, __er sendet für sie aber
dennoch ein `ACK = 1` zurück__ (man braucht also kein Timeout).

Dieses Verfahren heißt also Go-Back-$N$, weil selbst wenn der Sender alle $N$
Segmente in einem Sequenznummernaum senden würde, das erste aber verloren geht,
der Sender dennoch $N$ Segmente zurückgehen muss. Wie muss das Sendefenster bei
diesem Verfahren nun also gewählt werden, sodass nicht wie oben erläutert
Verwechlungen zwischen Fenstern auftreten? Hierzu die folgenden Beobachtungen:

* Da der Empfänger immer nur das nächste erwartete Segment akzeptiert, ist
  dieses nächste erwartete Segment auch immer die erste Sequenznummer im
  Schiebefenster des Empfängers. Der Empfänger kann ja sein Fenster dann für
  jedes erhaltene, nächste Segment auch immer gleich eins weiterschieben. Wir
  gehen also imemr nur um einen Schritt weiter. Bei selective Repeat (siehe
  unten) könnten wir auf einen Schlag auch viel mehr Schritte gehen, wenn nur
  das erste Segment des Fensters fehlt und die anderen Segmente schon gepuffert
  wurden.
* Die Gefahr, nicht zwischen Fenstern unterscheiden zu können, stammt aus der
  Überlappung von einem Fenster und dem darauffolgenden Fenster. Bei Go-Back-$N$
  ist es nun aber so, dass das nächste Fenster immer nur die allererste
  Sequenznummer akzeptiert, und nicht die darauffolgenden. Solange also nur
  nicht __die Anfänge__ des momentanen Fensters und des nächsten Fensters
  überlappen, wird es bei Go-Back-$N$ keine Fehler geben. Dann kann man nämlich
  alle Segmente des momentanen Fensters mehrmals senden. Da keines der Segmente
  der Anfang des nächsten sein wird, wird es nie eine Verwechslung
  geben. Genauer darf der Anfang des nächsten also gar keine Sequenznummer des
  momentanen Fensters sein. Es gibt aber auch nur einen Fall, wo der Anfang des
  nächsten Fensters eine Sequenznummer des momentanen Fensters ist, und in
  diesem Fall überlappen sich genau die Anfänge.
* Dieser Fall, dass sich der Anfang es nächsten Fensters mit einem Segment des
  momentanen überlappt, ergibt sich wie gesagt nur in einem Fall. Das ist genau
  dann, wenn das Fenster gerade der ganze Sequenzraum ist. Dann ist das nächste
  Fenster nämlich genau eine 360 Rotation weiter, also genau dasselbe
  Fenster. Da die Rotation 360 Grad war, überlappen sich die Anfänge. Dadurch
  gibt es dann also eine Verwechslung, falls der Anfang des vorherigen Segments
  neu gesendet werden muss (Erinnernung: es ist nur der Anfang wichtig, weil
  wenn höhere Sequenznummer wiederholt werden müssen, dann sind diese für das
  nächste Fenster sowieso unininteressant, weil es noch auf sein Anfangssegment
  wartet).
* Wir können dieses Problem, dass sich die Anfänge des momentanen und nächsten
  Fensters überlappen, ganz einfach dadurch lösen, dass wir zwischen dem
  momentanen und nächsten Fenster immer genau eine Sequenznummer Platz
  lassen. D.h. wenn das erste Fenster die Sequenznummern $0, ..., N - 2$
  okkupiert, dann setzen wir den Anfang des nächsten Fensters einfach auf
  Sequenznummer $N - 1$. Dann überlappen sich zwar $N - 2$ Sequenznummern
  zwischen diesen Fenstern (alle außer $N - 2$ und $N - 1$), aber eben nicht die
  Anfänge (an Positionen $N - 1$ und $0$), welche zaehlen.

Zusammenfassend: Bei Go-Back-$N$ muss das Sendefenster also maximal $N - 1$
Sequenznummern okkupieren, *damit man den Anfang des nächsten Fensters in die
Lücke setzen kann*. Dann überlappen sich zwar $N - 2$ Sequenznummern, aber die
Anfänge der Fenster nicht, was die einzige Source of Badness bei Go-Back-$N$
wäre.

Go-Back-$N$ ist einfacher zu implementieren als Selective Repeat, aber weniger
effizient, weil eben moeglicherweise alle $N$ Segmente neu uebermittelt werden
muessen. Insbesondere interessant ist bei Go-Back-$N$ aber, dass man gar keinen
Speicherpuffer beim Empfänger braucht. Da es immer nur ein einziges mögliches
Segment gibt, das der Empfänger erwartet, muss er andere nicht
zwischenspeichern. Daher braucht er auch keinen Puffer.

###### Selective Repeat

Bei Go-Back-$N$ hatte der Empfänger immer nur das nächste erwartete Segment
akzeptiert und erhaltene Segmente mit darauffolgenden Sequenznummern einfach
verworfen. Bei Selective Repeat verwerfen wir diese höeheren Nummern nun nicht,
sondern speichern sie in einen Puffer. Fehlt eine erwartete Sequenznummer im
Fenster, so teilt der Empfänger dem Sender das durch ein passendes ACK mit. Der
Sender schickt dann aber nicht wie bei Go-Back-$N$ alle Segmente ab dem
bestätigten nochmals neu, sondern nur das mit dieser konkreten
Sequenznummer. Gehen mehrere erwartete Segmente verloren, sendet der Empfänger
dem Sender eben mehrere solcher ACKs.

Wählen wir für das Empfangsfenster eine Größe von $1$, so erhalten wir einfach
wieder Go-Back-$N$. Der Empfänger könnte nämlich immer nur das nächste erwartete
Segment speichern und müsste höhere ignorieren. Es wäre aber nicht zwingend
dasselbe wie Stop and Wait, da der Sender ja wohl ein größeres Fenster haben
kann und mehr als ein Segment wegsenden kann, auch wenn alle außer dem ersten
dann ignoriert würden, weil sie nicht ins momentane Empfangsfenster fallen. Im
besten Fall würde der Empfänger aber immer genau ein Segment nach dem anderen
erhalten (unter der Annahme von vernachlaessigbarer Verarbeitungszeit), wodurch
sich das $1$-große Fenster des Empfängers immer um eins weiterschieben
würde. Würde nun aber ein erwartetes Segment nicht ankommen, so würden höhere
ignoriert und müssten dann wie bei Go-Back-$N$ nochmals gesendet
werden. Praktisch gesehen ist das Sendefenster aber durch das Empfangsfenster
nach oben beschraenkt.

Wir überlegen uns wieder, wie groß nun die maximale Größe des Sendefensters sein
darf. Es sei wieder angemerkt, dass *das Empfangsfenster in dieser Diskussion
keine Rolle spielt*. *Nur Ueberlappungen im Sendefenster zaehlen*. Wir machen
das Empfangsfenster ja nur kleiner bzw. nicht gleich dem ganzen Sequenzraum, um
Flusskontrolle zu machen. Theoretisch koennte das Empfangsfenster beliebig gross
sein (beschraenkt durch den Sequenzraum). Überlegen wir uns zunächst, ob $N - 1$
wieder eine passende obere Schranke sein könnte. Dann würden sich das momentane
und nächste Sendefenster also wie oben besprochen in genau $N - 2$ stellen
überlappen (nur nicht im letzten des momentanen und im ersten des nächsten). Bei
Go-Back-$N$ war diese Überlappung kein Problem, da der Empfänger für das nächste
Segment sowieso nur die erste Sequenznummer ($N - 1$) akzeptieren würde, wo
keine Überlappung war. Nun ist es aber so, dass der Empfänger höhere Segmente
für das nächste Fenster __sehr wohl__ akzeptiert. Er puffert ja diese Segmente
bei Selective Repeat. D.h. wir haben mit dieser Überlappung von $N - 2$
Segmenten ein Problem. Und zwar genau $N - 2$ Probleme. Konkret kann man hier
erkennen, dass eine Unterscheidung vom momentanen und nächsten Fenster nur dann
möglich ist, wenn es __überhaupt keine Überlappung__ gibt. Denn jegliche
Überlappung würde bedeuten, dass wenn dieses Segment im momentanen Fenster neu
gesendet werden muss, der Empfänger aber schon auf die Segmente des nächsten
Fensters wartet (weil er alle Segmente erhalten hat, die ACKs aber alle in
Transit krepiert sind), er dieses Segment für das nächste Fenster puffern würde,
und nicht als erneutes Senden eines alten Segments erkennen würde.

Wir wollen also keinerlei Üeberlappung zwischen zwei konsekutiven Fenstern. Das
erreichen wir im günstigsten Fall genau dadurch, dass wir die Sendefenstergröße
auf $\lfloor N/2 \rfloor$ setzen. Dann würde das momentane Fenster nämlich die
eine Hälfte des Sequenznummernaums okkupieren und das nächste Fenster dann immer
die zweite Hälfte. Es gäbe also gar keine Überlappung. Natürlich kann man die
Fenstergrößen auch kleiner wählen, aber dies ist die obere Grenze.

Bei Selective Repeat benötigt der Empfänger einen Puffer (mit Platz fuer
$\lfloor N/2 \rfloor - 1$ Elementen), was bei Go-Back-$N$ noch nicht der Fall
war.

#### TCP

Wir wollen nun *TCP*, das *Transmission Control Protocol*, als ein Beispiel
eines verbindungsorientierten Protokolls untersuchen. Rund 90% des
Internetverkehrs funktioniert über TCP. Es bietet abgesicherte (im Sinne von,
Segmente kommen durch Neu-Übermittlung sicher irgendwann an) sowie
stromorientierte (anstatt nachrichtenorientierte) Übertragung mittels Sliding
Window und Selective Repeat. Bei TCP werden aber nicht ganze Segmente, sondern
immer einzelne Bytes bestätigt. D.h. wenn ein Segment $b$ Bytes enthält, wird
der nächste `ACK` den Offset $b$ gegenüber dem letzten ACK haben.

Ein TCP Segment hat natürlich einen TCP Header. Während der UDP Header wirklich
nur minimal klein war (leichter Wrapper um die Payload), enthält ein TCP Header
nun schon viel mehr Infromationen. Er ist minimal 20 Byte groß (wie ein IP
Header) besteht also aus:

* Einem *Source Port* (wie bei UDP)

* Einem *Destination Port* (wie bei UDP)

* Der *Sequence Number* (!)

* Der *Acknowledgement Number* (`ACK`) (!)
* Einem (Data) *Offset*, der angibt, an welchem Offset die Daten im ganzen TCP
  Segment anfangen. Ein TCP Header kann nämlich noch *Optionen* enthalten. Somit
  gibt dieser Wert also eigentlich an, wie groß der Header insgesamt ist.

* Reserved: Momentan auf Null gesetzter Bereich.

* Dann sechs Flags:

	- *URG* (urgent): Ist dieses Bit gesetzt, dann heißt das, dass das Segment
	*urgent data* enthält. Es gibt dann noch ein *Urgent Pointer* Feld im Header
	(siehe unten). Dann sind die Daten im Segment ab dem ersten Byte *bis zum*
	Urgent Pointer also die wichtigen (*out-of-band*) Daten. Diese sollen dann
	sofort an höhere Schichten weitergeleitet werden. Die restlichen Daten
	werden dann ganz normal verarbeitet. Im Linux-Kernel kann ein Prozess
	z.B. das `SIGURG` Signal erhalten. Dann weiß er, dass auf einem Socket nun
	urgent Daten anstehen. Ueber `select` kann man dann erfahren, auf welchem
	Socket (wenn man viele hat) die urgent Daten anliegen. Denn man kann mit
	`select` nicht nur read und write readiness, sondern auch nach exceptional
	conditions nachfragen.
	https://www.gnu.org/software/libc/manual/html_node/Out_002dof_002dBand-Data.html

   - *ACK* (acknowledgement): Erst durch dieses Flag wird das Acknowledgement
	 Number Feld "aktiv". D.h. wenn dieser Bit gesetzt ist, dann handelt es sich
	 im Ack. Number Field um eine Bestätigungsnummer. Ansonsten wird das Feld
	 ignoriert. Eine "reine Bestätigung" wie wir sie oben betrachtet haben,
	 hätte also keine Daten und dieses Bit gesetzt. In der Praxis ist es aber
	 möglich, dass der Empfänger (bzw. eben Knoten $B$) dem Sender Daten auch
	 senden möchte. Dann kann er in einem Segment gleichzeitig durch die
	 Ack. Number das letzte Segment von Knoten $A$ bestätigen (bzw. das naechste
	 anfordern), und seine Daten an $A$ senden. Dieses Konzept nennt man
	 *piggy-backing*.

   - *PSH* (push): Ist dieses Flag gesetzt, werden die Daten im Segment weder
	 beim Sender noch beim Empfänger gepuffert. Normalerweise würde der TCP
	 Stack des Senders Daten zuerst ein wenig puffern, bevor er ein Segment ins
	 Netzwerk wegschickt und Traffic macht. So wird nicht jeder einzelne Byte,
	 den man in einen Socket reinschickt, sofort zu einem TCP Segment und als
	 ein IP Paket mit einem Byte Nutzdaten weggeschickt. Viel eher wartet man,
	 bis der Puffer voll ist oder man zumindest genügend Daten hat, und schickt
	 dann ein Segment weg. Beim Empfänger ist das genau so. Der TCP Stack wird
	 die Anwendung nicht für jeden einzelnen Byte stören. Auch hier würden Daten
	 zuerst gepuffert. Hat man aber eine interaktive Anwendung, die Daten sofort
	 möchte, dann will man diese Pufferung nicht. Beispielsweise bei Telnet
	 möchte man lieber, dass jeder Byte (also jedes Text-Zeichen) sofort an die
	 Anwendung weitergeleitet wird. Dann setzt man also die push flag. Bei einem
	 Socket gibt es hierfür die `TCP_NODELAY` Sockopt, welche den Nagle
	 Algorithmus für Buffering deaktiviert
	 (https://en.wikipedia.org/wiki/Nagle%27s_algorithm).  Der Unterschied
	 zwischen urgent und push Flag ist nun der, dass selbst wenn ein Segment
	 urgent data enthält, diese dennoch gepuffert werden könnten. Meistens würde
	 man also urgent und push Flag für maximal schnellen Durchsatz gleichzeitig
	 senden.

   - *RST* (reset): Dient dem Abbruch einer TCP-Verbindung ohne ordnungsgemäßen
	 Verbindungsabbau (FIN).

   - *SYN* (synchronization): Ist diese Flag gesetzt, handelt es sich bei dem
	 Segment um eines, welches zum Verbindungsaufbau dient. Ein gesetztes SYN-Flag
	 inkrementiert Sequenz- und Bestätigungsnummer um 1, auch wenn keine Nutzdaten
	 geschickt werden. Das SYN Paket hat nämlich selbst auch eine Sequenznummer.

   - *FIN* (finish): Das Gegenstück zu *SYN*, dient also zum
	 Verbindungsabbau. Inkrementiert ebenso beide Nummern im Header.

* *Receive Window*: Die vom Empfänger mitgeteilte Größe des *Empfangsfensters*
  $W_r$ in Byte. Da die Größe des Sendefensters nur maximal so groß wie das
  Fenster vom Empfänger ist, bestimmt dieser Wert also indirekt auch die
  Sendefenstergröße. Durch diese Option kann der Empfaenger die Datenrate
  drosseln und Flusskontrolle betreiben.

* *Checksum*: Eine weitere Checksum, die wie bei UDP wieder einen Pseudo-Header
  verwendet.

* *Urgent Pointer*: Gibt das Ende der urgent Daten im Segment an. Anfangen tun
  die urgent Daten, sofern das Urgent Flag gesetzt ist, immer beim ersten Byte.

* *Zusätzliche Optionen*, z.B. für Window Scaling. Denn das Window Feld ist nur
  16 Bit groß. Da Sequenznummern bei TCP ja Bytes designieren, könnte man also
  maximal ein Sendefenster von $2^16 - 1$ Byte angeben. Wenn man nun mehr Byte
  möchte, dann kann man diese Zahl einfach skalieren. Der Wert in diesem Feld
  würde also nicht mehr in Vielfachen von Byte, sondern moeglicherweise von
  zwei oder mehr Byte berechnet sein.

http://stackoverflow.com/questions/9153566/difference-between-push-and-urgent-flags-in-tcp
http://www.tcpipguide.com/free/t_TCPPriorityDataTransferUrgentFunction-2.htm

#### MSS

Bei TCP ist immer eine *Maximum Segment Size* (MSS) bekannt. Sie gibt die
maximale Größe der *Nutzdaten* eines TCP Segments (also ohne TCP
Header). Vergleichsweise gibt die MTU auf Schicht zwei die maximale Größe eines
Ethernet Headers (zum Beispiel) an. In Praxis wird die MSS so gewählt, dass
keine IP-Fragmentierung geschehen muss. Die MTU (minus TCP Header) ist dann also
sozusagen eine obere Schranke für die MSS. Bei FastEthernet ist die MTU
beispielsweise 1500 Byte. Minus 20 Byte für IP Header und 20 Byte für TCP Header
bleiben also 1460 Byte für die MSS.

Die MSS Größe kann im Weiteren dann zum Zwecke der Fluss- und Staukontrolle
angepasst werden. Auch sei angemerkt, dass man auf Schicht 4 die MSS nicht immer
genau so wählen kann, dass keine Fragmentierung passiert. Insbesondere weiß man
auf Schicht 4 nicht, welches Schicht 3 oder 2 Protokoll überhaupt angewendet
wird. Selbst wenn man weiß, dass Ethernet und IP genutzt wird, weiß man dann
auch noch nicht, ob der IP Header vielleicht noch Optionen hat, was die MSS
weiter beschraenken wuerde.

#### Flusskontrolle

Da der Empfänger dem Sender die Größe des Empfangsfensters mitteilen kann, kann
der Empfänger auf *Über- oder Unterlast* auf seiner Seite dynamisch
reagieren. Das nennt man dann *Flusskontrolle*. Da weiter bei TCP nicht
Segmente, sondern immer Byte quittiert werden, bestimmt die Größe des
Empfangsfensters also die maximale Anzahl an Bytes, die der Sender in einem Mal
lossenden kann. Wie vorhin besprochen beschraenkt das Empfangsfenster folglich
das Sendefenster. Das ist also die maximale Anzahl an Bytes, die der Sender ohne
Abwarten einer Bestätigung wegsenden kann.

Es sei aber angemerkt, dass das Sendefenster nicht gleich der MSS ist. Denn die
MSS kann aufgrund der MTU des L3 Protokolls unter dem nach Empfaenger moeglichen
Empfangs- bzw. Sendefenster liegen. So ist es beispielsweise moeglich, dass in
einem Sendefenster mehrere MSS versendet werden.

Durch Herabsetzen dieses Fensters kann der Empfänger beispielsweise auf einen
drohenden Overflow seines Empfangspuffers reagieren, wenn er von einem Sender
schneller Daten erhält, als er sie verarbeiten kann.

#### Staukontrolle

Flusskontrolle reagiert auf Überlast beim Empfänger. *Staukontrolle* ist nun
dazu da, Überlast im Netzwerk, also auf dem gesamten Übertragungspfad, zu
vermeiden. Diese Kontrolle wird nun nicht mehr vom Empfänger über das Receive
Window getätigt, sondern vom Sender, wenn er merkt, dass seine Segmente nicht
mehr so oft bestätigt werden.

Hierfür wird beim Sender nun noch ein *Staufenster*, also *Congestion Window*,
$W_c$ eingeführt. Der Sender versucht dieses Fenster maximal groß und
insbesondere maximal nahe an dem Receive Window zu halten. Daher vergrößert er
das Fenster schrittweise, solange Daten verlustfrei übertragen werden. Treten
Verluste auf (kommen keine Quittierungen mehr), so verringert der Sender das
Fenster wieder.

Das Sendefenster wird also nicht nur durch das Empfangsfenster nach oben
begrenzt, sondern auch durch das Congestion Window. Somit gilt also:

$$W_s = \min\{W_c, W_r\}$$

Es gibt bei TCP grundsätzlich zwei *Phasen* der Staukontrolle:

1. *Slow-Start*: Hierbei wird das Congestion Window $W_c$ bei jedem bestätigten
   Segment um eine MSS vergrößert. Im Falle, dass das Receive Window größer als
   das Congestion Window ist, gilt also $W_s = W_c$, wobei wir annehmen, dass
   $W_s = k = MSS$ dann auch ein Vielfaches von der MSS ist. Wir duerfen dann
   also $k$ MSS-grosse Segmente wegsenden. Fuer jedes der $k$ Segmente
   vergroessern wir das Congestion Window (und somit das Sendefenster) um eine
   MSS. Nach Bestaetigung aller $k$ MSS haben wir das Congestion Window also um
   $k \cdot MSS$ Bytes vergroessert, was gerade dem Congestion Window vorher
   entsprach. Somit hat sich die Groesse des Congestion Windows verdoppelt. Nun
   koennen wir $2k$ viele MSS wegsenden, und werden dann nach erfolgreicher
   Bestaetigung aller MSS unser Fenster wieder um $2k$ MSS-Groessen vergroessert
   haben. Dann haben wir also $2 \cdot 2k$ viele MSS Platz und somit das
   Congestion Window wieder vergroessert. Wie man sieht hat man bei Slow-Start
   also *exponentielles* Wachstum nach $W_c \cdot 2^i$, wo $i$ die Anzahl
   erhaltener Congestion Windows, also jeweils $k \cdot MSS$ ist. Man bemerke
   hier auch, dass die MSS konstant ist, weil sie ja z.B. durch die L2 MTU
   beschraenkt ist.

2. *Congestion-Avoidance*: Wird bei Slow-Start nun erkannt, dass es zu einer
   Überlast gekommen ist, wird mit Congestion Avoidance begonnen. Hierbei wird
   das Congestion Window zunächst reduziert (entweder auf die Hälfte oder auf
   die ursprüngliche, kleinste MSS Größe). Dann wird das Congestion Window für
   jedes bestätigte Segment (also jede MSS) nicht um eine ganze MSS, sondern um
   das $MSS/W_c = MSS/(k \cdot MSS) = 1/k$-fache einer MSS erhoeht. Nach einer
   *Round-Trip-Time* (RTT), also nach der Bestaetigung aller MSS des
   Sendefensters (bzw. Staukontrollfensters), hat man das Fenster also $k$ mal
   um das $1/k$-fache einer MSS erhoeht. Somit haben wir nach einer RTT eine
   Steigerung um *exakt eine MSS*. Das bedeutet, dass wir das Congestion Window
   nur mehr *linear* (pro RTT) anwachsen lassen.

Da frühestens nach einer RTT (bei wiederholtem Senden aber manchmal erst nach
mehreren RTTs) eine Bestätigung für ein Segment beim Sender ankommt, ist die
Zeiteinheit auf unserer $x$-Achse in RTTs.

Wann genau mit Slow-Start aufgehört und mit Congestion Avoidance angefangen
wird, und wie der Übergang zu Congestion Avoidance geschieht, ist durch die
konkrete TCP Implementierung bzw. Version bestimmt. Es gibt nämlich mehrere
Versionen von TCP, z.B. TCP Tahoe, Reno, New Reno oder Cubic. Der Linux Kernel
verwendet momentan TCP Cubic, welches das Congestion Window als eine kubische
Funktion mit schnellerem Wachstum modelliert
(http://www4.ncsu.edu/~rhee/export/bitcp/cubic-paper.pdf). Wir behandeln in der
Vorlesung TCP Reno. Es geht auf folgende Weise von Slow Start zu Congestion
Avoidance über:

1. Wenn es drei duplizierte ACKs (*3 duplicate-ACKs*) erhält, also nach dreifach
   fehlgeschlagenenen Senden, reduziert es die Congestion Window Größe auf die
   Hälfte der momentanen Größe. Dann beginnt es mit der (bei TCP Reno) linearen
   Stauvermeidungsphase.
2. Wenn der Sender gar keine ACKs mehr bekommt (also nach einem *Timeout*), dann
   ist das Netzwerk anscheinend vollkommen im Eimer. Dann setzt es also einen
   *Schwellenwert* $W_c/2$ fest und reduziert es das Fenster auf eine MSS
   Größe. Dann beginnt es mit einem neuen (exponentiellen) Slow-Start, bis es
   den Schwellenwert erreicht, worauf es mit Congestion Avoidance beginnt. Es
   geht also erstmal komplett vom Gas, um das Netzwerk ausruhen zu lassen. Dann
   geht es mit Slow Start schnell auf den Schwellenwert.

In beiden Faellen setzen wir den Schwellenwert $W_c/2$ (halbes Congestion
Window) fest, ab welchem wir Congestion Avoidance machen. Der Unterschied ist,
dass wir bei 3-duplicate Acks sofort auf diesen Schwellenwert gehen und mit
Congestion Avoidance beginnen. Bei Timeout gehen wir erst ganz runter auf eine
MSS und fangen dann nach einem neuen Slow Start nach dem Erreichen dieses
Schwellenwertes auch mit Congestion Avoidance an.

#### Anmerkungen

Obwohl TCP eine gesicherte Verbindung ist, dient es nicht der Kompensation eines
unzuverlässigen Physical oder Data Link Layer (Schichten 2 und 3). Also man
kann, nur weil man mit TCP sicher davon ausgehen kann, dass Pakete irgendwann
einmal durch Retransmits ankommen, nicht einfach auf Fehlerkorrekturmechanismen
verzichten. TCP interpretiert Paketverlust nämlich stets als eine
Überlastsituation im Netzwerk. Dann wird die Datenrate also entsprechend
gedrosselt. Wenn der Paketverlust aber auf Grund von Bitfehlern geschieht, dann
ist das eine unnötige Reduktion des Durchsatzes. Insbesondere haben Bitfehler
nichts mit der Ueberlast im Netzwerk oder der momentanen Datenrate zu tun. In
Praxis ist TCP bereits mit 1% Paketverlust, der nicht Überlast zurückzuführen
ist, überfordert.

## NAT

*Network Adress Translation* (*NAT*) ist eine Methode, durch welche lokal, aber
nicht global, eindeutige Adressen auf global eindeutige abgebildet werden
können. Genauer erlaubt NAT eine Übersetzung von *privaten Adressen*,
beispielsweise dem 10.0.0.0/8 Adressraum, auf *öffentliche Adressen*. Die Hosts
innerhalb eines Netzwerkes können dann über ihre privaten Adressen miteinander
kommunizieren. Hierbei kann es in zwei verschiedenen privaten Netzwerken
durchaus diese privaten Adressen mehrmals geben. Um mit Hosts in anderen
Netzwerken zu kommunizieren, werden ihre privaten Adressen dann in global
eindeutige übersetzt.

Formal bezeichnet NAT eine Familie von Techniken, welche es ermöglichen, $N$
private (global nicht eindeutige) Adressen auf $M$ global eindeutige Adressen
abzubilden. Hierbei wird ferner wie folgt zwischen NAT-Typen unterschieden:

* Ist $N \leq M$, kann die Übersetzung statisch geschehen. Dann hat jeder Knoten
  eine private und mindestens eine öffentliche Adresse. Insofern ist dieser Fall
  nicht wirklich interessant, ausser vielleicht bei IPv6, wo wir viele Adressen
  haben und womoeglich mehrere globale Adressen zum selben privaten Rechner
  routen moechten.
* Ist $N > M$, wird eine öffentliche Adresse von mehreren Rechnern gleichzeitg
  verwendet. Eine eindeutige Unterscheidung kann dann mittels
  *Port-Multiplexing* geschehen. Häufig ist $M$ sogar gleich $1$, z.B. bei
  privaten Haushalten.

### Private Adressen

Zunächst wollen wir wiederholen, was private Adressen überhaupt sind. Private
IP-Adressn sind *spezielle Adressebereiche*, die zur privaten Nutzung ohne
vorherige Registeriung freigegeben sind. Sie können in unterschiedlichen
Netzwerken gleichzeitig genutzt werden und werden *nicht geroutet*. Die privaten
Adressbereiche sind:

* 10.0.0.0/8
* 172.16.0.0/12
* 192.168.0.0/16
* 169.254.0.0/16

Der Addressbereich 169.254.0.0/16 wird des Weiteren zur automatischen
Adressvergabe (*Automatic Private IP Adressing*) genutzt. Startet ein Computer
ohne statisch vergebene IP Adresse, so versucht dieser zunächst, einen DHCP
Server (durch eine DHCP Discovery) zu erreichen. Findet sich kein DHCP Server
(erhaelt der Computer keinen DHCP Offer), generiert das Betriebssystem eine
zufällig gewählte Adresse aus diesem Adressblock. Dann broadcastet es einen
ARP-Request, um zu prüfen, ob diese Adresse schon ein anderer Rechner im
Direktnetzwerk hat. Wenn ja, generiert es eine neue. Das passiert so lange, bis
kein ARP Reply mehr kommt und angenommen wird, dass kein anderer Rechner diese
Adresse besitzt.

### NAT im Detail

Im Allgemeinen übernehmen *Router* die NAT Funktion in einem Netzwerk. Der
Router hat dann eine *NAT-Tabelle*, in welcher er im einfachsten Fall vier
Spalten hält:

1. Die private Quelladresse eines Rechners.
2. Der Quellport dieses Rechners.
3. Ein globaler Port des Routers.
4. Das verwendete Protokoll.

Hierbei nehmen wir an, dass der Router nur eine globale Adresse hat ($M =
1$). Will ein Rechner im lokalen Link nun ein Paket an eine globale Adresse
außerhalb des Netzwerkes senden, so sendet er es an den Router (seinen
Gateway). Wenn das das erste Paket von diesem Rechner auf diesem Port ist,
erstellt der Rechner einen neuen Eintrag in der Tabelle. Er vermerkt darin dann
zunächst private Adresse und Port des Quellrechners. Dann gilt es, einen
globalen Port des Routers für diese $(\text{Quelladresse, Quellport})$
Kombination zu finden. Hierzu prüft der Router zunächst, ob der Quellport schon
als globaler Port vergeben wurde. Wenn nicht, übernimmt es den privaten Quellport
einfach als globalen Port und gibt ihn in die Tabelle. Ist der Port schon
vergeben, generiert es einfach einen zufälligen Port. Würde also ein weiterer
lokaler Rechner auf diesem selben Port kommunizieren wollen, würde er eben einen
solchen zufällig gewählten globalen Port zugewiesen bekommen. Die genau Wahl des
ausgehenden Ports ist letztendlich egal (sofern er keinem Well-Known Port
entspricht).

Nachdem der Router diesen Eintrag generiert hat (oder nicht, falls der Eintrag
fuer diesen Rechner und diesen Port schon existiert hat), geschieht die
eigentliche Adressübersetzung. Der Router ersetzt nun die Quell-IP Adresse des
privaten Rechners im Paket, durch die globale Adresse des Routers. Ebenso wird
im UDP/TCP Segment der private Port durch den zugewiesenen globalen Port
ersetzt. Dann schickt der Router das Paket into se world wide webs zum
urspruenglichen Zielrechner (die Ziel-IP-Adresse wurde ja nicht veraendert).

Wenn nun aus se world wide webs eine Antwort auf dieses Paket kommen soll, wird
dies also nicht an den Rechner adressiert sein, sondern an den *Router*. Die
Zieladresse wird also die globale Adresse des Routers sein und ebenso wird der
Port der gewählte globale Port des Routers sein. Dann muss der Router also die
Adressen zurückübersetzten. Hierzu sieht der Router in seiner Tabelle in der
Spalte der globalen Ports nach. Dort findet er die private Adresse und Port des
entsprechenden lokalen Rechners, durch welche er das IP Paket modifiziert (oder
die Payload einfach neu verpackt). Dann schickt er das Paket in den lokalen Link
an den privaten Rechner.

Während des ganzen Prozesses wissen also weder der lokale Quellrechner noch der
Zielrechner, dass sie eigentlich mit dem Router kommunizieren. Der lokale
Rechner sendet das Paket ja einfach an seinen Gateway, im Glauben, dass dieser
das Paket an den Ziel-Host weiterleiten wird. Der Zielrechner hingegen glaubt,
dass er eigentlich mit dem Router kommuniziert bzw. haelt den Router fuer den
Quell-Rechner. Wenn zwei lokale Rechner mit demselben Zielrechner kommunizieren,
glaubt der Zielrechner, dass der Router einfach zwei Verbindungen aufgebaut hat.

### NAT Typen

Die Art von NAT, die wir gerade besprochen haben, nennt sich *Full Cone NAT* und
ist eigentlich die einfachste Art von NAT. Andere Arten von NAT unterscheiden
sich in den Informationen die sie in eine Tabelle aufnehmen, sowie in ihrer
Restriktivität. Router könnten nämlich zusätzlich noch weitere Informationen in
ihre Tabelle aufnehmen, z.B.:

* Ziel-IP
* Ziel-Port
* Die globale IP des Routers, falls er mehrere hat.

Im Weiteren ist die Restriktivität der Übersetzung also ein
Unterscheidungsfaktor. Was meinen wir damit? Nun ja, Full Cone NAT hat folgende
Eigenschaften:

* Bei eingehenden Verbindungen findet keine Prüfung der Absender-IP oder des
  Absender-Ports statt. Der Router speichert auch keinerlei solche Informationen
  in seiner Tabelle.
* Existiert einmal ein Eintrag für einen Rechner in der NAT-Tabelle, so kann
  jeder externer Host mit diesem privaten Rechner über den entsprechenden Port
  kommunizieren (Anmerkung: das waere ohne NAT ja auch so).

Das machen andere Arten von NAT anders:

* *Restricted NAT* leitet nur dann ein Paket an einen internen Host weiter, wenn
  der interne Host zuvor schon ein Paket an die Quell-IP-Adresse des Pakets
  gesendet hat. Der Router wuerde also fuer jeden Eintrag auch noch die Ziel-IP
  speichern. *Der Port wird dabei nicht betrachtet*.

* *Port-Restricted NAT* erhöht die Restriktivität der Weiterleitung noch
  dadurch, dass nur Pakete weitergeleitet werden, wo der interne Host schon ein
  Paket an diese Quell-IP-Adresse und *an diesen Quell-Port* des eingehenden
  Pakets gesendet hat. Dann wird feur jeden Eintrag nicht nur Ziel-IP, sondern
  auch Ziel-Port gespeichert.

http://www.think-like-a-computer.com/2011/09/16/types-of-nat/

Es sei aber angemerkt, dass NAT trotz (Port-)Restriktivität nie als Firewall
betrachtet werden sollte. Eine Firewall hat viel maechtigere Moeglichkeiten,
nach bestimmten Regeln eingehende oder ausgehende Verbindungen zu erlauben, oder
nicht. https://en.wikipedia.org/wiki/Firewall_%28computing%29

Im Weiteren ist es so, dass ein Router einen Eintrag in seiner NAT Tabelle nach
einiger Zeit wieder löscht, um den Port wieder freizugeben. Denn ansonsten ist
das Maximum an Verbindungen eben $2^16$ pro globaler IP Adresse. Oft entfernt
ein Router einen Eintrag auch dann sofort, wenn er einen TCP Verbindungsabbau
erkennt.

Man kann Einträge in NAT Tabellen auch manuell erstellen, um z.B. einen Server
hinter einem NAT-Router auf einem globalen Port erreichbar zu machen. Das nennt
man dann *Port Forwarding*. Sonst müsste der Server zuerst von dem Port, wo er
eigentlich Daten erhalten möchte (z.B. HTTP Port 80), ein Paket an irgendeinen
Rechner senden, damit ein Eintrag generiert wird. Adress- und
Port-Restriktivität würde dann natürlich auch Probleme machen.

### NAT ohne Schicht 4

Natürlich kann man nur Ports in eine Tabelle eintragen, wenn es Ports
gibt. Nicht alle IP Payloads sind ja TCP oder UDP Segmente. Zum Beispiel könnte
ein IP Paket eine ICMP Nachricht enthalten. Dies haben keinen Port. Was tun? Bei
Echo Requests und Replies ist das weniger problematisch, da diese eine
*Identifier* Nummer in ihrem Header haben. Diese wird dann einfach an Stelle
des Ports in die Tabelle eingetragen. Wenn dann der entsprechende Echo Reply
kommt, enthaelt dieser auch denselben Identifier und die Nachricht kann
entsprechend weitergeleitet werden.

Bei anderen ICMP Nachrichten gibt es zwar meist keinen Identifier (z.B. *Time
Exceeded* oder *Destination Unreachable*), aber diese Arten von ICMP Nachrichten
enthalten in ihrer Payload meist noch den IP Header sowie __die ersten 8 Byte
der Payload dieses Pakets__. In diesen 8 Byte stehen bei UDP und TCP dann
insbesondere immer Source und Destination Port, sodass man auf den Port eben
auch zugreifen kann, um den privaten Rechner in der NAT-Tabelle zu
identifizieren.

### NAT bei IPv6

NAT kann auch bei IPv6 angewendet werden. Das ist weniger häufig nötig, da man
bei IPv6 das Problem der Adressknappheit nicht hat. IPv6 hat aber ebenso das
Konzept von privaten Adressen. Diese heißen bei IPv6 dann
*Unique-Local-Unicast-Adressen* und haben den Präfix `fc00::/7`. Die
NAT-Übersetzung benutzt bei IPv6-NAT aber keinerlei Merkmale (Ports, Identifier)
aus Schicht 4. Es werden nämlich einfach Präfixe übersetzt. Z.B. werden alle
lokalen Adressen mit Präfix `fd01:0203:0405::/48` auf den globalen Präfix
`2001:db8:0001::/48` des Routers abgebildet. Erhält ein Router also ein Paket,
das an eine Adresse mit diesem globalen Präfix adressiert ist, ersetzt er
einfach den Präfix. Anders als bei IPv4 passiert bei IPv6 NAT also nie Port
Multiplexing. Wir haben ja für $N$ lokale Adressen ebenso viele globale Adressen
bei dieser Präfixtauschmethode. NAT ist bei IPv6 also einfach eine
Bijektion. Insofern lösen wir das Problem der Adressknappheit nicht. Das ist bei
IPv6 aber eben auch nicht nötig.

%# OSI Model: Layer 5

Der hauptsaechliche Dienst, den die Sitzungsschicht bringt, sind *Sitzungen*
(Sessions). Sessions auf Schicht 5 haben aehnliche moegliche Eigenschaften wie
Verbindungen auf Schicht 4 (Transportschicht), aber eben auf einer hoeheren
Ebene. Insbesondere koennen Sitzungen auf Schicht 5 *mehrere* Verbindungen
unterer Schichten haben. Ein Beispiel dafuer ist das FTP Protokoll. Dieses ist
als Anwendung eigentlich auf Schicht 7, aber es hat auch Sitzungen zwischen
einem Server und einem Client. Waehrend einer solchen Sitzung werden dann *zwei*
TCP Verbindungen aufgebaut und erhalten. Eine als Kontrollkanal, der andere als
Datenkanal. Insofern gibt es also *eine Verbindung (Sitzung)* auf Schicht 5, die
mehrere Verbindungen auf Schicht 4 beinhalten kann.

Es gibt hierbei zwei grundlegende Arten von Sitzungen:

1. *Verbindungsorientierte* Sitzungen sind solche, wo die Verbindung zwischen
   Empfaenger und Sender ueber mehrere Datenuebertragungen, aber auch ueber
   mehrere Verbindungen niedrigerer Schichten gehen kann. Beispielsweise kann es
   in einer solchen verbindungsorientierten Sitzung mehrere TCP Verbindungen
   geben, die nacheinander auf- und wieder abgebaut werden. Wie auf der
   Transportschicht gibt es auch hier das Konzept von *Verbindungsaufbau*,
   *Datentransfer* und *Verbindungsabbau*.

2. *Verbindungslose* Sitzungen haben keinen Zustand. Beispielsweise kann ein
   HTTP Request (*ohne Cookies*) eine verbindunglose Sitzung sein, wo also nur
   eine einmalige Verbindung auf der Transportschicht initiiert wird, aber nach
   diesem Request sofort wieder abgebaut wird. Wuerde die Verbindung auch noch
   fuer den Reply offen gehalten werden, so waere es
   verbindungsorientiert. Wichtig hierbei ist wiederum, dass es auch fuer
   verbindungslose Sitzungen auf Schicht 5 verbindungsorientierte Verbindungen
   auf Schicht 4 geben kann. Auch wenn der HTTP Request alleine steht, kann er
   eine verbindungsorientierte TCP Verbindung nutzen, sodass Pakete gesichert
   uebertragen werden. So kann auch eine E-Mail ueber eine verbindungslose
   Sitzung einmalig gesendet werden, wobei verbindungsorientierte
   Transportprotokolle wie TCP verwendet werden.

Wichtig ist auch, dass eine Sitzung nicht nur zwei Teilnehmer haben muss,
sondern *auch mehr* haben kann. Das laesst sich leicht durch eine
Video-Konferenz mit vielen Teilnehmern verstehen. Das ist beispielsweise eine
Eigenschaft, wo es fuer die Transportschicht keine Parallele gibt.

Verbindungsorientierte Sitzungen koennen im Weiteren folgende Dienste anbieten:

* *Aufbau* und *Abbau* von Sessions,
* Normaler und beschleunigter Datentransfer (*Expedited Data Transfer*,
  z.B. fuer Interrupts),
* *Token-Management* zur Koordination der Teilnehmer (wenn es mehr als zwei gibt),
* Synchronisation und Resynchronisation,
* *Fehlermeldungen* und Aktivitaetsmanagement,
* Erhaltung und Wiederaufnahme von Sessions nach Verbindungsabbruechen (welche
  eine Sitzung ja nicht unterbrechen sollten).

### Transport Layer Security (TLS)

Transport Layer Security (TLS) ist ein Schicht 5/6 Protokoll, dass zur
Verschluesselung von Daten dient. Es ist der Nachfolger von SSL (Secure Sockets
Layer). Es bildet beispielsweise die *Grundlage fuer HTTP__S__* und bietet unter
anderem:

* Authentifizierung: Die beiden Parteien geben ihre Identitaet via Public-Key
  Kryptographieverfahren einander bekannt, um sich zu authentifizieren. Dann
  werden Algorithmus und Schluessel ueber einen privaten Kanal
  ausgetauscht. Insofern ist TLS kein *Verschluesselungsverfahren* sondern nur
  ein *Verschluesselungsprotokoll*, das z.B. AES als Verschluesselungsverfahren
  verwendet. Hierbei wird also sichergestellt, das Verbindungen ueberhaupt erst
  zwischen Teilnehmern erstellt werden, wo das erlaubt ist.
* Integritaetsschutz: Es wird sichergestellt, dass Daten *ohne Veraenderung*
  zwischen Empfaenger und Sender ausgetauscht werden. Das ist nun also eine
  Sicherheit, die angeboten wird, nachdem die Teilnehmer authentifiziert
  wurden.
* Verschluesselung: stellt sicher, dass Daten nicht von Augen gesehen werden
  koennen, die es nicht sehen sollen.

Eine TLS verschluesselte Sitzung kann dann natuerlich wieder viele Verbindungen
der Transportschicht beinhalten. Die Verschlueseslung selbst faellt dabei eher
in den Bereich der Darstellungsschicht (Schicht 6).

%# OSI Modell: Layer 6

Es gibt viele Moeglichkeiten, Daten darzustellen. Damit Sender und Empfaenger
eines Kommunikationskanals miteinander reden koennen, muessen sie sich jedoch
bezueglich einer Darstellung der Daten einig sein ("dieselbe Sprache
sprechen"). Die *Darstellungsschicht* (Schicht 6) ist nun dafuer zustaendig, den
Kommunikationspartnern einer Verbindung eine einheitliche Interpretation von
Daten zu ermoeglichen. Genauer bietet diese Schicht folgende Dienste an:

* Die *Darstellung* von Daten (*Syntax*),
* Die *Datenstrukturen* zur Uebertragung von Daten (z.B. JSON arrays),
* Die Darstellung der Aktionen an diesen Datenstrukturen (`.append`),
* *Datentransformationen* (Umwandlungen in andere Formate).

Hierbei ist anzumerken, dass die Darstellung auf Schicht 6 (z.B. XML) nicht der
Darstellung auf Schicht 7 entsprechen muss (z.B. eine schoene HTML
Website). Insofern ist die Darstellungsschicht nur fuer die *Syntax* der
Datenformate zustaendig, nicht fuer ihre Interpretation, was der *Semantik*
entsprechen wuerde (was mit den Daten dann angefangen wird, also was mit einem
String oder Array dann z.B. gemacht wird).

## Funktionen

Die Darstellungsschicht hat zwei grundlegende Funktionen:

1. Kodierung von Daten (lower level), insbesondere:
  * Uebersetzung zwischen Zeichensaetzen und Codewoertern gemaess
    standardisierter Kodierungsvorschriften (z.B. Unicode als UTF-8)
  * Kompression von Daten (Entfernung unnoetiger Redundanz, Quellenkodierung!)
  * Verschluesselung (z.B. TLS)

2. Strukturierte Darstellung von Daten (higher level)
 * Plattformunabhaengige, einheitliche Darstellung von Daten
   (z.B. JSON/XML/HTML)
 * Uebersetzung zwischen Datenformaten
 * Serialisierung von Binaerdaten (base64 encoding,
   https://en.wikipedia.org/wiki/Base64); *Marshalling*.

Man sieht hier, dass TLS sowohl zur Sitzungsschicht (Verbindungskoordination
sowie Authentifizierung) als auch zur Darstellungsschicht (Verschluesselung
allgemein) gehoert.

## Zeichensaetze und Kodierung

Daten koennen in einer von zwei Formen vorliegen: als menschenlesbare
Textzeichen, oder als computerlesbare Binaerdaten. Wir wollen diese beiden
Varianten naeher betrachten.

### Menschenlesbare Daten

Bei menschenlesbaren Textdaten und -symbolen beschaeftigen wir uns meist mit
*Zeichensaetzen*. Ein Zeichensatz ist dabei eine Menge darstellbarer Textzeichen
(z.B. Buchstaben, Zahlen, Sonderzeichen) sowie deren Zuordnung zu *Codepoints*
-- eine eindeutige Kennzahl bzw. Identifizierung des Zeichens im Zeichensatz
(Codebuch). Die Codepoints muessen hierbei injektiv, aber nicht notwendigerweise
surjektiv vergeben sein (um Platz fuer neue Zeichen zu lassen).

Fuer jeden Zeichensatz (z.B. *Unicode*) muss es dann zumindest eine
*Kodierungsvorschrift* geben. Eine Kodierungsvorschrift legt fest, wie ein
Codepoint in binaerer Form dargestellt wird. Fuer Unicode waere das z.B. UTF-8
oder UTF-16 oder UTF-32 etc. Es kann also viele Kodierungsvorschriften fuer nur
einen Zeichensatz geben.

```
Zeichen ==Zeichensatz==> Codepoint ==Kodierungsvorschrift==> Codewort
   A                      U+0065                             01000001
```

### Binaere Daten

Binaere Daten bestehen zunaechst aus irgendwelchen atomaren Dateneinheiten,
wobei man eine solche Einheit als *Datum* bezeichnet. Daten sind heutzutage
meist Sequenzen von 8 Bit, welche man als *Oktette* oder sprachgebrauchlich auch
als Bytes bezeichnet. Was ein Datum repraesentiert (die *Semantik* der Daten)
ist vom Kontext bzw. der anwendungsspezifischen Interpretation abhaengig.

Ein Zeichensatz verknuepft also ein Zeichen (Text) mit einem Codepoint. Gaengige
Beispiele fuer Zeichensaetze sind ASCII, Unicode oder ISO-8859-1 (Latin-1). Ein
Zeichensatz kann dabei druckbare Zeichen (Buchstaben, Zahlen, Sonderzeichen)
sowie auch Steuerzeichen enthalten (\n, \t, \a).

Zur Uebertragung von Daten muessen Zeichen, bzw. die entsprechenden Codepoints,
*kodiert* werden. Hierbei unterscheidet man zwischen:

* *Fixed-Length* Codes, bei welchen alle Zeichen mit Codewoertern derselben
  Laenge kodiert werden. Beispiele hierfuer sind ASCII (8 Bit) oder UCS-2 (16
  Bit) oder UTF-32, welches Codepoints stets mit 32 Bit kodiert.
* *Variable-Length* Codes, bei denen Zeichen als Codewoerter unterschiedlicher
  Laenge kodiert werden koennen. Bei UTF-8 sind das beispielsweise 1-4 Byte, bei
  UTF-16 2-4 Byte (16-Bit *code units*).

Es sei angemerkt, dass Zeichensaetze nicht immer die dazugehoerige
Kodierungsvorschrift festlegen. ASCII (7 Bit; MSB 0) oder Latin-1 (8 Bit) legen
sie fest. Unicode legt sie nicht fest (UTF-8,-16,-32 wurden demnach also separat
entwickelt).

#### UTF-8

Das Unicode-Transformation-Format-8 (UTF-8) ist eine Kodierungsvorschrift fuer
Unicode Codepoints. Abhaengig vom jeweiligen Codepoint werden Zeichen mit 1 bis
4 Oktetten kodiert:

| Unicode Bereich    | Laenge |             Kodierung               | Nutzbits |
|:-------------------|:-------|:------------------------------------|:---------|
| U+0000 - U+007F    |   1B   | 0xxxxxxx                            |    7     |
| U+0080 - U+07FF    |   2B   | 110xxxxx 10xxxxxx                   |    11    |
| U+0800 - U+FFFF    |   3B   | 1110xxxx 10xxxxxx 10xxxxxx          |    16    |
| U+10000 - U+1FFFFF |   4B   | 11110xxx 10xxxxxx 10xxxxxx 10xxxxxx |    21    |

Hierbei gibt die Notation `U+xxxx` an, dass es sich um einen Unicode Codepoint
handelt. Die Zahlen dahinter sind hierbei in hexadezimaler Notation. Die ersten
Codepoints 128 Codepoints von UTF-8 sind genau ASCII, und koennen deswegen mit
nur einem Byte kodiert werden. Hierbei wird der oberste Bit auf 0 gesetzt, wie
es der ASCII Standard verlangt. Man merkt also, dass UTF-8 rueckwaerts
kompatibel zu ASCII ist (d.h. ASCII-kodierter Text kann als UTF-8 interpretiert
werden, aber nicht immer umgekehrt). Das ist bei UTF-16 beispielsweise nicht der
Fall.

Fuer Codepoints, die hoeher als die ersten 128 liegen, gibt dann immer der erste
Byte der mit variabler Laenge kodierten Codewoerter an, wieviele Bytes fuer
dieses konkrete Codewort *insgesamt* verwendet werden. Die Anzahl (aller Bytes,
inklusive dem ersten) wird hierbei unaer kodiert und mit einer Null
terminiert. Somit enthaelt beispielsweise das erste Byte der Codepunkte zwischen
`U+0080` und `U+07FF`, welche mit zwei Bytes kodiert werden, den Praefix
`110`. Jeder weitere Byte eines Codewoertes (bzw. hier *der* weitere Byte)
beginnt dann mit einem `10` Praefix.

Eine wichtige Eigenschaft von UTF-8 ist, dass es *praefixfrei* ist. Das
bedeutet, dass kein Codewort ein Praefix eines anderen Codewortes ist, und somit
als dieses interpretiert werden koennte. Das bedeutet wiederum, dass man jedes
Codewort individuell betrachten kann und keine speziellen Marker
braucht. Insbesondere kann es so nie Verwechslungen zwischen Codewoertern
geben. Ein Beispiel fuer einen praefixfreien Code waere die Menge $\{55,
9\}$. Hierbei koennte man eine beliebige Folge dieser Codewoerter betrachten,
ohne in die Gefahr zu laufen, dass es zu einer Verwechslung kommt. In
`559999559559559` weiss man z.B. immer, ob eine Ziffer ein Codewort ist, oder
nicht. Bei $\{55, 9, 5, 9\}$ ginge das nicht, weil z.B. $5$ sowohl ein Codewort
fuer sich alleine, als auch als ein Praefix fuer $55$ und $59$ betrachtet werden
koennte (eben nicht *praefixfrei*). Aus `555955` kaeme so keine eindeutige
Interpretation.

UTF-8 ist gerade deswegen praefixfrei, weil das erste Byte, das angibt, wieviele
Bytes dieses Codewort insgesamt hat, eine unaere Kodierung verwendet (und weil
es alle Bytes zaehlt). Wuerde es beispielsweise durch einen `10` Praefix
angeben, dass das Codewort aus zwei Bytes besteht, dann wuesste man bei einem
Byte `10xxxxxx` nicht, ob es sich um den Start eines neuen Codewortes handelt,
oder um einen intermediaeren Byte. Das Problem waere, dass ein mit `10xxxxxx`
kodiertes Codewort eben ein *Praefix* eines anderen sein koennte.

## Strukturierte Darstellung

Damit Anwendungen Daten untereinander austauschen koenenn, muessen sie eine
inheitliche Syntax fuer die ausgetauschten Daten festlegen. Moeglichkeiten
hierfuer waeren:

* (gepackte) `struct`s: Daten wuerden so wie sie im Speicher vorliegen
  uebertragen. Das hat nicht nur das Problem, dass man ueber Huerden springen
  muss, um die Groesse des Structs (wegen moeglichem Compiler-Padding)
  einheitlich zu bestimmen (`__attribute__((packed))` bei GCC) und dass man
  sicherstellen muss, dass beide Parteien dieselbe Programmiersprache, denselben
  Compiler und dieselben Datenstrukturen verwenden. Ebenso muss die Endianness
  der System uebereinstimme. Das ist also sicherlich keine portable oder gute
  Idee.
* Ad-Hoc Datenformate: Formate, die spezifisch fuer die Anwendung entworfen
  werden. Hierbei muss man sich um Dokumentation, Erweiterbarkeit,
  Fehlerfreiheit, Eindeutigkeit etc. kuemmern. Es violated auch das DRY Prinzip.
* Strukturierte, populaere Serialisierungsformate wie JSON oder XML. Fuer diese
  gibt es natuerlich auch mehr Tools.

### JSON

Wir wollen die JavaScript Object Notation (JSON) etwas naeher betrachten. Es
wurde urspruenglich im ECMA-404 (ECMA ist eine Standardisierungsorganisation)
und RFC 7159 spezifiziert und ist ein sprachenunabhaengiges, menschenlesbares
Datenformat. Es definiert folgende *Datentypen*:

* `number`
* `string`
* `boolean`
* `array`
* `object`
* `null`

Zwischen Elementen kann hierbei beliebig viel Whitespace eingefuegt werden. JSON
ist vollkommen textbasiert und wird allgemein mit UTF-8 kodiert. Eine JSON Datei
ist eine ungeordnete Sammlung von Key/Value Paaren, wobei der Key immer ein
Unicode-String ist. Die Values koennen dabei von einem der oben genannten
Datentypen sein.

## Datenkompression

Kompressionsverfahren haben allgemein das Ziel, Redundanz, im
informationstheoretischen Sinn, aus Daten zu nehmen. Wir greifen hierbei also
wieder das Themengebiet der Quellenkodierung aus der physikalischen Schicht
auf. Esw ird hierbei zwischen zwei Arten von Kompressionsverfahren
unterschieden:

1. *Verlustfreie Kompression* (*lossless compression*): hierbei koennen
   komprimierte Daten verlustfrei, d.h. exakt und ohne Informationsverlust
   wiederhergestellt werden. Beispiele fuer lossless Kompressionsformate sind
   *ZIP*, *PNG* (deswegen groesser als JPG) oder *FLAC* (free lossless audio
   codec).

2. *Verlustbehaftete Kompression* (*lossy compression*): ermoeglicht es nicht,
   komprimierte Daten *exakt* wiederherzustellen. Es tritt also bei einer
   Dekomprimierung ein *Verlust* von Information auf (dieser ist meist nicht
   merkbar gross, vor allem wenn man nicht oft komprimiert). Dafuer haben solche
   Daten meist kleinere Groessen und erlaube eine variable
   Kompressionsrate. Beispiele sind *MP3*, *MPEG* oder *JPG*.

### Huffman-Code

*Huffman-Kodierung* bzw. *Huffman-Codes* sind ein gaengiges
Kompressionsverfahren. Es wird beispielsweise von TLS verwendet (vor der
Verschluesselung). Die grundlegende Idee des Huffman Codes ist es, die
informationstheoretische Entropie einer Datenquelle bzw. die
Auftrittswahrscheinlichkeiten von einzelnen Codewoertern auszunutzen. Es ist ja
beispielsweise so, dass nicht alle Buchstaben im Alphabet in der deutschen
Sprache mit derselben Haeufigkeit auftreten. Beispielsweise tritt der Buchstabe
*E* mit 17.4% am haeufigsten auf, gefolgt von *N* mit 9.8%. Eine naive Kodierung
wie ASCII wuerde diese Buchstaben allesamt mit 8 Bit kodieren. Der
Informationsgehalt dieser Codewoerter, $I(E) = -\log(0.174) \approx 1.749$
bzw. $I(N) \approx 2.323$, sagt uns aber, dass man in einem optimalen Code sogar
nur $1.749$ Bits fuer das $E$ braeuchte. Offensichtlich liegt in einem ASCII
Code (7 Bit fuer $E$) also jede Menge Redundanz vor. Der Huffman-Code will diese
rausnehmen.

Betrachten wir anhand eines Beispieles, wie man einen Huffman-Code
konstruiert. Gegeben sei das Alphabet $\mathcal{A} = \{A, B, C, D\}$ sowie
Auftrittswahrscheinlichkeiten $\Pr[X = z]$, dass das von einer Quelle ueber
diesem Alphabet als naechstes emittierte Zeichen gerade $z \in \mathcal{A}$
ist. Hierbei nehmen wir an, dass alle Zeichen unabhaengig voneinander
auftreten. Dann gebe die folgende Tabelle die Auftrittswahrscheinlichkeiten der
einzelnen Zeichen an:

| z | Pr[X = z] |
|:--|:----------|
| A |    0.05   |
| B |    0.15   |
| C |    0.30   |
| D |    0.50   |

Dann tun wir nun folgendes: solange bis wir alle Zeichen bearbeitet haben
waehlen wir jenes von den verbleibenden aus, das die *kleinste*
Auftrittswahrscheinlichkeit hat. Dieses geben wir dann in einen *Baum*, wobei
wir in den Blaettern immer die Zeichen haben werden. Je zwei Blaetter verbinden
wir dann zu einem inneren Knoten, in welchem wir die Summe der
Wahrscheinlichkeiten der einzelnen Zeichen addieren. Schritt fuer Schritt
wandern (addieren) wir so den Baum rauf, bis wir am Ende bei einer
Wahrscheinlichkeit von 1.0 sind. Das saehe dann so aus:

Schritt I: Wir nehmen das Zeichen mit der kleinsten Auftrittswahrscheinlichkeit
           und geben es in den Baum:

```
0.05 (A)
```

Schritt II: Wir nehmen das Zeichen mit der naechst kleineren Wahrscheinlichkeit
            und verbinden es mit $A$ zu einem inneren Knoten:

```
      0.20
     /     \
    /       \
0.05 (A)  0.15 (B)
```

Dann immer so weiter:


```
                 1.0
                /   \
               /     \
              /       \
            0.50       \
           /    \       \
          /      \       \
         /        \       \
      0.20         \       \
     /     \        \       \
    /       \        \       \
0.05 (A)  0.15 (B) 0.30 (C) 0.50 (D)
```

Die Idee ist nun die folgende: linke Wege repraesentieren immer eine 0 und
rechte Wege immer eine 1. Um den Code fuer ein Zeichen zu bestimmen, wandern wir
__von der Wurzel__ den Baum entlang __zum Zeichen__ (einem Blatt) und notieren
fuer jede Verbindung den entsprechenden Bit. Mit Labels saehe das so aus:

```
                 1.0
                /   \
              0/     \
              /       \
            0.50       \
           /    \       \1
         0/      \       \
         /        \1      \
      0.20         \       \
    0/     \1       \       \
    /       \        \       \
0.05 (A)  0.15 (B) 0.30 (C) 0.50 (D)
```

So ergeben sich die folgenden Codes:

| z | Code |
|:--|:-----|
| A | 000  |
| B | 001  |
| C | 01   |
| D | 1    |

Wir koennen nun zwei Sachen erkennen. Zum Einen ist diese Kodierung effizienter
als eine naive Kodierung. Mit einer naiven Kodierung meinen wir jene, die
einfach $\lceil\log_2(N)\rceil$ Bits verwendet, wobei $N$ die Anzahl an Zeichen
ist. Diese Kodierung geht naemlich davon aus, dass alle Zeichen dieselbe
Auftrittswahrscheinlichkeit haben (dass die Quellenentropie also maximal
ist). Hier koennen wir uns die Codeffizienz nun so berechnen, dass wir immer die
Auftrittswahrscheinlichkeit eines Zeichens mit der Laenge des entsprechenden
Huffman-Codewortes fuer dieses Zeichen multiplizieren (also der Erwartungswert
in der Laenge):

$$E[l_H(z)] = \sum_{z \in \mathcal{A}} \Pr[X = z] \cdot l_H(z)$$

In unserem Fall waere das also:

$$p_A \cdot l_H(A) + p_B \cdot l_H(B) + p_C \cdot l_H(C) + p_D \cdot l_H(D)$$

Das ergibt:

$$0.05 \cdot 3 + 0.15 \cdot 3 + 0.30 \cdot 2 + 0.50 \cdot 1 = 1.7$$

Wie wir sehen benoetigt dieser Code also durchschnittlich nur 1.7 Bit pro
Codewort. Das ist eine Verbesserung von:

$$1 - \frac{E[l_H(z)]}{E[l(z)]} = 1 - \frac{1.7}{2} = 15\%$$

Wir sehen also, dass der Huffman-Code die Wortlaenge reduziert hat. Genauer: er
minimiert sie sogar. Deswegen nennt man ihn auch einen *optimalen Code*. Wichtig
ist auch, dass er *praefixfrei* ist. Wir koennen oben z.B. beobachten, dass
keines der Codewoerter ein Praefix eines anderen Codewortes ist (deswegen fangen
alle Codewoerter ausser $D$ mit 0 an, damit wir $D$ mit *nur einem Bit* (!)
kodieren koennen).

Zur Erstellung eines Huffman Codes braucht die dekomprimierende Seite die
Auftrittswahrscheinlichkeiten oder einfach das Codebuch. Bei komplexeren Daten,
z.B. ganzen Woertern anstelle von einzelnen Zeichen, sind diese oft schwerer
festzulegen. Theoretisch koennen Zeichenhaeufigkeiten dynamisch bestimmt
werden. Dann braucht der Empfaenger aber eben noch das Codebuch.

### Run-Length Codes

Run-Length Encodings sind eine weitere verlustlose Kompressionsstrategie. Die
grundlegende Idee hierbei ist es, haeufige Wiederholungen einzelner Zeichen
(*runs*) so zu kodieren, dass es nur einmal vorkommt und irgendwie festzulegen,
wie oft dieses Zeichen wiederholt werden soll. Im einfachsten Fall wird vor das
betroffene Zeichen die Anzahl der Wiederholungen kodiert.  Betrachten wir
beispielsweise die Zeichenkette `AAAAAAAAAAAAAAAAAAABBC`. Diese hat, so wie sie
hier steht, die Laenge 22. Run-Length Encoding wuerde sie nun einfach als
`19A2BC` kodieren. Nun hat die Zeichenkette nur mehr eine Laenge von 6 und kann
verlustlos dekomprimiert werden. Ein weiteres, aehnilches und maechtigeres
Verfahren ist Lempel-Ziv Kodierung (z.B. LZ77
https://de.wikipedia.org/wiki/LZ77). Im Allgemeinen ist das Run-Length Encoding
Verfahren nicht optimal.

%# OSI Model: Layer 7

Die Anwendungsschicht ist die hoechste und letzte Schicht im Open Systems
Interconnect (OSI) Modell (se final layer, tadadada, tadadadada). Auf dieser
Ebene beschaeftigen wir uns mit vielen Protokollen, mit denen wir als
Internetnutzer oftmals direkt interagieren. Beispiele hierfuer sind:

* Domain Name System (*DNS*)
* HyperText Transfer Protokol (*HTTP*)
* File Transfer Protocol (*FTP*)
* Simple Mail Transfer Protocol (*SMTP*)
* Post Office Protocol / Internet Message Access Protocol (*POP/IMAP*)
* *Telnet*
* Secure Shell (*SSH*)

Wir wollen die meisten dieser Protokolle nun naeher betrachten.

## DNS

Das *Domain Name System* (*DNS*) ist dazu da, menschenlesbare Website
Domain-Namen in IP-Adressen umzuwandeln. Menschen ist es natuerlich (meist) viel
lieber, einfach `www.google.com` einzugeben, als die IP Adresse eines Google
Servers (`1e100.com`). Mittels DNS kann man bestimmte Knoten im Netzwerk
ansprechen, um die IP Adresse hinter einem Domain Namen zu erhalten. Denn nur
ueber IP Adressen kann man auch Daten (via IP Paketen) senden.

Das Domain Name System besteht aus drei wesentlichen Komponenten:

1. Dem *Domain Namespace*. Das ist ein hierarchisch aufgebauter Namensraum mit
   baumartiger Struktur, *aus* welchem man die Abbildung von Domain Namen auf IP
   Adressen erhalten kann.
2. *Nameservern*: Diese speichern Informationen ueber den Namensraum, wobei
   jeder Server nur einen kleinen Ausschnitt (Teilbaum) der ganzen Abbildung
   kennt.
3. *Resolver* sind Programme, die via *DNS Requests* die IP Adressen von
   Nameservern im Domain Namespace extrahieren. Will man als Nutzer einen Domain
   Namen in eine IP Adresse umwandeln, so wird man (bzw. der Browser) ueber den
   Resolver agieren.

### Namespace

Ein kleiner Ausschnitt aus dem Domain Namespace sieht in etwa so aus:

```
          ___ . ____
         /          \
        /            \
       /              \
     com              org
    /   \            /   \
   /     \          /     \
google reddit  wikipedia google
  |                |       |
 www              www     www
```

In diesem Namespace Baum nennt man jeden Knoten ein *Label*. Ein *Domain Name*
ist dann eine Sequenz solcher Labels. Hierbei unterscheiden wir noch zwischen
*Fully Qualified Domain Names* (*FQDN*) und *relativen Domain Names*. Ein FQDN
ist dabei eine Sequenz von Labels, die __von einem Knoten bis zur *Wurzel*
geht__. Man bemerke, dass das Label in der Wurzel des Namespaces ein *Punkt*
ist. Insofern ist `www.google.com` noch kein FQDN, sondern erst
`www.google.com.`. Alle anderen Domain Namen sind naemlich relativ zu einem
anderen. Beispielsweise ist `hangouts.google` relativ zum FQDN
`hangouts.google.com.` (man beachte den hinteren Punkt!). Wir wissen anhand
eines Domain Namen im Uebrigen noch nicht, ob dieser ueberhaupt valide ist und
in eine Adresse umgewandelt werden kann. So ist `asdfaspfihaeorugofadsdaf.com.`
zwar ein syntaktisch gueltiger FQDN, er kann aber nicht auf eine IP Adresse
abgebildet werden.

Hierbei befolgen Labels folgende Syntax:

* Es sind nur Buchstaben, Zahlen und Bindestriche erlaubt.
* Der Bindestrich darf hierbei nicht das erste oder letzte Zeichen sein.
* Zwischen Gross- und Kleinschreibung wird nicht unterschieden.

Fuer die ersten drei Ebenen im Namespace Baum haben wir eigene Bezeichnungen:

1. Ganz oben ist die *Root Ebene*, die nur aus dem `.` Label besteht.
2. Darunter kommt die *Top Level Domain* (*TLD*) Zone. Sie beinhaltet Labels wie
   `.net`, `.com` oder `.org`.
3. In der dritten Ebene befinden sich die *Second Level Domains* (*SLD*) wie
   `google`, `reddit` oder `facebook`.
4. (Inoffiziell) folgen nach den Second Level Domains noch die *Subdomains*,
   z.B. `plus`(`google.com.`) sowie auch `www`.

Genauer gesagt ist das Label der Root Zone nicht der `.`, sondern der *empty
string*! Der `.` ist lediglich die Trennung zwischen zwei Labels, also bei
`www.google.com.` die Trennung zwischen `www.google.com` und dem emtpy label.

### Nameserver

Der Domain Namespace kann in gewisserweise als eine grosse, verteilte Datenbank
betrachtet werden. Die einzelnen Server, auf welchen die Daten (also Labels
bzw. Domain Namen) des Namespaces gespeichert werden, nennt man hierbei
*Nameserver*. Jeder *Nameserver* kennt dabei nur einen kleinen Teil des gesamten
Raumes. Genauer kennt er nur eine *Zone*. Hierbei bezeichnen wir mit einer Zone
einen zusammenhaengenden *Teilbaum* des ganzen Baumes, welcher viele Knoten auf
vielen Ebenen *unter* sich enthalten kann. Jeder Knoten ist hierbei natuerlich
wieder ein Label.

Ein Nameserver kennt immer nur eine Zone. Man nennt diesen Server dann
*authoritativ* fuer diese Zone. Wenn er authoritativ ist, muss er alles ueber
seine Zone wissen, oder zumindest wissen, wie er an Informationen ueber seine
Zone kommt (also wie er andere Nameserver in seiner Zone anfragen kann). Es muss
aber nicht nur einen authoritativen Nameserver fuer eine Zone geben, sondern es
kann auch mehrere geben. Hierbei gibt es aber immer einen *primaeren*
authoritativen Server und alle anderen werden dann *sekundaer* genannt. Der Sinn
hinter dieser Redundanz ist es, dem primaeren Nameserver bei Ueberlast zu
unterstuetzen sowie ihn bei dessen Ausfall (temporaer) zu ersetzen. Die
Aufteilung *primaer* und *sekundaer* ist zusaetzlich auch dazu da, um
Aenderungen an den Informationen eines Nameservers zu koordinieren. Will man
einen Nameserver updaten, so fuehrt man diese Updates immer am primaeren
Nameserver einer Zone aus. Die sekundaeren sind dann lediglich dazu da bzw. dazu
befugt, Kopien der Informationen des primaeren Servers zu speichern. Auch ist es
moeglich, die Verantwortung ueber Zonen zwischen Nameservern auszutauschen. Das
nennt man dann *Zone Transfers*.

Hierbei gelten die Konventionen:
* Anfragen, die 512 Byte nicht uebersteigen, werden an UDP Port 53 gestellt.
* Anfragen, die groesser als 512 Byte sind, werden an TCP Port 53 gestellt.
* Zone Transfers finden immer ueber TCP Port 53 statt.

Es gibt 13 logische Nameserver (logisch im Sinne dessen, dass das nur Reverse
Proxies sind) fuer die Root Zone (empty label). Diese kann man hier finden:
http://www.root-servers.org und haben auch den `root-servers.org` Suffix
(z.B. `a.root-servers.org`, welcher von Verisign operiert wird). Die root zone
file ist 1 MB gross (enthaelt die ganzen
TLDs).

https://en.wikipedia.org/wiki/Root_name_server
http://www.internic.net/domain/root.zone

### Zone Files

Informationen ueber eine Zone werden in *Zone Files* auf einem authoritativen
Nameserver gespeichert. Ein Zone File enthaelt hierbei eine Anzahl an *Resource
Records* (RR), welche fuer die Art von Resource Record spezifische Informationen
ueber die Zone angeben. Zone Files werden in Plain Text gespeichert. Resource
Records haben dabei das folgende (kanonische) Format in einer Zone File:

```
name ttl record-class record-type record-data
```

Hierbei sind je zwei Spalten (Attribute) durch Whitespace (Tab oder Leerzeichen)
getrennt. Auf einer Zeile darf es dann immer nur einen RR geben. Zu den
einzelnen Komponenten:

* `name`:
  - Hier gibt man den Domain Namen an, fuer welchen der RR etwas aussagen soll.
  - Laesst man ihn weg, so bezieht sich der Eintrag implizit auf den FQDN des
    vorherigen Eintrags in der Zone File.
  - Ist es kein FQDN, sondern nur ein relativer Domain Name, so ist er relativ
   zum `$ORIGIN`. Man kann naemlich am Anfang (bzw. irgendwo vor einem
   entsprechenden Eintrag) in der Datei einen solchen `$ORIGIN` Marker
   festlegen. Beispielsweise `$ORIGIN example.com.`. Dann beziehen sich also
   alle `name`s, die nicht mit einem Punkt enden (FQDNs sind), auf den
   `$ORIGIN`. `foobar` wuerde dann z.B. `foobar.example.com.` meinen. `@`
   bezeichnet dann auch immer den in der Variable `$ORIGIN` gehaltenen
   Namen. Somit kann man z.B. einen Resource Record fuer `@`, also den
   `$ORIGIN`, spezifizieren.

* `ttl`:
  - Hier kann man eine *Time to Live* Zeit in Sekunden oder, seit neueren BIND
    Versionen (das ist die Software, mit welcher DNS auf Servern implementiert
    wird), auch in anderen Zeiteinheiten angeben. Diese Zeit gibt an, wie lange
    DNS *Clients*, also insbesondere Resolver, einen Eintrag cachen duerfen. Er
    sagt also insofern nichts darueber aus, wie lange der Eintrag fuer den
    *Server* gueltig ist -- nur fuer Resolver (oder allgemein Clients, die
    direkt Anfragen an den Server machen). Alternativ kann man einen globalen
    `$TTL` Eintrag haben (das ist wie `$ORIGIN` oder `$INCLUDE` eine Direktive)
    und da einfach die Zeit angeben. Dann wird fuer jeden RR, der keinen TTL
    explizit angibt, die globale TTL als Default genommen.

* `record class`:
  - Gibt den *Namespace* des RR an. Das ist heutzutage immer `IN`, also das
    Internet. DNS gibt es naemlich schon seit den 1980ern, wo man noch an
    anderen Namespaces neben dem Internet, z.B. dem CHAOSNet
    (https://en.wikipedia.org/wiki/Chaosnet), gearbeitet hat.

* `record type`:
  - Record types sind neben dem FQDN der wichtigste Eintrag. Es gibt an, welche
    Art von Information mit dem FQDN assoziiert werden soll.
  - *SOA* (*Start of Authority*) records enthalten Metainformationen ueber die
	Zone, fuer die ein Nameserver authoritativ ist (jeder Nameserver ist fuer
	irgendeine Zone authoritativ). Sie sind insbesondere fuer sekundaere
	Nameserver interessant.
  - *NS* Records geben den *FQDN* eines Nameservers an, welche auch in anderen
	Zonen liegen koennen. Nach [RFC 1034](https://tools.ietf.org/html/rfc1034)
	muss jeder Nameserver mindestens zwei NS Eintraege fuer seine Zone (dem
	Domain Namen, fuer welchen er authoritativ ist) haben, die nicht zur selben
	IP Adresse resolvieren. Der erste Eintrag kann dabei der Nameserver selbst
	sein, der zweite muss (zumindest beim primaeren Nameserver) auf einen
	anderen (*sekundaeren*) Nameserver zeigen. Das verlangt die IANA bzw. der
	RFC so, einfach um die Reliability von Websiten (bzw. Domains) im Internet
	zu erhoehen.
  - *A* Records: assoziieren einen FQDN aus der Zone, fuer welche der Nameserver
	authoritativ ist, mit einer bekannten *IPv4* Adresse.
  - *AAAA* (*quad-A*) Records: wie *A* records, aber fuer *IPv6* Adressen.
  - *CNAME* (*canonical name*) Records sagen aus, dass der FQDN nur ein Alias
    ist, und dass sein kanoonischer (echter) Name der angegebene ist (in
    `data`). Der Resolver soll also bitteschoen eine neue Anfrage fuer den
    kanonischen Namen machen.
  - *MX* Records: geben den FQDN eines *Mailservers* (Mail Transfer Agent; MTA)
    fuer eine Domain an.
  - *TXT* Records: assoziieren einen FQDN mit einem String (Text).
  - *PTR* Records: diese werden fuer *Reverse-DNS* verwendet, wobei *IP Adressen
    auf Domain Namen* abgebildet werden. Ein PTR Record speichert also fuer eine
    IP Adresse einen entsprechenden Domain Namen. Solche Eintrage haben
    eigentlich nur die `in-addr.arpa` server.

#### Start of Authority (SOA) Records

*Jeder Nameserver* muss einen Start of Authority (SOA) Eintrag haben. Dieser
enthaelt wichtige Metainformationen zur Zone File, die vor allem fuer sekundaere
(Slave) Nameserver wichtig sind. Ein SOA Record enthaelt:

1. Alles was auch jeder andere Eintrag braucht: `FQDN` des Servers; TTL, falls
   kein globaler gesetzt oder gewuenscht ist; record class (`IN`) und record
   type `SOA`. Erst `data` ist interessant.
2. `MNAME`: Primary (*Master*) *Name Server*. Das ist der FQDN des Primary (Master)
   Name Server fuer diese Zone. Es darf immer nur einen authoritativen Master
   Name Server. Im Falle des Master Servers ist das der FQDN des Servers selbst
   (bzw. der Zone, fuer welche er authoritativ ist). Fuer sekundaere (Slave)
   Server steht hier aber der FQDN des Master Servers.
3. `RNAME`: *Responsible Person*. Hier steht die Email Adresse des
   Administrators fuer diese Zone. Die Email Adresse muss hierbei das Format
   `username.domain.` haben, wobei man normal `username@domain` schreiben
   wuerde.
4. `SERIAL`: Gibt einen monoton ansteigenden Veraenderungscount der Zone File
   an. Jedes Mal wenn man am Primary Server was aendert, sollte die Serial
   inkrementiert bzw. erhoeht werden. Sekundaere Server werden dann ab und an
   (siehe `REFRESH` unten) beim Master nachfragen, ob er denn Veraenderungen
   hat, die die sekundaeren Server auch haben sollten. Das erkennen sie daran,
   dass ihr eigener `SERIAL` geringer als der `SERIAL` des Masters waere. Dann
   wuerden sie also ein *Zone Transfer Update* initialisieren. Oftmals gibt man
   diesen Wert auch als Datum `YYYYMMDDNN` an, wo `NN` noch ein Count fuer den
   Tag ist, falls die Zone File oefter als einmal an einem Tag geandert wird.
5. `REFRESH`: Sagt aus, in welchem Zeitintervall sekundaere Nameserver bei ihrem
   Master nach Updates sehen sollen. Dieser Wert sollte nicht zu klein sein, um
   den Primary Server nicht zu ueberschwemmen. Er sollte aber auch nicht zu
   gross sein, um schnell Konvergenz zu erreichen. Typische Werte liegen
   zwischen einem Tag und 30 Minuten.
6. `RETRY`: Wie lange Slave Server nach einer gescheiterten Update-Anfrage
   warten sollen, bevor sie eine neue Anfrage an den Master machen. Es kann
   naemlich zum Beispiel sein, dass der Master gerade einfach zu beschaeftigt
   war (oder tot). Das ganze waere also, nachdem die `REFRESH` Zeit verstrichen
   ist. Typische Werte liegen zwischen einer Stunde und 10 Minuten.
7. `EXPIRE`: Wie lange es dauern soll, bevor ein sekundaerer Nameserver sich
   selbst invalidieren soll, wenn er keine Updates von seinem Master erhalten
   kann. Das waere beispielsweise der Fall, wenn der Master Server gerade down
   ist. Es ist wichtig, diesen Wert relativ hoch zu halten, damit eine Website
   auch noch laenger erreichbar ist, wenn der Primary Server stirbt. Typische
   Werte sind zwischen vier Wochen und einer Woche.
8. `MINIMUM`: Dieser Wert hat viele Interpretationen. Die Offizielle ist aber
   (nach neuestem RFC), dass dieser Wert angibt, wie lange *negative* Antworten
   dieses Servers von Resolvern gecached werden sollen. Also falls ein Resolver
   eine IP Adresse fuer eine Domain haben moechte, fuer welche der Nameserver
   authoritativ ist und somit die Antwort wissen muss, *falls es ein bekannter
   bzw. existenter FQDN ist*, der Nameserver aber keine Antwort weiss
   (bzw. seine delegierten Server), dann soll der Resolver sich so lange
   speichern, dass es diesen Eintrag *nicht gibt*. Laut RFC 2308 ist der
   maximale Wert hierfuer drei Stunden.

http://www.peerwisdom.org/2013/05/15/dns-understanding-the-soa-record/
http://www.zytrax.com/books/dns/ch8/soa.html

#### Quest for "How to Internet": Registrars

Was macht ein Registar? Wie funktioniert das, dass mein Namecheap Registrar
`.me` Adressen (Montenegro) speichern darf. Antwort: Der Registrar darf Domains
aus der `.me` Zone verkaufen, muss dann aber an die montenegrinische
Internetbehoerde Geld zahlen. Die montenegrinischen Server werden dann fuer den
`goldsborough.me.` einen `NS` Eintrag auf einen Nameserver von Namecheap
speichern, sodass jeglicher Traffic auf meine Website dorthin geforwarded wird.

http://stackoverflow.com/questions/2235607/how-does-domain-registration-work/2235644#2235644

### Resolver

Resolver sind Knoten im Netzwerk, die fuer einen Client Informationen aus dem
DNS extrahieren. Da das DNS eine Art verteilte Datenbank ist und kein einzelner
Nameserver alle Zonen kennt, sind in der Regel mehrere Anfragen vom Resolver
notwendig.

Moechte ein Client einen Domain Namen zu einer IP Adresse aufloesen, so macht
der Client eine *rekursive Anfrage*. Das bedeutet, dass er die volle,
letzendliche Antwort will, also die IP Adresse. Kann der Resolver die
Information nicht genau bestimmen, sondern gar nicht (der Domain Name ist
ungueltig) oder nur teilweise (nur den naechstbesten Nameserver), so soll der
Resolver eine Fehlermeldung an den Client (Computer) zurueckgeben.

Um die rekursive Anfrage zu erfuellen macht der Resolver selbst *iterative*
Anfragen an authoritative Nameserver. *Iterativ* meint, dass wenn der Resolver
einen solchen DNS Request macht, es auch OK ist, wenn er nur teilweise eine
Antwort erhaelt. Beispielsweise wenn der Resolver einen authoritativen
Nameserver der `com.` Zone anfragt, um auf `hangouts.google.com..` zu kommen,
und der `com.` Server ihm nur den DNS Nameserver von `google.com.`
zurueckgibt. Dann ist der Resolver mit dieser unvollstaendigen Antwort (im Sinne
dessen, dass es noch nicht die gewuenschte aufgeloeste IP Adresse ist) erstmal
zufrieden, und fragt dann *iterativ* den erhaltenen Nameserver von `google.com.`
an. Dieser weiss dann hoffentlich die IP Adresse von `hangouts.google.com` oder
einen Nameserver, den man diesbezueglich weiter *iterativ* anfragen kann.

Sobald der Resolver die Adresse hat, kann er sie *cachen* und dan den Client
zurueckgeben. Bei der naechsten Anfrage des Clients nach `hangouts.google.com.`
kann der Resolver die Antwort sogleich, ohne weitere Anfragen an Nameserver,
zurueckgeben, sofern die `TTL` des Eintrags noch gueltig (groesser 0) ist. Diese
TTL weiss der Resolver aus dem Resource Record, den er letztendlich vom
Nameserver erhalten hat, der fuer `hangouts.google.com.` authoritativ war.

Damit ein Resolver weiss, wo er ueberhaupt mit seinen Anfragen anfangen soll,
hat er sogenannte *Root Hints*. Das ist eine statische Liste der 13 *logischen*
Root Nameserver. Diese Server sind logisch, weil sich dahinter eigentlich
hunderte, per *anycast* erreichbare, Server befinden. Sie sind also nur *Reverse
Proxies*. Diese werden von verschiedenen Institutionen, z.B. der ICANN, NASA,
U.S. Army und anderen betrieben. Es gibt genau 13 Server, damit man alle 13
Adressen (32-Bit IP Adressen) in ein Datagramm unter 512 Byte geben kann. Das
ist naemlich was von der MTU nach allen weiteren Daten fuer eine solche
Nachricht uebrig bleibt (oder fuer eine 512 Byte UDP Nachricht?). Denn da diese
Informationen oft angefragt werden, ist es ratsam, IP Fragmentierung zu
vermeiden.

Damit ein Client im Uebrigen weiss, was sein Resolver ist, muss er einen
entsprechenden Verweis auf einen Resolver in einer lokalen Datei `resolver.conf`
speichern. Oftmals haben Router Resolverfunktionen. Man kann aber auch den
Google Resolver bei `8.8.8.8` als Resolver benutzen (damit Google seinen Traffic
tracken kann). Oft benutzt man auch einen Resolver seines ISPs. Meist
funktioniert das dann so, dass man via DHCP mit jedem IP Lease auch einen
entsprechenden Resolver zugeordnet bekommt.

https://technet.microsoft.com/en-us/library/cc961401.aspx

### Reverse DNS

Via dem DNS kann man Domain Namen zu IP Adressen umwandeln. Es ist aber auch
moeglich, den umgekehrten Weg zu gehen und IP Adressen auf Domain Namen zu
mappen. Das nennt man dann *Reverse DNS* und wird durch *PTR* (Pointer) Records
in Zone Files ermoeglicht. Hierfuer gibt es zwei eigene Zonen im DNS:

* `in-addr.arpa` fuer IPv4
* `ip6.arpa` fuer IPv6

Der `in-addr.arpa` Baum ist ein 4-stufiger Trie mit Fanout fuer je einem Oktett,
also 8 der 32 IPv4 Adressbits, bzw. 256 Kindern pro Knoten. Fuer IPv6 gibt es
einen aehnlichen Baum, wobei auch nach 4 Bit Grenzen (einer Hex Ziffer)
unterschieden wird. Man hat dann also 32 Ebenen anstelle von nur vier.

Man kann sich die entsprechende Zone File hier ansehen:
https://www.internic.net/domain/in-addr.arpa

Man sieht in dieser Zone File, dass der Nameserver `in-addr.arpa.` eigentlich
nur 256 `NS` Eintraege hat, fuer je ein Klasse A Netzwerk. Z.B. gibt es einen
Eintrag

`175.in-addr.arpa.  86400  IN  NS  ns1.apnic.net.`

Hier steht also, dass Resolver ihre Reverse-Lookups fuer `175.xxxx.xxxx.xxxx`
Adressen beim `ns1.apnic.net` Nameserver machen sollen. Dieser haette dann wohl
weitere `NS` Eintraege und irgendwann hat ein Server dann letzendlich die `PTR`
Eintraege, die eine IP Adresse mit einem FQDN assoziieren.

### Beispiel

Ich greife auf `goldsborough.me` zu, was passiert? (Hypothese)

1. Die Anfrage geht ueber UDP Port 53 an meinen Resolver, den ich z.B. via DHCP
   bekommen habe.
2. Der Resolver macht eine erste Anfrage bei einem der 13 logischen Root
   Nameservern (`[a-m].root-servers.org`). Einer der Root Server antwortet ihm
   mit dem FQDN des Nameservers der montenigrinischen Internet-Behoerde.
3. Der Namecheap Registrar einen Vertrag mit dieser Internet-Behoerde
   geschlossen, sodass der Registar `.me` Domains verkaufen kann. Verwalten muss
   diese Domains aber Namecheap. In der Zone File der authoritativen `.me.`
   Nameserver steht also ein `NS` Resource Record fuer `goldsborough.me.`, wo
   der FQDN eines Nameservers von Namecheap steht.
4. Da `goldsborough.me` nur ein Alias fuer `goldsborough.github.io` ist, hat der
   Namecheap NS lediglich einen `CNAME` Eintrag fuer meine Domain.
5. Erst der NS von `github.io.` hat einen `A` und womoeglich `AAAA` Eintrag fuer
   `goldsborough` (relativ zur `$ORIGIN github.io.` Direktive).

### Das Internet

https://www.internic.net/domain/

## URLs

Mit FQDN koennen wir Rechner global ansprechen bzw. deren Schicht 3 (IP)
Adressen herausfinden. Es gibt uns aber noch nicht die Moeglichkeit, eine
bestimmte Resource, z.B. eine HTML Datei, zu adressieren. Auch gibt der FQDN
alleine noch nicht an, auf welche Weise, also ueber welches Schicht 7 Protokoll,
der FQDN bzw. dann die Ressource geholt werden soll. Hierfuer nutzen wir
*Uniform Resource Locator*s (*URL*s). Diese haben die allgemeine Form:

```
<protocol>://[<username>[:<password]@]<fqdn>[:<port>][/<path>][?<query>][#<fragment>]
```

Hier haben wir also:

* *protocol*: das verwendete Anwendungsprotokoll: HTTP, FTP oder SMTP.
* *username[:password]@* ermoeglicht die optionale Angabe eines Benutzernamens
  und Kennworts. Das mit dem Kennwort ist natuerlich eine schlechte Idee, weil
  das vollkommen unverschluesselt uebertragen werden wuerde.
* *fqdn*: der vollqualifizierte Domain Name, ueblicherweise ohne `root` Punkt,
  der das Ziel auf Schicht 3 identifiziert.
* *port*: optional der Port, auf welchem ueber dem jeweiligen Protokoll mit dem
  Client kommuniziert werden soll. Per Default ist das immer der
  *well-known port* des Protokolls (z.B. Port 80 bei HTTP). Diesen muss der
  Browser (bzw. die Anwendung) selbst wissen.
* */path* ermoeglicht die Angabe eines Pfads auf dem Zielrechner relativ zur
  Wurzel seiner Verzeichnisstruktur.
* *?query* erlaubt die Uebergabe von Key-Value Paaren der Form
  `key=value`. Mehrere solcher Angaben koennen dabei via `&` konkateniert
  werden.
* *#fragment* macht es moeglich, einzelne Abschnitte einer Datei zu
  referenzieren.

## HTTP

Das beliebteste Protokoll auf Schicht 7, das zur Uebertragung von Daten zwischen
Servern und Clients genutzt wird, ist das *HyperText Transfer Protokoll*
(*HTTP*). Es definiert einen Request/Reply Mechanismus und definiert, welche
Anfragen ein Client stellen darf und wie ein Server darauf zu reagieren hat. Mit
einem HTTP Kommando (z.b. `GET`, `PUT`, `POST`) kann *hoechstens ein Objekt*,
z.B. eine Textdatei oder eine Grafik, uebertragen werden. HTTP ist hierbei ein
vollkommen *textbasiertes* Protokoll, d.h. alle Anfragen werden in ASCII
kodiert. Eingehende HTTP Verbindungen hierbei werden auf dem well-known Port TCP
*80* erwartet.

Der heutige HTTP Standard ist in Version 2.0 und wurde 2015 auf Basis eines
Protokolls von Google <3 veroeffentlicht. Der allererste Standard, HTML 1.0,
hatte noch ein grosses Problem. Es schlug naemlich vor, dass nach jedem
Anfrage/Abfrage Paar die assoziierte Schicht 4 (TCP) Verbindung wieder abgebaut
werden soll. Das machte bei der Geburt des Internets noch Sinn, als Websiten
noch nicht viel Content hatten. Heutzutage waere es aber absolut katastrophal,
fuer jedes kleine Element auf einer Website (z.B. jedes kleine Bild) einen TCP
Verbindungsaufbau und -abbau haben zu muessen. Deswegen wuerde mit HTTP 1.1
festgelegt, dass TCP Verbindungen auch laenger offen sein duerfen.

Es gibt grundsaetzlich zwei Arten von HTTP Nachrichten: Requests und
Responses. Requests gehen dabei vom Client zum Server und haben immer eine
*Method*. Die Method gibt die Aktion an, die sich der Client vom Server
wuenscht. Das koennte z.B. eine `GET` Aktion fuer eine Ressource auf der Website
sein. Allgemein gibt es folgende moegliche HTTP Methoden:

* `GET`: Requested eine Ressource auf dem Server. Diese Anfrage sollte keine
  Nebeneffekte haben.
* `HEAD`: Wir eine `GET` Anfrage, wobei aber nur der Header zurueckgegeben
  wird. Das ist nuetzlich, um Metainformationen zu erhalten, ohne die ganze
  Payload transportieren zu muessen.
* `POST`: Erlaubt es dem Client, dem Server Daten zu uebergeben, die dann meist
  auf diesem gespeichert werden sollen.
* `PUT`: Der Client moechte eine Ressource (URI) mit neuen Daten updaten oder
  die entsprechende Ressource erstellen, falls es sie noch nicht gibt.
* `DELETE`: Loescht die entsprechende Ressource auf dem Server.
* `TRACE`: Echoed den Request, damit der Client sehen kann, welche
  Veraenderungen an der HTTP Nachricht Server auf dem Weg dorthin machen.
* `OPTIONS`: Sagt dem Client, welche Arten von Anfragen er machen darf.

Der *Pfad* des URLs wird hierbei besonders wichtig, da er die angefragte Datei
genau beschreiben kann. Auch die Query Parameter sind fuer bestimmte APIs
notwendig. Ein HTTP Request hat auch immer einen *HTTP Header*, welcher weitere
Informationen enthaelt, naemlich:

* Den FQDN des angefragten Hosts,
* Zeichensatz und Encoding, in dem die Antwort erwartet wird (Accept-Charset)
* Von wo eine Anfrage kam (z.B. bei Weiterleitung). Wird dann *Referer* (falsch
  geschrieben) genannt.
* Den *User-Agent*, was die verwendete Client-Software sein soll.

HTTP *Responses* sagen dann zunaechst als aller Erstes immer aus, was der
*Status* der Nachricht ist. Dieser Status besteht aus einem numerischen Code
(dem Status Code, z.B. 404) sowie ein Text zur Angabe von Fehlern. Moegliche
Status Codes sind hierbei die folgenden, wobei die erste der drei Ziffern jedes
Codes immer die Klasse angibt:

* 1xx: *Informational* Codes. Sagen, dass der Request erhalten wurde und die
  Verbindung weiterschreiten kann. Beispiele:
  - 101: Switching Protocols
* 2xx: *Success*. Diese bestaetigen eine erfolgreiche Uebertragung. Beispiele:
  - 200: OK
  - 201: Created (after POST)
* 3xx: *Redirection*. Bedeutet, dass weitere Aktionen vom Client benoetigt
  werden um den Request vollkommen abzuschliessen. Beispiele:
  - 301: Moved Permanently (die angefragte Ressource ist nun woanders)
  - 302: Redirect (`location` gibt neuen URL an, wo der Client anfragen soll)
  - 305: Use Proxy (benutze bitte den (reverse) Proxy, der im `location` Header steht)
* 4xx: *Client Error*. Die Anfrage war auf Grund eines Fehlers vom Client nicht
  verarbeitbar. Das kann Syntaxgruende, aber auch semantische Gruende
  haben. Beispiele:
  - 404: Page Not Found
  - 400: Bad Request (Server hat die Anfrage einfach nicht verstanden)
  - 403: Forbidden
  - 418: I'm a Teapot
* 5xx: *Server Error*. Die Anfrage koennte auf Grund eines Fehlers beim Server
  nicht korrekt beantwortet oder verarbeitet werden. Beispiele:
  - 500: Internal Error (Unerwarteter Fehler beim Server)
  - 502: Bad Gateway (Server bekam eine falsche Antwort von einem anderen
    Server)
  - 503: Service Unavailable (temporarily down)

Dann haben HTTP Responses, wie Requests, auch Header die weitere Optionen
beeinhalten koennen. Das ganze wird dann via CRLF (Carriage Return Line Feed;
`\r\n`) von den eigentlichen Nutzdaten abgetrennt.

### Verschluesselung

HTTP selbst kennt keine Verschluesselungsmechanismen fuer uebertragene
Daten. Man kann aber zwischen Transport- und Anwendungsschicht (insbesondere
Darstellungsschicht) Verschluesselungsprotokolle nutzen.

Beispielsweise gibt es HTTP__S__, welches das TLS (Transport Layer Security) auf
der Sitzungs-/Darstellungsschicht (5/6) nutzt. Bei HTTPS wird der ganze HTTP
Request, inklusive Header und Daten (Payload) ver- und entschluesselt. Einzig
die Website (im Sinne ihrer IP Adresse in der Transportschicht PDU) sowie Port
können nicht geschuetzt werden, einfach weil TLS auf einer höheren Ebene
arbeitet. Als Loesung gibt es beispielsweise IPSec
(https://en.wikipedia.org/wiki/IPsec), welches schon auf Schicht 3
verschluesselt, und somit auch die gesamte IP Payload, inklusive Destination und
Source IP, schuetzt.

Fuer HTTPS wird TCP Port 443 verwendet anstelle von TCP Port 80, zur
Unterscheidung von HTTP.

https://en.wikipedia.org/wiki/HTTPS

## Proxy

Proxies sind Rechner, die zwischen einem Client und einem Server stehen, auf
welchen der Client zugreifen will. Dann hat der Client also keine Ende-zu-Ende
Verbindung zum Server, sondern nur indirekt ueber den Proxy. Anstelle den
Server direkt zu kontaktieren, wird der Client seine *HTTP* Anfragen zunaechst
an den Proxy senden. Dieser verwendet dann oftmals andere Well-Known
Portnummern, z.B. 3128 anstelle von 80.

Sobald der Proxy dann eine Anfrage von einem Client erhaelt, baut er selbst eine
Verbindung zum Zielserver auf und stellt die Anfrage des Clients. Der Server
selbst denkt dann, dass er mit dem Proxy kommuniziert und nicht mit dem
Client. Der Proxy wuerde die entsprechende Antwort dann wieder zum Client
senden.

Eine wichtige Aufgabe von Proxies ist im Weiteren, die Anfragen der Clients dann
zu *cachen*. Das heisst, dass ein Proxy die Antwort auf eine Anfrage eines
Clients fuer einen gewissen Zeitraum speichert, sodass darauffolgende,
identische Anfragen (auch von anderen Clients) dann nicht mehr zum Server gehen
muessen. Der Proxy kann einfach direkt die im Cache liegende Antwort
zurueckgeben (z.B. ein Bild, das mittels HTTP `GET` angefragt wurde).

Proxies koennen auch transparent arbeiten und Netzwerkpakete filtern. Das ist in
Schulen oder Firmennetzen haeufig anzutrefen, um die Produktivitaet von
Mitarbeitern zu maximieren, indem man sie nicht auf Spieleseiten laesst. Hierbei
muss ein Administrator den Proxy aber manuell einrichten (siehe unten).

Eine weitere Eigenschaft, die mit Proxies assoziiert ist, ist *Anonymitaet*. Da
der Server den Client, der ueber einen Proxy mit dem Server kommuniziert, nicht
sieht, ist dessen Identitaet nicht sichtbar fuer den Server. Der Client ist also
gewissermassen anonym. Das ist beispielsweise das Konzept hinter TOR (The Onion
Router). Requests werden durch mehrere, global verteilte Proxies geroutet, damit
am Ende die Identitaet des urspruenglichen Clients geschuetzt ist (durch
Verzwiebelung der Anfrage, also zwiebelaehnliche Verdeckung).

Proxies, wie sie bisher beschrieben wurden, sind eigentlich nur eine bestimme
Art von Proxy. Man nennt sie *Forward Proxies*. Neben *Forward Proxies* gibt es
auch noch *Reverse Proxies*. Der hauptsaechliche Unterschied hierbei liegt
darin, wer den Proxy konfiguriert:

1. Forward Proxies:
   * Muessen vom Client konfiguriert werden.
   * Dienen manchmal dazu, Traffic zu kontrollieren. Beispielsweise koennte eine
     Schule jeglichen Traffic, der nicht durch ihren Proxy geht, einfach
     unterbinden (durch eine Firewall an ausgehenden Routern). D.h., dass alle
     Requests notwendigerweise durch den Proxy gehen muessen (das wuerden
     Netzwerkadministratoren auf allen PCs in der Schule so
     konfigurieren). D.h. wiederum, dass man auf dem Proxy dann alle Requests
     inspizieren und pruefen kann, ob man den Request ueberhaupt zulassen
     soll. Geht der Request beispielsweise auf `playboy.com`, so wuerde man ihn
     verhindern und das HTML fuer eine schulinterne Fehlerwebsite
     zurueckgeben. Andere Requests koennte man dann einfach durchlaufen lassen.
   * Manchmal dienen sie auch dazu, das obere gerade zu umgehen. Wenn ich
     beispielsweise dennoch `playboy.com` aufrufen moechte, koennte ich mich mit
     einem externen Proxy Server `definitely-not-a-proxy-for-playboy.com`
     verbinden, der meine Requests dann einfach zu `playboy.com` weiter
     routet. Der obige ("Firewall") Proxy wuerde dann ja nicht `playboy.com`
     sehen, sondern nur die Adresse dieses Forward Proxies. Insofern wuerde er
     den Traffic nicht unterbinden.
   * Insbesondere bleibt der Client hinter dem Proxy verborgen. Der Server sieht
     nur den Proxy. Hierdurch kann der Client also eine gewisse Anonymitaet
     erreichen.

2. Reverse Proxies:
   * Werden vom *Server* konfiguriert (Server-Side Konzept).
   * Ein Server kann einstellen, dass Requests, die eigentlich an ihn gerichtet
     sind durch einen *Reverse Proxy* gehen muss (der Domain Name bzw. die IP
     Adresse wuerde wohl den Proxy, nicht den Server identifizieren).
   * Das koennte z.B. den Sinn haben, Traffic fuer eine beliebte Website zu
     load-balancen. Das ist gerade was Content Delivery Networks (CDNs) wie
     CloudFront oder Akamai machen. Sie bieten den Dienst an, global Reverse
     Proxies aufzustellen, sodass Requests fuer, z.B. `apple.com`, an den
     geographisch naechsten Proxy geroutet werden. Dieser erfuellt den Request
     dann, indem er entweder eine gecachede Antwort zurueckgibt, oder selbst
     einen Request an den Server von Apple im Hintergrund stellt.
   * Reverse Proxies sind auch dazu nuetzlich, Verschluesselung uber TSL einfach
     zu "poolen". Sagen wir, der Reverse Proxy hat 10 Server im
     Hintergrund. Anstelle die 10 Server mit TSL auszustatten (was entsprechende
     Zertifikate benoetigit), stattet man einfach den einen Reverse Proxy mit
     TLS aus und hat somit jegliche Client-Server Kommunikation bis zum Reverse
     Proxy geschuetzt. Dieser Reverse Proxy waere dann wohl im selben lokalen
     Netzwerk wie die Server, also muesste man sich um den restlichen Weg nicht
     mehr kuemmern.

Das wichtige ist, dass Forward Proxies als Zwischenstationen von Client-Requests
auf den Server dienen, wohingegen Reverse Proxies als Zwischenstationen von
Server-Responses fuer den Client dienen. Insbesondere sieht der Client bei
Reverse-Proxies den Server nicht und weiss gar nicht, dass dieser
existiert. Fuer den Client ist der Reverse-Proxy der Server. Nur der
Reverse-Proxy weiss, dass er die Requests an die eigentlichen Server im
Hintergrund weiterleiten muss (und Sachen dann cached).

Forward Proxy:
```
A ---|
     |
B --- Proxy --> Server
     |
C ---|
```

Reverse Proxy:
```
                    Server A
                  /
                 /
Client --> Proxy -- Server B
                 \
				  \ Server C
```

https://www.quora.com/Whats-the-difference-between-a-reverse-proxy-and-forward-proxy

## SMTP

Das *Simple Mail Transfer Protocol* (*SMTP*) ist das gaengigste Protokoll zum
Versenden von E-Mails im Internet. Es ist ein textbasiertes Protokoll und hat
drei grundlegende Aktoren:

1. *Mail User Agents* (*MUAs*), welche E-Mail versenden und empfangen wollen.
2. *Mail Transfer Agents* (*MTAs*), welche E-Mail durch das Internet routen. MTA
   sind hierbei ein Synonym fuer "Mailserver". Solche Server benutzen oft Postfix
   (http://www.postfix.org) als Software. Sie vermitteln Mail zwischen MUAs
   (z.B. Apple Mail) und MDAs (z.B. `imap.gmail.com`).
3. *Mail Delivery Agents* (*MDA*), welche E-Mails fuer den Empfaenger in
   *Mailboxen* speichern. Sie erhalten also E-Mails von MTAs und platzieren sie
   in eine Mailbox fuer den User, sodass dieser dann ueber POP oder IMAP darauf
   zugreifen kann.

```
Sender ==> MUA ==SMTP==> MTA ==SMTP==> ... ==SMTP==> MTA
	   ==SMTP==> MDA ==POP/IMAP==> MUA ==> Empfaenger
```

https://en.wikipedia.org/wiki/Message_transfer_agent
http://ccm.net/contents/116-how-email-works-mta-mda-muaA

Es gibt wie oben bemerkt also zwei Protokolle, ueber welche man auf E-Mails
zugreifen kann, die ein MDA speichert:

1. Das *Post Office Protocol* (*POP*): POP ist momentan in Version 3 und ein
   vergleichsweise viel simpleres Protokoll als IMAP. Es dient mehr oder weniger
   nur dazu, E-Mail in der Mailbox des Mail-Servers zwischenzuspeichern, bis
   *ein* Client die Nachricht downloaded. Per Default wird die Nachricht dann
   vom Mail-Server geloescht. Das ist insofern bloed, wenn mehr als ein Client,
   also mehr als ein Geraet, auf die selbe Mailbox zugreifen moechte. Das waere
   beispielsweise so, wenn man von seinem Laptop und von seinem Smartphone
   zugreifen will. Offensichtlich ist das heutzutage der Regelfall. Hierfuer
   gibt es dann noch die Option, dass die Mail auf dem Server bleiben kann. Aber
   das wichtige ist, dass es ueberhaupt *keine Synchronisation zwischen dem
   Server und einzelnen Geraeten* gibt. Wenn man auf einem Geraet eine Mail
   downloaded, dann ist das auf anderen Geraeten noch nicht so. Wenn man sie auf
   einem Geraet loescht, ist *sie noch nicht auf dem Server geloescht* und auch
   sicher nicht auf anderen Geraeten. Auch hat POP kein Konzept von Flags fuer
   "read", "replied to", "deleted" oder "important".

2. Das *Internet Message Access Protocol* (*IMAP*) ist viel flexibler und
   komplexer als POP. Insbesondere hat es viel staerkere Synchronisation
   zwischen Server und Client. Clients cachen Nachrichten nur. Somit sind
   Mail-Clients (MUAs) als eher Interfaces zu diesen Servern. Jegliche
   Veraenderungen werden auf dem Server durchgefuehrt. Das bedeutet dann aber
   auch, dass diese Aenderungen auf allen Clients der Mailbox, also heutzutage
   auf allen Geraeten, reflektiert sind. Loesche ich meine E-Mail auf meinem
   Smartphone, so wird sie auch auf dem Server geloescht und das sehe ich dann
   auch auf meinem Laptop. Auch hat IMAP ein Konzept von Flags ("read", "replied
   to").

http://www.pop2imap.com

Um einen Mailserver (MTA) im DNS zu registrieren, muss man in Zone Files entsprechende
*MX* Resource Records eintragen, diese sehen beispielsweise so aus:

```
tum.de. 3600 IN MX 100 postrelay2.lrz.de.
```

1. Hier steht links der FQDN, fuer dessen Domain der Mailserver zustaendig ist.
2. Der TTL fuer DNS Clients
3. Die Record Class `IN`
4. `MX` als Record Type.
5. Dann folgt die Resource Data, bestehend aus:
   1. Einer *Priority* fuer den Mailserver. Wenn man mehrere hat, kann man diese
      nach Priority sortieren, sodass Mailserver mit hoeherer Priority zuerst
      angefragt werden (hierbei bedeuten kleinere Werte hoehere Priority).
   2. Dem FQDN des Mailservers.

Eine E-Mail geht also von MTA zu MTA, wobei jeder MTA selbst wieder MX-Eintraege
hat. So kann eine E-Mail also von `gmail.com` zu `goldsborough.me` Adressen
geroutet werden.

## FTP

Ein weiteres Schicht 7 Protokoll ist das *File Transfer Protocol* (*FTP*). Es
wird genutzt, um Binaerdaten, also Dateien, zwischen Server und Client zu
uebertragen. FTP verwendet hierbei Well-Known Port TCP 21.

Eine Eigenheit von FTP ist, dass es *zwei* getrennte TCP-Verbindungen fuer eine
Verbindung nutzt:

1. Die erste Verbindung ist der *Kontrollkanal*, zur Uebermittlung von Befehlen
   und Statuscodes zwischen Client und Server. Hier wird beispielsweise
   festgelegt, ueber welche Ports Daten ausgetauscht werden sollen. Generell
   erfolgt ueber diesen Kanal der Verbindungsaufbau und -abbau. Auch werden
   Fehlermeldungen ueber diesen Kanal gesendet. Da dieser Kanal ueber mehrere
   Datenuebertragungen hinweg erhalten bleibt, spricht man von FTP als ein
   *stateful* Protokoll. Ueber diesen Kanal muessen auch Benutzername und
   Passwort, welche bei FTP immer benoetigt werden, ausgetauscht werden. Falls
   ein Server auch ohne Benutzername und Passwort Zugriffe erlaubt (wenn auch
   mit restriktiven Permissions), dann gibt es oft noch einen Benutzernamen
   `anonymous` oder `ftp`, der kein (bzw. leeres) Passwort benoetigt.

2. Die zweite Verbindung ist der *Datenkanal*, welcher fuer die Uebertragung der
   eigentlichen Daten genutzt wird.

Es gibt im Weiteren zwei Modi, in welchen FTP ausgefuehrt werden kann, *active*
und *passive* Mode:

* In beiden Faellen baut der Client den Kontrollkanal zum Server auf TCP 21 auf.
* Im *active mode* koordiniert der Client den Datenkanal. Er uebermittelt dem
  Server dann durch das `PORT`-Kommando ueber den Kontrollkanal eine Portnummer
  und IP Adresse mit, auf welcher sich der *Server* dann mit dem *Client*
  verbindet. Der Server `connect`ed sich dann also ueber TCP Quellport 20 auf
  den angegebenen Port und Adresse des Clients, welcher dort `listen()`ed. Ueber
  diesen Kanal werden Daten dann ausgetauscht.
* Im *passive mode* sendet der Client das Kommando `PASV` ueber den
  Kontrollkanal und erhaelt vom Server eine IP Adresse und zugehoeriegen Port,
  ueber welchen der Client dann eine neue Verbindung *zum Server* aufbaut.

Wieso gibt es diese zwei Modi? Betrachten wir, welche Probleme es geben kann,
wenn sich der Client im *active mode* mit dem Server verbinden will:

1. Der Client ist hinter NAT und hat lokale (private) Adresse `10.0.0.7`.
2. Der Client sendet `PORT 10,0,0,7,123,64` um dem Server mitzuteilen, dass der
   Client nun auf Port `31,552` und IP `10.0.0.7` `listen`ed(). Der Client
   kriegt vom NAT ja nichts mit. Wie man sieht, werden diese Daten byteweise in
   Dezimal und Komma-getrennt uebertragen. Deswegen ist der Port $123 * 256 + 64
   = 31,552$.
3. Beim Router wird die IP Adresse dieses Pakets von der privaten Adresse
   `10.0.0.7` auf eine global eindeutige Adresse des Routers geaendert.
3. Der Server erhaelt die Port Information und versucht nun, sich mit dem Client
   an Adresse `10.0.0.7` und Port `31,552` zu verbinden.
4. Er scheitert, weil die IP Adresse im `PORT` Kommand natuerlich nicht geandert
   wurde, und noch immer die private Adresse `10.0.0.7` enthielt. Als private
   Adresse wird sie bzw. das Paket natuerlich nicht geroutet.

In diesem Fall erkennt man, dass Passive Mode notwendig waere, damit der Server
dem Client eine Verbindungsmoeglichkeit anbietet und nicht andersrum. Waere der
Server hinter NAT, wuerde natuerlich wiederum nur Active Mode
funktionieren. Sind beide hinter NAT, hat man ein Problem. Ausser der Server
richtet dann explizit einen Forwarding Eintrag fuer die Kontrollkanaladresse
ein. Beim Programmieren wuerde man dann fuer jeden `socket()` Aufruf sowieso
einen eigenen Socket fuer jede eingehende Verbindung erhalten, insofern wuerde
das gut funktionieren. Ein Socket ist ja durch das Fuenf-Tupel identifiziert,
also wuerde jeder Socket eine dedizierte Verbindung mit einem Client ueber
denselben Port bedeuten.


\end{document}
